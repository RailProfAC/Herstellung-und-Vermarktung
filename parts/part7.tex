% !TEX root = SFV-15014_HuV.tex
\section{Projektplanung}
%\sectionpage

\frame{\frametitle{Was bedeutet Projektplanung?}
\framesubtitle{\textit{[...] I have always found that plans are useless but planning is indispensable.}  \\ \hspace{9cm} {Dwight D. Eisenhower}}
\begin{columns}[t] 
     \begin{column}[T]{6cm} 
     	\begin{itemize}
     		\item Organisation verschiedener Projektaspekte:
		\begin{itemize}
		\item Projektumfang
		\item Arbeitspakete
		\item Projektrisiken
		\item Finanz- und Kostenplanung
		\item Einsatzmittelplanung
		\item Materialplanung
		\item Berichten des Fortschritts
		\item Verfolgen von Abweichungen
		\end{itemize}
		\item Begleitung der Projektdurchf\"uhrung
		\begin{itemize}
		\item Ggf. Plananpassung, Krisenmanagement
		\end{itemize}
     	\end{itemize}
     \end{column}
     	\begin{column}[T]{6cm} 
         	\begin{center}
            		\includegraphics[width=0.9\textwidth]{NasaPP}\\
		\tiny {\color{gray} NASA Project Planning Process}
        		\end{center}
     \end{column}
 \end{columns}
}

\frame{\frametitle{Projekt: Mars Voyage}
\framesubtitle{Zur Veranschaulichung der Projektplanungsmethodiken dient das recht engagierte Ziel, innerhalb von 10 Monaten Astronauten zum Mars zu senden.}
\begin{center}
\begin{tabular}{|c|c|c|c|}
\hline
Step & Description & Estimated Effort & Predecessor\\ \hline
0 & Kick Off & Milestone & -\\ \hline
A & Build spaceship & 2 & 0\\ \hline
B & Equip spaceship & 1 & A\\ \hline
C & Test spaceship & 2 & B\\ \hline
D & Train astronauts & 4 & 0\\ \hline
E & Flight to Mars & 5 & C, D\\ \hline
1 & Mars Landing & Milestone & E\\ \hline
\end{tabular}
\end{center}
}

%\frame{\frametitle{M\"ogliche Sequenzen von Arbeitspaketen}
%\framesubtitle{}
%\begin{itemize}
%\item Abh\"angig
%\item Unabh\"angig
%\item Ineinandergreifend
%\end{itemize}
%}


\frame{\frametitle{Kritischer Pfad}
\framesubtitle{Der kritische Pfad zeigt unter parallelen Aktivit\"aten die (derzeit) terminbestimmende Kombination an.}
     	\begin{itemize}
     		\item Voraussetzungen:
		\begin{itemize}
		\item Liste der Arbeitspakete (z.B. aus WBS)
		\item Dauer der Aufgaben
		\item Abh\"angigkeiten
		\item Zwischen-/Endpunkte, z.B. Meilensteine, Deliverables
		\end{itemize}
     	\end{itemize}
             	\begin{center}
		\begin{tikzpicture}[->,>=stealth',shorten >=1pt,auto,node distance=2cm,
  thick,main node/.style={circle,fill=blue!20,draw,font=\sffamily\Large\bfseries}]

  \node[main node, fill = gray] (0) {0};
  \node[main node] (A) [below right of=0] {A:2};
  \node[main node] (B) [right of=A] {B:1};
  \node[main node] (C) [right of=B] {C:2};
  \node[main node] (D) [above of=B] {D:4};
  \node[main node] (E) [above right of=C] {E:5};
  \node[main node, fill = gray] (1) [right of=E] {1};


  \path[every node/.style={font=\sffamily\small}]
    	(0) edge[draw = red!80!black] node {} (A)
	(A) edge[draw = red!80!black] node {} (B)
	(B) edge[draw = red!80!black] node {} (C)
	(0) edge node {} (D)
	(C) edge[draw = red!80!black] node {} (E)
	(D) edge node {} (E)
	(E) edge[draw = red!80!black] node {} (1);
	%(0) edge [bend left] node[above]{10} (1);
\end{tikzpicture}
	        		\end{center}
  }

\frame{\frametitle{Gantt-Diagramme}
\framesubtitle{}
\begin{columns}[t] 
     \begin{column}[T]{6cm} 
     	\begin{itemize}
     		\item Tabellenform:
		\begin{itemize}
		\item Erste Zeile: Zeitachse
		\item Erste Spalte: Aktivit\"aten
		\item Aktivit\"aten als Balken
		\end{itemize}
		\item Weitere Elemente:
		\begin{itemize}
		\item Gruppen
		\item Meilensteine
		\item Abh\"angigkeiten
		\end{itemize}
		\item Un\"ubersichtlich f\"ur gro{\ss}e Projekte
		\item Toolunterst\"utzung:
		\begin{itemize}
		\item MS Project
		\item Project Libre
		\end{itemize}
     	\end{itemize}
     \end{column}
     	\begin{column}[T]{6cm} 
         	\begin{center}
            		\begin{ganttchart}[vgrid, hgrid, 
		%today = 4, 
		progress=today,
		progress label text=\relax,
		today=3,
		y unit chart = 5 mm,
		x unit = 4 mm,
		bar/.append style={fill=blue!80!black},
		%chart element start border=right
		]{1}{12}
			%\gantttitle{Mars voyage}{12} \\
			\gantttitlelist{1,...,12}{1}\\
			\ganttgroup{}{2}{11} \\
			\ganttmilestone{0}{1}\\
			\ganttbar{A}{2}{3} \\
			\ganttbar{B}{4}{4} \\
			\ganttbar{C}{5}{6} \\
			\ganttbar[progress = 90]{D}{2}{5} \\
			\ganttbar{E}{7}{11} \\
			\ganttmilestone{1}{11}
			\ganttlink{elem1}{elem2}
			\ganttlink{elem1}{elem5}
			\ganttlink{elem2}{elem3}
			\ganttlink{elem3}{elem4}
			\ganttlink{elem4}{elem6}
			\ganttlink{elem4}{elem6}
			\ganttlink{elem5}{elem6}
			\end{ganttchart}

        		\end{center}
     \end{column}
 \end{columns}
}

\frame{\frametitle{Projektplanung im Gantt-Diagram}
\framesubtitle{}
         	\begin{center}
            		\begin{ganttchart}[vgrid, hgrid, 
		%today = 4, 
		progress=today,
		progress label text=\relax,
		today=3,
		y unit chart = 7 mm,
		bar/.append style={fill=blue!80!black},
		%chart element start border=right
		]{1}{12}
			%\gantttitle{Mars voyage}{12} \\
			\gantttitlelist{1,...,12}{1}\\
			\ganttgroup{Get there}{2}{11} \\
			\ganttmilestone{0: Kick Off}{1}\\
			\ganttbar{A: Build spaceship}{2}{3} \\
			\ganttbar{B: Equip spaceship}{4}{4} \\
			\ganttbar{C: Test spaceship}{5}{6} \\
			\ganttbar[progress = 90]{D: Train astronauts}{2}{5} \\
			\ganttbar{E: Flight to Mars}{7}{11} \\
			\ganttmilestone{1: Mars landing}{11}
			\ganttlink{elem1}{elem2}
			\ganttlink{elem1}{elem5}
			\ganttlink{elem2}{elem3}
			\ganttlink{elem3}{elem4}
			\ganttlink{elem4}{elem6}
			\ganttlink{elem4}{elem6}
			\ganttlink{elem5}{elem6}
			\end{ganttchart}
        		\end{center}
}

%\frame{\frametitle{Netzplanmethode}
%\framesubtitle{}
%\begin{columns}[t] 
%     \begin{column}[T]{6cm} 
%     	\begin{itemize}
%     		\item Graph, Elemente:
%		\begin{itemize}
%		\item Knoten
%		\item Kanten
%		\end{itemize}
%		\item Darstellungen u.a.:
%		\begin{itemize}
%		\item Vorgangsknoten-Netzplan
%		\item Vorgangspfeil-Netzplan
%		\end{itemize}
%     	\end{itemize}
%     \end{column}
%     	\begin{column}[T]{6cm} 
%         	\begin{itemize}
%		\item Daten eines Vorgangs:
%		\begin{itemize}
%		\item Aus Vorw\"artsplanung:
%		\begin{itemize}
%		\item Fr\"uhester Anfangs-, Endzeitpunkt
%		\end{itemize}
%		\item Aus R\"uckw\"artsplanung:
%		\begin{itemize}
%		\item Sp\"atester Anfangs-, Endzeitpunkt
%		\end{itemize}
%		\item Dauer
%		\item Pufferzeiten
%		\end{itemize}
%		\end{itemize}
%     \end{column}
% \end{columns}
%\begin{center}
%            		\includegraphics[width=0.7\textwidth]{Netzplan}
%			\rotatebox{90}{\tiny \color{gray} Quelle: Wikipedia/Skaler}
%        		\end{center}
%}
%
%\frame{\frametitle{Design Structure Matrix}
%\framesubtitle{}
%\begin{columns}[t] 
%     \begin{column}[T]{6cm} 
%     	\begin{itemize}
%     		\item Modellierung des Informationsflusses
%		\item Erleichtert:
%		\begin{itemize}
%		\item Finden von (sequentiellen) Aufgaben-Clustern
%		\item Entdecken von Interationen
%		\end{itemize}
%		\item Eintr\"age:
%		\begin{itemize}
%		\item Hauptdiagonale: Dauer
%		\item Unterhalb: Sequentielle Arbeitspakete
%		\item Oberhalb: Iterationen
%		\end{itemize}
%		\item Cluster:
%		\begin{itemize}
%		\item Bilden eine Submatrix
%		\item K\"onnen zusammengefasst werden
%		\end{itemize}
%     	\end{itemize}
%     \end{column}
%     	\begin{column}[T]{6cm} 
%         	\begin{center}
%            		\only<1-2>{\begin{tabular}{c|c|c|c|c|c|c|c|}
%                            & 0 & A & B & C & D & E & 1\\ \hline
%                            0 & 0 &  &  &  &  &  & \\ \hline
%                            A &  X &\only<2>{\cellcolor{red!25} \color{red!80!black}} 2 & \only<2>{\cellcolor{red!25}} & \only<2>{\cellcolor{red!25}} &  &  & \\ \hline
%                            B &  & \only<2>{\cellcolor{red!25} \color{red!80!black}}X & \only<2>{\cellcolor{red!25}\color{red!80!black}}1 & \only<2>{\cellcolor{red!25}} &  &  & \\ \hline
%                            C &  &  \only<2>{\cellcolor{red!25}} & \only<2>{ \cellcolor{red!25}\color{red!80!black}}X & \only<2>{\cellcolor{red!25}\color{red!80!black}}2 &  &  & \\ \hline
%                            D & X &  &  &  &\only<2>{\cellcolor{blue!25} \color{blue!80!black}} 4 &  & \\ \hline
%                            E &  &  &  & X & X & 5 & \\ \hline
%                            1 &  &  &  &  & X & X & 0\\ 
%                            \end{tabular}}
%                            \only<3>{\begin{tabular}{c|c|c|c|c|c|}
%                            & 0 & ABC & D & E & 1\\ \hline
%                            0 & 0 &  &  &  & \\ \hline
%                            ABC &  X &\cellcolor{red!25} \color{red!80!black} 5 &  &  & \\ \hline
%                            D & X  &  &\cellcolor{blue!25} \color{blue!80!black} 4 &  & \\ \hline
%                            E &  & X & X & 5 & \\ \hline
%                            1 &  &  & X & X & 0\\ 
%                            \end{tabular}
%                            }
%        		\end{center}
%     \end{column}
% \end{columns}
%}