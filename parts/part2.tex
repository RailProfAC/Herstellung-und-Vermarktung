% !TEX root = SFV-15014_HuV.tex
\section{Marketing}
\sectionpage

\subsection{Marktsegmentierung}
\subsectionpage

\frame{\frametitle{Warum Marktsegmentierung?}
\framesubtitle{Any customer can have a car painted any colour that he wants so long as it is black. - Henry Ford}
\begin{itemize}
\item Marktidentifizierung
\begin{itemize}
		\item Abgrenzung des relevanten Gesamtmarktes
		\item Bestimmung der relevanten Teilmärkte
		\item Auffinden vernachlässigter Teilmärkte (Marktlücken, Marktnischen)
\end{itemize}
\item Rechtzeitige Beurteilung von Neueinführungen der Konkurrenz und rechtzeitiges Ergreifen von Gegenmaßnahmen
\item Beurteilung der eigenen Markenpositionierung im Vergleich zur Positionierung der Konkurrenzprodukte
\item Richtige Positionierung von Neuprodukten
\item Fundierte Prognose der (segmentspezifischen) Marktentwicklung
\item Optimale Allokation des Budgets auf einzelne Segmente
\item Erhöhung der Zielerreichungsgrade 
\item Preisfindung
\end{itemize}
}

\offslide{Segmentierungskriterien f\"ur Schienenfahrzeuge und ihre Komponenten}

\subsection{Marktgrundlagen}
\subsectionpage

\frame{\frametitle{Gr\"o{\ss}e und Entwicklung des Marktes}
\framesubtitle{}
\begin{columns}[t] 
     \begin{column}[T]{4cm} 
     	\begin{itemize}
     		\item Gesamtmarkt Schienenfahrzeuge weltweit: 47 Mrd EUR (Stand 2012)
		\item Dominierende Teilm\"arkte:
		\begin{itemize}
		\item Asien
		\item Europa
		\end{itemize}
		\item Relevante Fahrzeugsegmente variieren lokal stark
     	\end{itemize}
     \end{column}
     	\begin{column}[T]{8cm} 
         	\begin{center}
		\begin{tikzpicture}%[scale = 0.9]
		\begin{axis}[ybar stacked,
		ylabel={bn EUR/a},
		legend style={at={(0.5,-0.4)},
		anchor=north, legend columns=3},
		symbolic x coords={EMEA, ASEAN, CIS, E. Europe, NAFTA, RoA, W. Europe},
		xtick=data,
		x tick label style={rotate=45,anchor=east},
		title={Conventional Rail Market Size}]]
		\addplot[fill = RYB1, draw = RYB1!50!black] coordinates
			{(EMEA,0.220) (ASEAN,1.367) (CIS, 1.219) (E. Europe, .248) (NAFTA, .192) (RoA, .037) (W. Europe, .421)};
		\addplot[fill = RYB2, draw = RYB2!50!black] coordinates
		{(EMEA,.294) (ASEAN,.823) (CIS, .196) (E. Europe, .103) (NAFTA, .992) (RoA, .218) (W. Europe, .244)};
		\addplot[fill = RYB3, draw = RYB3!50!black] coordinates
		{(EMEA,.012) (ASEAN, .067) (CIS, .212) (E. Europe, .026) (NAFTA, .027) (RoA, .005) (W. Europe, .128)};
		\addplot[fill = RYB4, draw = RYB4!50!black] coordinates
		{(EMEA,.172) (ASEAN,.330) (CIS, .686) (E. Europe, .196) (NAFTA, .383) (RoA, .209) (W. Europe, .501)};
		\addplot[fill = RYB5, draw = RYB5!50!black] coordinates
		{(EMEA,.129) (ASEAN, .932) (CIS, 2.671) (E. Europe, .281) (NAFTA, 1.337) (RoA, .125) (W. Europe, .582)};
		\legend{E-Locos, D-Locos, Shunters, Coaches, Wagons};
		\end{axis}
		\end{tikzpicture}
        		\end{center}
     \end{column}
 \end{columns}
}

\frame{\frametitle{Gr\"o{\ss}e und Entwicklung des Marktes}
\framesubtitle{}
\begin{columns}[t] 
     \begin{column}[T]{6cm} 
     	\begin{center}
		\begin{tikzpicture}[scale = 0.9]
		\begin{axis}[ybar stacked, 
		colormap/bluered,
		ylabel={bn EUR/a},
		legend style={at={(0.5,-0.4)},
		anchor=north, legend columns=2},
		symbolic x coords={EMEA, ASEAN, CIS, E. Europe, NAFTA, RoA, W. Europe},
		xtick=data,
		x tick label style={rotate=45,anchor=east},
		title={Multiple Unit Market Size}]]
		\addplot[fill = RYB2, draw = RYB2!50!black] coordinates
		{(EMEA,0) (ASEAN,0.049) (CIS, 0) (E. Europe, 0) (NAFTA, 0) (RoA, 0) (W. Europe, 1.006)};
		\addplot[fill = RYB3, draw = RYB3!50!black] coordinates
		{(EMEA,0) (ASEAN, 1.638) (CIS, .893) (E. Europe, .668) (NAFTA, .283) (RoA, .463) (W. Europe, 3.342)};
		\addplot[fill = RYB5, draw = RYB5!50!black] coordinates
		{(EMEA,0) (ASEAN,0) (CIS, 0) (E. Europe, 0) (NAFTA, 0) (RoA, 0) (W. Europe, 0)};
		\legend{IC EMU, Regional EMU, IC DMU, Regional DMU};
		\addplot[fill = RYB4, draw = RYB4!50!black] coordinates
		{(EMEA,.033) (ASEAN,.512) (CIS, .024) (E. Europe, .145) (NAFTA, .054) (RoA, .004) (W. Europe, .556)};
		\end{axis}
		\end{tikzpicture}
        		\end{center}
     \end{column}
     	\begin{column}[T]{6cm} 
         	\begin{center}
		\begin{tikzpicture}[scale = 0.9]
		\begin{axis}[ybar stacked, 
		colormap/bluered,
		ylabel={m EUR/a},
		legend style={at={(0.5,-0.40)},
		anchor=north, legend columns=3},
		symbolic x coords={EMEA, ASEAN, CIS, E. Europe, NAFTA, RoA, W. Europe},
		xtick=data,
		x tick label style={rotate=45,anchor=east},
		title={Urban Vehicle Market Size}]]
		\addplot[fill = RYB1, draw = RYB1!50!black] coordinates
			{(EMEA,.054) (ASEAN,.152) (CIS, .082) (E. Europe, .392) (NAFTA, .563) (RoA, .043) (W. Europe, 1.309)};
		\addplot[fill = RYB2, draw = RYB2!50!black] coordinates
		{(EMEA,.257) (ASEAN,1.638) (CIS, .052) (E. Europe, .215) (NAFTA, 1.131) (RoA, .607) (W. Europe, .661)};
		\addplot[fill = RYB3, draw = RYB3!50!black] coordinates
		{(EMEA,.142) (ASEAN, .094) (CIS, 0) (E. Europe, 0) (NAFTA, .085) (RoA, .151) (W. Europe, .104)};
		\legend{LRV, Metro, APM};
		\end{axis}
		\end{tikzpicture}
        		\end{center}
     \end{column}
 \end{columns}
}

\frame{\frametitle{Gesamtmarkt Schienenfahrzeuge}
\framesubtitle{}
\begin{center}
\pgfplotsset{width=8.6cm}
		\begin{tikzpicture}[scale = 0.95]
		\begin{axis}[ybar stacked,
		ylabel={bn EUR/a},
		legend style={at={(1.5,1)}},
		%anchor=north, legend columns=3},
		symbolic x coords={EMEA, ASEAN, CIS, E. Europe, NAFTA, RoA, W. Europe},
		xtick=data,
		x tick label style={rotate=45,anchor=east},
		title={Rail Vehicle Market Size}]]
		\addplot[fill = RYB11, draw = RYB11!50!black] coordinates
			{(EMEA,0.220) (ASEAN,1.367) (CIS, 1.219) (E. Europe, .248) (NAFTA, .192) (RoA, .037) (W. Europe, .421)};
		\addplot[fill = RYB12, draw = RYB12!50!black] coordinates
		{(EMEA,.294) (ASEAN,.823) (CIS, .196) (E. Europe, .103) (NAFTA, .992) (RoA, .218) (W. Europe, .244)};
		\addplot[fill = RYB13, draw = RYB13!50!black] coordinates
		{(EMEA,.012) (ASEAN, .067) (CIS, .212) (E. Europe, .026) (NAFTA, .027) (RoA, .005) (W. Europe, .128)};
		\addplot[fill = RYB14, draw = RYB14!50!black] coordinates
		{(EMEA,.172) (ASEAN,.330) (CIS, .686) (E. Europe, .196) (NAFTA, .383) (RoA, .209) (W. Europe, .501)};
		\addplot[fill = RYB15, draw = RYB15!50!black] coordinates
		{(EMEA,.129) (ASEAN, .932) (CIS, 2.671) (E. Europe, .281) (NAFTA, 1.337) (RoA, .125) (W. Europe, .582)};
		\addplot[fill = RYB16, draw = RYB16!50!black] coordinates
		{(EMEA,0) (ASEAN,0.049) (CIS, 0) (E. Europe, 0) (NAFTA, 0) (RoA, 0) (W. Europe, 1.006)};
		\addplot[fill = RYB17, draw = RYB17!50!black] coordinates
		{(EMEA,0) (ASEAN, 1.638) (CIS, .893) (E. Europe, .668) (NAFTA, .283) (RoA, .463) (W. Europe, 3.342)};
		\addplot[fill = RYB18, draw = RYB18!50!black] coordinates
		{(EMEA,0) (ASEAN,0) (CIS, 0) (E. Europe, 0) (NAFTA, 0) (RoA, 0) (W. Europe, 0)};
		\addplot[fill = RYB19, draw = RYB19!50!black] coordinates
		{(EMEA,.033) (ASEAN,.512) (CIS, .024) (E. Europe, .145) (NAFTA, .054) (RoA, .004) (W. Europe, .556)};
		\addplot[fill = RYB20, draw = RYB20!50!black] coordinates
			{(EMEA,.054) (ASEAN,.152) (CIS, .082) (E. Europe, .392) (NAFTA, .563) (RoA, .043) (W. Europe, 1.309)};
		\addplot[fill = RYB21, draw = RYB21!50!black] coordinates
		{(EMEA,.257) (ASEAN,1.638) (CIS, .052) (E. Europe, .215) (NAFTA, 1.131) (RoA, .607) (W. Europe, .661)};
		\addplot[fill = RYB22, draw = RYB22!50!black] coordinates
		{(EMEA,.142) (ASEAN, .094) (CIS, 0) (E. Europe, 0) (NAFTA, .085) (RoA, .151) (W. Europe, .104)};
		\legend{E-Locos, D-Locos, Shunters, Coaches, Wagons, IC EMU, Regional EMU, IC DMU, Regional DMU, LRV, Metro, APM};
		\end{axis}
		\end{tikzpicture}
        		\end{center}

}



\frame{\frametitle{Vertrieb und Beschaffung}
\framesubtitle{}
\begin{columns}[t] 
     \begin{column}[T]{5cm} 
     	\begin{itemize}
     		\item Typisch: B-to-B-Markt
		\item Eigenschaften:
		\begin{itemize}
		\item Investition statt Konsum
		\item Abgeleitete Nachfrage
		\item Multipersonalit\"at
		\item Formalisierte Beschaffung
		\item Individualisierung
		\item Internationalit\"at
		\end{itemize}
     	\end{itemize}
     \end{column}
     	\begin{column}[T]{7cm} 
         	\begin{center}
            		\includegraphics[width=\textwidth]{Oligopol}
        		\end{center}
     \end{column}
 \end{columns}
}

\frame{\frametitle{Bechaffungsmanagement}
\framesubtitle{}
%\begin{columns}[t] 
    % \begin{column}[T]{8cm} 
     	\begin{itemize}
     		\item Ziele:
		\begin{itemize}
		\item Kosten 
		\item Qualit\"at
		\item Risiko
		\item Flexibilit\"at 		
		\end{itemize}
		\item Strategien:
		\begin{itemize}
		\item Mutiple Sourcing
		\begin{itemize}
		\item[+] Wettbewerb, Risiken minimieren
		\item[-] Aufwand z.B. bei Qualit\"atsunterschienden 
		\end{itemize}
		\item Single Sourcing
		\begin{itemize}
		\item[+] enge Zusammenarbeit, Entwicklung
		\item[-] Wettbewerb eingeschr\"ankt 
		\end{itemize}
		\item Dual Sourcing
		\begin{itemize}
		\item[+/-] Vereint Vor- und Nachteile
		\end{itemize}
		\end{itemize}
     	\end{itemize}
     }
     
\frame{\frametitle{Konzepte der Beschaffung}
\framesubtitle{}
\begin{itemize}
\item Komplexit\"at und Umfang
	\begin{itemize}
	\item System / Module Sourcing
	\item Component Sourcing
	\item Parts Sourcing
	\end{itemize}
\item Ort der Beschaffung
	\begin{itemize}
	\item Lokal oder global
	\item Intern oder extern (Make or Buy)
	\end{itemize}
\item Bereitstellung
	\begin{itemize}
	\item Stock Sourcing
	\item Demand Tailored Sourcing
	\item Just-In-Time-Sourcing
	\end{itemize}
\end{itemize}
}

\frame{\frametitle{Marktschranken}
\framesubtitle{In der Bahnindustrie sind im Vergleich zu anderen Branchen die Marktschranken hoch.}
     	\begin{itemize}
     		\item Markteintrittsschranke: erschwert den Eintritt neuer Marktteilnehmer
		\item In Bahnm\"arkten h\"aufig:
		\begin{itemize}
		\item Regulatorische Schranken
		\item Normative Schranken
		\item K\"auferpr\"aferenzen
		\end{itemize}
		\item Marktaustrittsschranke: erschwert den Austritt der Marktteilnehmer
		\item Typisch f\"ur M\"arkte mit hohen Schranken: Hohe Margen, Oligopole
     	\end{itemize}
    }
    
\offslide{Sammlung Marktschranken}

\frame{\frametitle{Marktrisiken}
\framesubtitle{}
\begin{columns}[t] 
     \begin{column}[T]{6cm} 
     	\begin{itemize}
     		\item Marktspezifische Risiken:
		\begin{itemize}
		\item Implizite Anforderungen
		\item Marktverdr\"angung durch Wettbewerber
		\end{itemize} 
     	\end{itemize}
     \end{column}
     	\begin{column}[T]{6cm} 
         	\begin{center}
            		%\includegraphics[width=0.8\textwidth]{}
        		\end{center}
     \end{column}
 \end{columns}
}

%\offslide{Sammlung von Marktrisiken}
%
%\frame{\frametitle{Diversifikation: Analyse mittels Ansoff-Matrix}
%\framesubtitle{}
%\begin{columns}[t] 
%     \begin{column}[T]{6cm} 
%     	\begin{itemize}
%     		\item Strategische Marktziele
%		\begin{itemize}
%		\item z.B. Markteintritt in Australien
%		\end{itemize}
%		\item ben\"otigen Produkte
%		\begin{itemize}
%		\item z.B. Kupplung AAR Type H
%		\end{itemize}
%		\item Sowohl der Markteintritt als auch das neue Produkt bergen Risiken!
%     	\end{itemize}
%     \end{column}
%     	\begin{column}[T]{6cm} 
%         	\begin{center}
%            		\includegraphics[width=\textwidth]{Ansoff}
%        		\end{center}
%     \end{column}
% \end{columns}
%}
