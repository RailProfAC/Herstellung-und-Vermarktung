% !TEX root = SFV-15014_HuV.tex
\section{Kosten- und Aufwandssch\"atzung}
\sectionpage

\frame{\frametitle{Warum Kosten- und Aufwandssch\"atzung?}
\framesubtitle{Kosten- und Aufwandssch\"atzung ist die Grundlage f\"ur erfolgreiche Projektbearbeitung.}
\begin{itemize}
\item Aufwandssch\"atzung (Gr\"o{\ss}e: Zeit)
\begin{itemize}
	\item Identifikation von Arbeitspaketen
	\item Input f\"ur Kostensch\"atzung
	\item Ressourcenplanung und -allokation
	\item Terminplanung (auch projekt\"ubergreifend)
\end{itemize}
\item Kostensch\"atzung (Gr\"o{\ss}e: Geld)
\begin{itemize}
	\item Bestimmung von:
	\begin{itemize}
		\item Einmalkosten \textit{non recurring cost (NRC)}
		\item St\"uckkosten \textit{recurring cost (RC)}
	\end{itemize}
	\item Identifikation von Investitionen
\end{itemize}
\item Entscheidungshilfe im Entwicklungsprozess
\item Bestimmung des Angebotspreises
\end{itemize}
}

\frame{\frametitle{Herausforderungen Aufwands- und Kostensch\"atzung}
\framesubtitle{}
\begin{itemize}
\item Informationen:
\begin{itemize}
	\item unvollst\"andig
	\item unsicher
	\item fehlerbehaftet 
	\item Daher: Sch\"atzung, d.h. wahrscheinlichste Vorhersage \"uber den wahren Aufwand %{\color{red!80!black} Was wird gesch\"atzt?}
\end{itemize}
\item Projektdefinition:
\begin{itemize}
	\item Anforderungen nicht final (``to be defined during design stage'')
	\item \"Anderungen m\"oglich
\end{itemize}
\item Projektablauf:
\begin{itemize}
	\item Beginn durch Angebotsrunden verz\"ogert
	\item Projektverlauf durch externe Einfl\"usse (teil-)gesteuert 
\end{itemize}
\item Projektressourcen:
\begin{itemize}
	\item Durch andere Projekte Ressourcen blockiert oder eingeschr\"ankt nutzbar 
\end{itemize}
\end{itemize}
}

\subsection{Aufwandssch\"atzung}
\subsectionpage

\frame{\frametitle{Ans\"atze zur Aufwandssch\"atzung}
\framesubtitle{}
\begin{itemize}
\item Expertensch\"atzverfahren, z.B.:
\begin{itemize}
	\item Projektstrukturplan-basiert \textit{(WBS-based)}
	\item Gruppensch\"atzung
\end{itemize}
\item Formale Sch\"atzverfahren, z.B.:
\begin{itemize}
	\item Analogie-basiert (z.B. Bremszange wie ..., jedoch mit ...)
	\item Parametrische Modelle (z.B. E-Kupplung Verkabelung: 100 h)
	\item Gr\"o{\ss}enbasiert: (z.B. Anpasskonstruktion: 500 h)
\end{itemize}
\item Kombinierte Sch\"atzverfahren, z.B.:
\begin{itemize}
	\item Zerlegung mit WBS, parametrische Sch\"atzung der Pakete
\end{itemize}
\item Auswahl des Verfahrens:
\begin{itemize}
	\item Abh\"angig von der Organisation
	\item Formale Verfahren weniger ``lernf\"ahig''
	\item Expertensch\"atzverfahren anf\"allig f\"ur ``wishful thinking''
\end{itemize}
\item Psychologische Herausforderungen \textit{(Cognitive biases)}:
\begin{itemize}
	\item \textit{Planning fallacy, cognitive dissonance, anchoring, confirmation bias, wishful thinking}
\end{itemize}
\end{itemize}
}


\frame{\frametitle{Projektstrukturplan}
\framesubtitle{Work Breakdown Structure (WBS) f\"ur die Ermittlung von Arbeitspaketen}
\begin{columns}[t] 
     \begin{column}[T]{6cm} 
     	\begin{itemize}
		\item Dekomposition eines Projekts
		\begin{itemize}
		\item Hierarchisch
		\item Inkrementell
		\end{itemize}
		\item Baumstruktur
		\item Gliederung gem\"a{\ss} DIN 69900
		\begin{itemize}
		\item Funktionsorientiert
		\item \textbf{Objektorientiert}
		\item Zeitorientiert
		\end{itemize}
		\item Starke Abh\"angigkeit von Deliverables
		\item Erstellung \"ublicherweise Top-Down
		\item Nutzen: Vollst\"andige \"Ubersicht
		\item Hilfreich: ``Tickler list'' 
     	\end{itemize}
     \end{column}
     	\begin{column}[T]{6cm} 
         	\begin{center}
            		\includegraphics[width=1\textwidth]{WBS}
        		\end{center}
     \end{column}
 \end{columns}
}

%\offslide{WBS f\"ur Beispielprojekt}

\frame{\frametitle{Aufbereitung WBS f\"ur Projektplanung}
\framesubtitle{Die Identifikation der Arbeitspakete allein l\"asst keine Planung des Projekts zu.}
\begin{itemize}
\item Sch\"atzung des Aufwands
\begin{itemize}
	\item Besprechung: m\"oglicher Bias
	\item Alternative: Planning Poker
\end{itemize}
\item Abh\"angigkeit (Reihenfolge) der Projektbearbeitung
\item Externe Inputs oder Vorbedingungen f\"ur Arbeitspakete
\item Vertraglich zugesicherte Termine
\item Zuordnung zu:
\begin{itemize}
	\item Ressourcen
	\item Phasen
\end{itemize}
\item Zieldefinition (Definition of Done)
\end{itemize}
}

\subsection{Kostensch\"atzung}
\subsectionpage

\frame{\frametitle{Kostensch\"atzung}
\framesubtitle{}
\begin{columns}[t] 
     \begin{column}[T]{6cm} 
     \textbf{NRC estimation}
     	\begin{itemize}
     		\item Basierend auf Aufwandssch\"atzung
		\item Erg\"anzend:
		\begin{itemize}
		\item Kundenbetreuung
		\item Reisekosten
		\item Externe Dienstleistungen (z.B. Tests, Abnahmen, ...)
		\item Prototypen, Muster
		\item Investitionen
		\end{itemize}
		\item Zu beachten:
		\begin{itemize}
		\item Stundens\"atze
		\item Kostenentwicklung
		\end{itemize}
		\item N\"utzlich: Checkliste
     	\end{itemize}
     \end{column}
     	\begin{column}[T]{6cm} 
	\textbf{RC estimation} (\cite{niazi05, pahlbeitz})
         \begin{itemize}
         	\item Intuitive Verfahren:
		\begin{itemize}
		\item Basierend auf Expertensch\"atzung
		\item Unterst\"utzt durch Regeln
		\end{itemize} 
         	\item Analogiebasierte Sch\"atzung
		\begin{itemize}
		\item \"Ahnlichkeit
		\item Komplexit\"at 
		\end{itemize} 
     		\item Parametrische Sch\"atzung:
		\begin{itemize}
		\item z.B. Gewicht, Material oder kombiniert %, z.B. \tiny
%		\begin{equation*} 
%		C = FC+\left(C_{co} N_{co} +  \frac{C_{rm}TF}{1-SC} \right)W
%		\end{equation*}
		\end{itemize}
		\item Analytische Verfahren, z.B.
		\begin{itemize}
		\item Bearbeitungssimulation
		\item Feature based cost estimation
		\end{itemize}
     	\end{itemize}
     \end{column}
 \end{columns}
}

%\frame{\frametitle{Kostenrechnung}
%\framesubtitle{Grundbegriffe}
%\begin{itemize}
%\item Fixe Kosten:
%\begin{itemize}
%	\item Produktionsunabh\"angig anfallende Kosten 
%\end{itemize}
%\item Variable Kosten:
%\begin{itemize}
%	\item Produktionsabh\"angig anfallende Kosten 
%\end{itemize}
%\item Gemeinkosten:
%\begin{itemize}
%	\item Kosten, die keinem Kostentr\"ager zugeordnet werden k\"onnen 
%\end{itemize}
%\item Einzelkosten:
%\begin{itemize}
%	\item Kosten, die direkt einem Kostentr\"ager zugeordnet werden k\"onnen 
%\end{itemize}
%\item Deckungsbeitrag:
%\begin{itemize}
%	\item DB I: Umsatz - variable Kosten
%	\item DB II DB I - fixe Kosten 
%\end{itemize}
%\end{itemize}
%}


