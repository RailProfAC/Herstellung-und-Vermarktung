% !TEX root = SFV-15014_HuV.tex
\section{Schraubenverbindungen}
\sectionpage

\frame{\frametitle{Schraubenverbindungen an Schienenfahrzeugen - Allgemeines}
\framesubtitle{}
\begin{definition}[Schraubenverbindungen]
Schraubverbindungen erm\"oglichen das l\"osbare Verbinden von Werkst\"ucken mittels Verbindungselementen. Eine Schraubenverbindung umfasst sowohl die Verbindungelemente als auch die zu verbindenden Teile.  
\end{definition}
\textbf{Anwendung in der Schienenfahrzeugtechnik}
\begin{itemize}
	\item Bremszange an Drehgestellrahmen
	\item Radbremsscheibe an Rad
	\item Elektrische Kontakte (z.B. Erdung, Stromversorgung)
\end{itemize}
\textbf{Herausforderungen in der Schienenfahrzeugtechnik}
\begin{itemize}
	\item Lange Produktlebensdauer, hohe Schwingspielzahlen
	\item Hohes Sicherheitsbed\"urfnis
	\item Zertifizierungs- und Pr\"ufaufwand
\end{itemize}
}

\frame{\frametitle{Grundlage: DIN 25201}
\framesubtitle{}
\begin{itemize}
\item Sieben Teile:
\begin{enumerate}
\item Einteilung, Kategorien der Schraubverbindungen
\item Konstruktion - maschinenbauliche Anwendungen
\item Konstruktion - elektrische Anwendungen
\item Sichern von Schraubenverbindungen
\item Korrosionsschutz
\item Anschlussma{\ss}e
\item Montage
\end{enumerate}
\end{itemize}
}

\frame{\frametitle{Risikoklassen der Schraubenverbindungen}
\framesubtitle{}
\begin{itemize}
\item Risikoklasse H (hoch)
	\begin{itemize}
		\item Das Versagen der Schraubenverbindung stellt eine direkte oder indirekte Gefahr f\"ur Leib und Leben dar.
	\end{itemize}
\item Risikoklasse M (mittel)
	\begin{itemize}
		\item Das Versagen der Schraubenverbindung f\"uhrt zu einer Funktionsst\"orung des Fahrzeugs.
	\end{itemize}
\item Risikoklasse G (gering)
	\begin{itemize}
		\item Das Versagen der Schraubenverbindung f\"uhrt maximal zu Komforteinbu{\ss}en f\"ur die Fahrg\"aste oder das Bedienpersonal.
	\end{itemize}
\end{itemize}
}

\frame{\frametitle{Anforderungen DIN 25201}
\framesubtitle{}
         	\begin{center}
            		\includegraphics[width=0.8\textwidth]{25201T1}\source{\cite[Tab. 1]{din252011}}
        		\end{center}
}

%\frame{\frametitle{Beispiele f\"ur Risikoklassen: Hoch}
%\framesubtitle{}
%         	\begin{center}
%            		\includegraphics[width=0.8\textwidth]{25201TA1}\source{\cite[Tab. A.1]{din252011}}
%        		\end{center}
%}
%
%\frame{\frametitle{Beispiele f\"ur Risikoklassen: Mittel}
%\framesubtitle{}
%         	\begin{center}
%            		\includegraphics[width=0.8\textwidth]{25201TA2}\source{\cite[Tab. A.2]{din252011}}
%        		\end{center}
%}
%
%\frame{\frametitle{Beispiele f\"ur Risikoklassen: Gering}
%\framesubtitle{}
%         	\begin{center}
%            		\includegraphics[width=0.8\textwidth]{25201TA3}\source{\cite[Tab. A.3]{din252011}}
%        		\end{center}
%}

\frame{\frametitle{Schraubf\"alle}
\framesubtitle{Verhindern von Klaffen oder Schlupf}
         	\begin{center}
            		\includegraphics[width=0.8\textwidth]{DIN25201A11}\source{\cite[Abb. 1]{din252012}}
        		\end{center}
}

\frame{\frametitle{Schraubf\"alle}
\framesubtitle{Verhindern von Klaffen oder Schlupf}
         	\begin{center}
            		\includegraphics[width=0.8\textwidth]{DIN25201A12}\source{\cite[Abb. 1]{din252012}}
        		\end{center}
}

\frame{\frametitle{Belastung und Versagen der Schrauben}
\framesubtitle{Grundlage: Schraube als schw\"achstes Element}
         	\begin{center}
            		\includegraphics[width=0.8\textwidth]{DIN25201A2}\source{\cite[Abb. 2]{din252012}}
        		\end{center}
}

\frame{\frametitle{Klemml\"ange $l_{K}$}
\framesubtitle{Grundlage: $l_{K} > 3{,}5 d$, damit elastische Verspannung erhalten bleibt}
         	\begin{center}
            		\includegraphics[width=0.8\textwidth]{DIN25201A3}\source{\cite[Abb. 3]{din252012}}
        		\end{center}
}

\frame{\frametitle{Konstruktive Grundlagen der Verschraubung}
\framesubtitle{}
         	\begin{center}
            		\includegraphics[width=0.8\textwidth]{DIN252012T1}\source{\cite[Tab. 1]{din252012}}
        		\end{center}
}

\frame{\frametitle{Haftreibungszahlen in der Trennfuge}
\framesubtitle{}
         	\begin{center}
            		\includegraphics[width=0.8\textwidth]{DIN252012T2}\source{\cite[Tab. 2]{din252012}}
        		\end{center}
}

\frame{\frametitle{Anforderungen an die Verbindung}
\framesubtitle{}
\begin{itemize}
\item Elastische Nachgiebigkeit: Hoch bei Schraube, gering bei verspannte Bauteilen
\item Schrauben und Muttern gleicher Festigkeitsklassen
\item Anzahl Unterlegeteile minimiert
\item Montagewerkzeuge: Innen- bzw Au{\ss}ensechskant oder -sechsrund
\item Metrisches ISO-Regelgewinde
\item Schraubenwerkstoff:
\begin{itemize}
	\item Bevorzugte Festigkeitsklassen: 8.8, A2-70 und A4-80
	\item Festigkeitsklasse 12.9 wird nicht betrachtet (vgl. DB G\"utep\"ufung)
\end{itemize}
\item Oberfl\"achenbeschichtung
\begin{itemize}
	\item Korrosionbest\"andigkeit
	\item Bei hohen Festigkeiten: Waserstoffverspr\"odung vermeiden
	\item Definiertes und enges Reibungszahlfenster
\end{itemize}
\end{itemize}
}

\subsection{Schraubensicherung}
\frame{\frametitle{Schraubensicherung - L\"osen der Schraubenverbindung}
\framesubtitle{}
         	\begin{center}
            		\includegraphics[width=0.7\textwidth]{DIN252014A1}\source{\cite[Abb. 1]{din252014}}
        		\end{center}
}


\frame{\frametitle{Schraubensicherung - Methoden}
\framesubtitle{}
         	\begin{center}
            		\includegraphics[width=0.8\textwidth]{DIN252014T1}\source{\cite[Tab. 1]{din252014}}
        		\end{center}
}
\subsection{Schraubensicherung}
\frame{\frametitle{Schraubensicherung - Sicherungsmittel}
\framesubtitle{}
         	\begin{center}
            		\includegraphics[width=0.9\textwidth]{DIN252014TA1}\source{\cite[Tab. A.1]{din252014}}\\
%        		\end{center}
%}
%\subsection{Schraubensicherung}
%\frame{\frametitle{Schraubensicherung - L\"osen der Schraubenverbindung}
%\framesubtitle{}
%         	\begin{center}
            		\includegraphics[width=0.9\textwidth]{DIN252014TA11}%\source{\cite[Tab. A.1]{din252014}}
        		\end{center}
}


