% !TEX root = SFV-15014_HuV.tex
\section{Projektdokumentation}
%\sectionpage

\frame{\frametitle{Dokumentationssystem}
\framesubtitle{Im Bereich Bahn bestehen zum Teil lange Aufbewahrungspflichten und noch l\"angere Aufbewahrungsinteressen, z.B. f\"ur Ersatzteile, Prozesse, etc.}
\begin{itemize}
\item Identifizieren erwarteter Dokumente
	\begin{itemize}
		\item z.B. Vertragsdokumente, Kalkulationen, Berichte, Entwicklungs- und Testdokumentation
	\end{itemize}
\item Kennzeichnungssystem
	\begin{itemize}
		\item Eindeutigkeit, Aktualit\"at, Relevanz des Dokuments
	\end{itemize}
\item Anforderungen an Dokumente
	\begin{itemize}
		\item Formale Anforderungen: Name und Status des Dokuments, Projekt, Ersteller, Pr\"ufer, Freigeber, Verteiler, Integrit\"at (z.B. durch Seitenzahlen)
		\item Inhaltliche Anforderungen
	\end{itemize}
\item Verantwortlichkeiten
	\begin{itemize}
		\item z.B. anhand einer Dokumentenmatrix
	\end{itemize}
\item Ablagestruktur
	\begin{itemize}
		\item z.B. gemeinsamer Netzwerkordner
	\end{itemize}
\item Datensicherung
\end{itemize}
}


\frame{\frametitle{Warum Projektdokumentation?}
\framesubtitle{}
\begin{itemize}
\item Multipersonalit\"at
\item Langer Projektlebenszyklus
\item Rechtliche Auswirkungen
	\begin{itemize}
		\item Strafrechtlich (Sorgfaltspflicht!)
		\item Zivilrechtlich 
		\begin{itemize}
		\item Nachforderungen
		\item Nichterf\"ullung von Anforderungen
		\end{itemize}
	\end{itemize}
\item Dokumentationspflichten
	\begin{itemize}
		\item Kundenforderung
		\item Normative Anforderungen (z.B. ISO 9001, IRIS)
	\end{itemize}
\item Wiederverwendbarkeit der Entwicklung
\item Nachvollziehbarkeit von Entscheidungen, Kalkulationen, etc.
\item Zulassung	
\end{itemize}
}

\frame{\frametitle{Welche Arten von Dokumenten?}
\begin{itemize}
 \item<1-> Projektmanagement-Dokumente
\begin{itemize}
	\item<2-> Projektauftrag, Aufgabenlisten, -zuordnungen, Terminpl\"ane, Dokumentenmatrix, Statusberichte, Budget-Reporting, Lieferstaffeln, Gespr\"achsprotokolle, Re...
	%\item Steuerung des Projekts
\end{itemize}
\item<1-> Technische Dokumente
\begin{itemize}
	\item<2-> Anforderungs-Dokumente, St\"ucklisten, Zeichnungen, Nachweise (z.B. Berechnungen), Berichte, Abweichungs- und \"Anderungsmitteilungen, ...
\end{itemize}
\item<1-> Betriebswirtschaftliche Dokumente
\begin{itemize}
	\item<2-> Kalkulationen, Preiseskalation, Angebote von Zulieferern und Dienstleistern, Verhandlungsprotokolle, 
\end{itemize}
\item<1-> Dokumente des Qualit\"atsmanagements
\begin{itemize}
	\item<2-> Pr\"ufanweisungen, Ergebnisse, Zeugnisse, Lieferantenaudits
\end{itemize}
\item<1-> Rechtliche Dokumente
\begin{itemize}
	\item<2-> Vertrag, Rahmenvertrag, 
\end{itemize}
\end{itemize}
}

\frame{\frametitle{Dokumentenlenkung}
\framesubtitle{Lenkung von Informationen gem\"a{\ss} ISO 9001}
\begin{itemize}
\item Erstellung und Aktualisierung
\begin{itemize}
	\item Angemessene Kennzeichnung und Beschreibung
	\begin{itemize}
		\item z.B. Titel, Datum, Autor, Referenznummer
	\end{itemize}
	\item angemessenes Format und Medium
	\begin{itemize}
		\item z.B. Sprache, Softwareversion, Grafiken
	\end{itemize}
	\item Angemessene \"Uberpr\"ufung und Genehmigung im Hinblick auf Eignung und Angemessenheit
\end{itemize}
\item Lenkung der Informationen
\begin{itemize}
	\item Informationen sind verf\"ugbar und geeignet
	\item Informationen sind angemessen gesch\"utzt
\end{itemize}
\item Besondere Aufgaben der Dokumentenlenkung
\begin{itemize}
	\item Verteilung, Zugriff, Auffindung und Verwendung
	\item Ablage/Speicherung und Erhaltung (einschlie{\ss}lich Lesbarkeit)
	\item \"Uberwachung von \"Anderungen (z.B. Versionskontrolle)
	\item Aufbewahrung und Verf\"ugung \"uber den weiteren Verbleib
\end{itemize}
\end{itemize}
}


%\offslide{Informationsgehalt wichtiger Dokumente}{Generell, Aktionsliste, Besprechungsprotokoll, Dokumentenmatrix}
%
%\offslide{Erstellung Projektauftrag}

\section{Vertragspr\"ufung}
%\sectionpage

%\offslide{Warum Vertragspr\"ufung?}{Teilweise normativ vorgeschrieben, z.B. IRIS, EN 15085,...}

\frame{\frametitle{Aufgaben der Vertragspr\"ufung}
\framesubtitle{}
\begin{itemize}
\item Angebotsphase
\begin{itemize}
	\item Pr\"ufung auf Vollst\"andigkeit
	\item Pr\"ufung auf Risiken
\end{itemize}
\item Vertragsabschlussphase
\begin{itemize}
	\item Pr\"ufung auf Vollst\"andigkeit
	\item Pr\"ufung auf Unstimmigkeit
	\item Pr\"ufung auf Widerspr\"uchlichkeit
\end{itemize}
\item Abwicklungsphase
\begin{itemize}
	\item Verfolgen von \"Anderungen
	\item Verfolgen von Abweichungen
\end{itemize}
\end{itemize}
}

\frame{\frametitle{Vorgehen in der Angebotsphase}
\framesubtitle{}
\begin{itemize}
\item Rechte und Pflichten der Vertragsparteien
\begin{itemize}
	\item Dokumentenhierarchie
	\item Liefer- und Leistungsumfang
\end{itemize}
\item Mitwirkungspflichten
\begin{itemize}
	\item Auftraggeber
	\item Auftragnehmer
\end{itemize}
\item Analyse der Regelungen u.a. zu
\begin{itemize}
	\item Vertragsstrafen (z.B. Gewichtsp\"onale, Lieferverzug,...)
	\item Abnahmen
	\item \"Anderungen
	\item Verz\"ogerungen
\end{itemize}
\item Beurteilen besonderer vertraglicher Risiken
\end{itemize}
\begin{enumerate}
\item Lesen der Dokumente
\item Herausforderungen erkennen
\item Ma{\ss}nahmen erarbeiten und umsetzen
\end{enumerate}
}

\frame[allowframebreaks]{\frametitle{Wichtige Aspekte bei der Vertragspr\"ufung}
\framesubtitle{}
\begin{itemize}
\item Anwendbares Recht, Gerichtsstand
\item Regelung von Folgesch\"aden
\item Verzeichnis der Vertragsdokumente (inkl. Ausgabestand)
\item Liefer- und Leistungsumfang
\item Preisstellung (DDP Oslo vs. EXW), Preiseskalation
\item Umgang mit Abweichungen, technischem Fortschritt
\item Technische Termine
\item Optionen
\item Teillieferungen
\item Versp\"atung bei Lieferung, Dokumentation, IBS und P\"onalen
\item Nichteinhalten der vertraglichen Leistungswerte (Qualit\"at, RAMS, LCC,...)
\item Force-Majeur-Klausel
\item Produktionsstandorte
\item Logistik, Verpackung und Konservierung
\item Pr\"ufungen und Tests
\item Schulungen (Kunde und Betreiber)
\item Zertifikate
\item Gew\"ahrleistung
\end{itemize}
}
