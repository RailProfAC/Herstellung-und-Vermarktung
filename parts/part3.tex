% !TEX root = ../HuV-Folien.tex
\section{Projektablauf, -kriterien und organisation}
%\sectionpage

\subsection{Projektablauf}
\frame{\subsectionpage}

\offslide{Tafelbild Projektablauf, Ausblick V-Modell, Phasen, Meilensteine}

\subsection{Projektkriterien}
\frame{\subsectionpage}

\frame{\frametitle{Fragen an die Projektorganisation}
\framesubtitle{Eine gute Vorbereitung der Organisation und der Infrastruktur macht erfolgreiche Projektarbeit m\"oglich.}
\begin{itemize}
\item Was verstehen wir unter einem Projekt?
\item Wie binden wir Projekte in unser Unternehmen ein?
\item Welche standardisierten Vorgehensmodelle wenden wir an?
\item Wie stellen wir sicher, dass alle Informationen zur richtigen Zeit verf\"ugbar sind?
\item Welche Dokumente/Dokumentenarten werden eingesetzt? Wie werden sie verwaltet?
\item Gibt es Verhaltensregeln f\"ur das Projektteam?
\item Wie sichern wir die Qualit\"at der Projektbearbeitung?
\end{itemize}
}

\frame{\frametitle{Projektmerkmale}
\framesubtitle{Notwendige Projektmerkmale nach \citep{felkai}}
\begin{itemize}
\item Zeitliche Befristung
\item Eindeutige Zielsetzung
\item Eindeutige Zuordnung der Verantwortungsbereiche
\item Einmaliger (azyklischer) Ablauf/Einmaligkeit
\item Vorgegebener finanzieller Rahmen und begrenzte Ressourcen
\item Komplexit\"at
\item Interdisziplin\"arer Charakter der Aufgabenstellung
\item Relative Neuartigkeit
\item Projektspezifische Organisation
\item Arbeitsteilung
\item Unsicherheit und Risiko
\end{itemize}
}

\subsection{Projektorganisation}
\frame{\subsectionpage}

%\offslide{Formen der Aufbauorganisation}

\offslide{Formen der Projektorganisation}



