% !TEX root = HuV-Folien.tex
\section{Einf\"uhrung in die Veranstaltung}
\begin{frame} % Cover slide
\titlepage
\end{frame}

\frame{\sectionpage}

%\frame{\frametitle{Ihre Dozenten}
%\framesubtitle{}
%\begin{columns}[t] 
%     \begin{column}[T]{5cm} 
%     	\begin{center}
%         \includegraphics[width=0.8\textwidth]{Axel} 
% 	\end{center}
%	Prof. Dr. Axel Thomas\\
%	Gesch\"aftsf\"uhrer\\
%	WFG Aachen\\
%	A.Thomas@wfg-aachen.de
%     \end{column}
%     	\begin{column}[T]{5cm} 
%         	\begin{center}
%          \includegraphics[width=0.8\textwidth]{Raphael}
% 	\end{center}
%	Prof. Dr. Raphael Pfaff \\
%	Schienenfahrzeugtechnik\\
%	pfaff@fh-aachen.de\\
%	@RailProfAC
%     \end{column}
% \end{columns}
%}
%
%
%\frame{\frametitle{Vorstellungsrunde}
%\framesubtitle{Beispielfragen - stellen Sie sich vor!}
%\begin{itemize}
%\item (Schul-)Bildung?
%\item (Berufs-)Erfahrung?
%\item Warum Schienenfahrzeugtechnik?
%\item Was erwarte ich vom BEng Schienenfahrzeugtechnik?
%\item Was erwarte ich von Betriebswirtschaft und Technik der Eisenbahnen?
%\item Sprachkenntnisse (insbesondere Englisch)?
%\item Bus/Bahn/Auto/Fahrrad/...?
%\item Hobbies?
%\item ...
%\end{itemize}
%\note{\"Ubung Name mit/ohne BEng}
%}
%
%\frame{\frametitle{Anforderungen ``First Cycle'' - Bachelor}
%\framesubtitle{Anforderungen gem\"a{\ss} Dublin Descriptors \citep{joint2004shared}}
%\begin{columns}[t] 
%     \begin{column}[T]{7cm} 
%     	\begin{itemize}
%     		\item Knowledge and understanding in a field of study
%		\begin{itemize}
%		\item Typically supported by textbooks
%		\item Some aspects informed by knowledge on the forefront of the field of study
%		\end{itemize}
%		\item Apply knowledge and understanding indicating a professional approach
%		\item Gather and interpret data to inform judgement
%		\item Communicate information, ideas, problems and solutions
%		\item Learning skills to undertake further study with high degree of autonomy
%     	\end{itemize}
%     \end{column}
%     	\begin{column}[T]{5cm} 
%         	\begin{center}
%	\vspace{1cm}
%            		\includegraphics[width=0.8\textwidth]{GraduationHat}
%        		\end{center}
%     \end{column}
% \end{columns}
%}
%
%
%\frame{\frametitle{Anforderungen ``Niveau 6'' - Bachelor}
%\framesubtitle{Anforderungen gem\"a{\ss} Deutschem Qualifizierungsrahmen}
%\begin{columns}[t] 
%     \begin{column}[T]{7cm} 
%     	\begin{itemize}
%     		\item Breites und integriertes Wissen
%		\begin{itemize}
%		\item Wissenschaftliche Grundlagen
%		\item Praktische Anwendungen 
%		\end{itemize}
%		\item Breites Spektrum an Methoden
%		\begin{itemize}
%		\item Neue L\"osungen erarbeiten und bewerten
%		\end{itemize}
%		\item Verantwortlich in Expertenteams arbeiten oder leiten
%		\item Ziele f\"ur Lern- und Arbeitsprozesse definieren, reflektieren und bewerten
%		\item Lern- und Arbeitsprozesse eigenst\"andig und nachhaltig gestalten
%     	\end{itemize}
%     \end{column}
%     	\begin{column}[T]{5cm} 
%         	\begin{center}
%	\vspace{1cm}
%            		\includegraphics[width=0.8\textwidth]{GraduationHat}
%        		\end{center}
%     \end{column}
% \end{columns}
%}
%
%
%\frame{\frametitle{Anforderungen BEng Schienenfahrzeugtechnik}
%\framesubtitle{Was zeichnet einen Bachelor der Schienenfahrzeugtechnik aus?}
%\begin{columns}[t] 
%     \begin{column}[T]{6cm} 
%     	\begin{itemize}
%		\item Wissenschaftliches Arbeiten
%		\begin{itemize}
%		\item Nutzung Prim\"arliteratur und Normen
%		\item Erstellung Seminararbeiten
%		\end{itemize}
%     		\item Selbstlernkompetenz
%		\begin{itemize}
%		\item Beispiel: Nutzung Lehrbuch statt Skript
%		\end{itemize}
%		\item Verfassung wissenschaftlicher und technischer Texte
%		\item Fachvortrag zu Seminararbeit
%		\item Bericht zum Praktikum
%     	\end{itemize}
%     \end{column}
%     	\begin{column}[T]{6cm} 
%         	\begin{center}
%	\vspace{1cm}
%            		\includegraphics[width=0.8\textwidth]{GraduationHat}
%        		\end{center}
%     \end{column}
% \end{columns}
%}

\frame{\frametitle{Themenplan}
\framesubtitle{Plan - Verschiebungen sind m\"oglich!}
\begin{center}
\small
\vspace{-.5cm}
\begin{tabular}{|c|l|l|}
\hline
Datum & Thema & Dozent\\ \hline
 & Verkaufsgespräche richtig führen & Thomas\\ \hline
 & Verkaufsgespräche richtig führen & Thomas\\ \hline
 & Vertragsinhalte & Thomas\\ \hline
 & Vertragsinhalte & Thomas\\ \hline
 & Verkaufsgespr\"ache/Vertragsinhalte & Thomas\\ \hline
 & Kostenmanagement & Thomas\\ \hline
 & Finanzierungsaspekte zur Vertriebsunterstützung & Thomas\\ \hline
 30.5. & Einf\"uhrung, Marktsegmente, Marktschranken & Pfaff\\ \hline
 & Lebenszyklusmodelle, Projektmanagement & Pfaff\\ \hline
13.6. & Requirements Engineering, Aufwandssch\"atzung & Pfaff\\ \hline
20.6. & Projektplanung & Pfaff\\ \hline
27.6. & Schwei{\ss}en  & Pfaff\\ \hline
18.7. & Schrauben & Pfaff\\ \hline
 & Korrosionsschutz, DB G\"utepr\"ufung, IBS & Pfaff\\ \hline
 & Lessons learned Railway Challenge & Pfaff \\ \hline
\end{tabular}
\end{center}
}

