% !TEX root = SFV-15014_HuV.tex
\section{Materialien, Korrosionsschutz, DB G\"utepr\"ufung}
\frame{\sectionpage}

\subsection{Materialien}
\frame{\subsectionpage}

\frame{\frametitle{Anforderungen an Materialien f\"ur Schienenfahrzeuge}
\framesubtitle{}
\begin{itemize}
\item Statische Festigkeit
\item Dauerfestigkeit
\item Gut zu f\"ugen
\item In Abmessungen und Mengen verf\"ugbar
\item Geringes Gewicht (gemessen an der Festigkeit)
\item Best\"andig gegen Umwelteinfl\"usse
\item Recyclingf\"ahig
\item Keine Freisetzung gef\"ahrlicher Substanzen
\item Angemessene Kosten
\item Reparierbarkeit
\item Betriebserfahrung
\item G\"unstiges Brandverhalten
\end{itemize}
}

\frame[allowframebreaks]{\frametitle{Materialien f\"ur Schienenfahrzeuge}
\framesubtitle{}
\begin{itemize}
\item Baustahl (S235, S355):
	\begin{itemize}
		\item Blechst\"arke bis 12 mm
		\item Zus\"atzlich: Tieftemperatureignung
	\end{itemize}
\item Feinkornbaustahl (S500, S690):
	\begin{itemize}
		\item Gewichtsersparnis bei hochbelasteten Teilen
	\end{itemize}
\item Edelst\"ahle
	\begin{itemize}
		\item Korrosionsbest\"andige St\"ahle, z.B. X5CrNi18-10: Rohrleitungen
		\item Verschleisbest\"andige St\"ahle, z.B. X120Mn12: Gleitelemente
	\end{itemize}	
\item Aluminiumwerkstoffe
	\begin{itemize}
		\item Strangpressprofile: AlMgSi
		\item Bleche: AlMg
	\end{itemize}\newpage
\item Gusswerkstoffe
	\begin{itemize}
		\item Grauguss:
		\begin{itemize}
		\item Gusseisen mit Lamellargraphit, z.B. EN-GJL-300: Geh\"ause, Bremsenteile
		\item Gusseisen mit Kugelgraphit, z.B. EN-GJS-500: Zugstangen, Bremszangen, Bremsscheiben
		\item Bainitisches Gusseisen mit Kugelgraphit, z.B. EN-GJ-800/1000: Hochbelastete Bauteile, Bremshebel
		\end{itemize}
		\item Gussstahl, z.B. G18NiMoCr3-6: hochbelastete Teile, Bremsscheiben
		\item Aluminiumguss
	\end{itemize}
	\item Kunststoffe, z.B. PA, PE
	\item Elastomere, z.B. Silikon, Fluorelastomere (Viton)
\end{itemize}
}


\subsection{Korrosionsschutz}
\frame{\subsectionpage}

\frame[allowframebreaks]{\frametitle{Korrosionsschutz}
\framesubtitle{}
\begin{itemize}
\item Aufgaben:
	\begin{itemize}
		\item Lebensdauer und Atmosph\"are belasten Schienenfahrzeugkomponenten extrem
		\item (Extrem-)Beispiel: Kanaltunnelz\"uge, Metro Uijeongbu
	\end{itemize}
\item Anforderungen:
\begin{itemize}
		\item Hohe Feuchtigkeitseintr\"age
		\item Temperaturschwankungen
		\item Metalleintr\"age in Umgebung
		\item Korrosive Substanzen
		\item Schotterflug
		\item Wartbarkeit (Vandalismus)
		\item Reparierbarkeit
		\item \"Asthethik
		\end{itemize} \newpage
	\item Lack:
\begin{itemize}
		\item Prozess:
	\begin{itemize}
		\item Rohbau bzw. Komponentenfertigung
		\item Strahlen und Reinigen
		\item Abkleben (f\"ur Fl\"achen ohne Grundierung)
		\item Grundierung (f\"ur DB gem. TL 918300: $(30\ldots 80)\, \mu \mathrm{m}$ 2K-EP)
		\item Ggf. Zwischenschicht 
		\item Ggf. F\"uller/Spachtel
		\item Decklack (f\"ur DB gem. TL 918300: $(200\ldots 300)\, \mu \mathrm{m}$ 2K-EP)
		\end{itemize}
		\item \"Ublich: Lacksystem der Betreiber, z.B.
		\begin{itemize}
		\item DB: 2-Komponenten EP-Lack
		\item SNCF: PU-Lack
		\end{itemize}
		\end{itemize}
	\item Verzinken
	\item Chromatieren
	\item GEOMET
	\item Pulverbeschichten
	\item Fett
\end{itemize}
}


\subsection{DB G\"utepr\"ufung}
\frame{\subsectionpage}

\frame{\frametitle{Einleitung}
\framesubtitle{DB G\"utepr\"ufung als Beispiel f\"ur bahnspezifische Qualit\"atsforderungen.}
\begin{itemize}
\item Einkaufsvolumen DB AG 2014: 23{,}2 Mrd. EUR
	\begin{itemize}
		\item Industrielle Produkte: 4{,}3 Mrd. EUR
	\end{itemize}
\item Langj\"ahrige Kenntnis qualit\"ats- und sicherheitsrelevanter Aspekte der Produkte
\item \"Ahnlich bei vielen ehemals staatlichen Bahnen
\end{itemize}
\begin{definition}[Qualit\"at]
Qualit\"at ist der Grad, zu welchem Anforderungen an Produkte, Systeme und Dienstleistungen von diesen erf\"ullt werden.
\end{definition}
}

\frame{\frametitle{Zweck der G\"utepr\"ufung}
\framesubtitle{}
\begin{columns}[t] 
     \begin{column}[T]{6cm} 
     	\begin{itemize}
     		\item Zweck:
		\begin{itemize}
		\item Regelung des Umfangs der QS-Ma{\ss}nahmen
		\item Beschaffung f\"ur DB AG und verbundene Unternehmen
		\item Gilt auch f\"ur Unterlieferanten
		\end{itemize}
		\item Fokus auf Sicherheit (und Verf\"ugbarkeit)
		\item Achtung: nur kostenneutral, wenn Bestellung durch die DB vorliegt
     	\end{itemize}
     \end{column}
     	\begin{column}[T]{6cm} 
         	\begin{center}
            		\includegraphics[width=0.8\textwidth]{LgPS1}\source{}
        		\end{center}
     \end{column}
 \end{columns}
}

\frame{\frametitle{Pr\"ufstufen - Lieferanteneinstufung}
\framesubtitle{}
\begin{columns}[t] 
     \begin{column}[T]{6cm} 
     	\begin{itemize}
     		\item Pr\"ufstufe 1:
		\begin{itemize}
		\item Hochsicherheitsrelevante Teile, z.B.
		\begin{itemize}
		\item Fahrzeuge
		\item Bremsscheiben, Bremszylinder
		\end{itemize}
		\end{itemize}
		\item Pr\"ufstufe 2:
		\begin{itemize}
		\item Sicherheitsrelevante Teile, z.B.
		\begin{itemize}
		\item Herzst\"uck Kupplung
		\item Notausstiege
		\end{itemize}
		\end{itemize}
     	\end{itemize}
     \end{column}
     	\begin{column}[T]{6cm} 
         	\begin{itemize}
		\item Ermittelt im Rahmen der Herstellerbezogenen Produktqualifikation ``HPQ'':
		\begin{itemize}
		\item Q1: Stichprobenpr\"ufung f\"ur P1, Herstellerabnahme f\"ur P2
		\item Q2: 100\%-Pr\"ufung f\"ur P1, Stichprobenpr\"ufung f\"ur P2
		\item Q3: 100\%-Pr\"ufung f\"ur alle Lieferungen, Sperrung m\"oglich
		\end{itemize}
		\item F\"ur bestimmte Produkte, darunter Guss- und Schmiedeteile im sicherheitsrelevanten Bereich
		\end{itemize}     
		\end{column}
 \end{columns}
}

\frame{\frametitle{Qualifikationspflichtige Produkte und Fertigungsverfahren}
\framesubtitle{}
\begin{itemize}
\item Produkte
	\begin{itemize}
		\item Rads\"atze und Radsatzteile
		\item Gesenkschmiedeteile aus dem Bereich Zug- und Sto{\ss}einrichtung
		\item Zughaken, Schraubenkupplung
		\item Puffer
		\item Bremsklotzsohlen gegossen
		\item Bremsscheiben
		\item Radsatzlager
		\item Kunststoffk\"afige f\"ur Rollenlager
		\item Sicherheitsglas f\"ur Schienenfahrzeuge
		\item Molybd\"anbeschichtete Radsatzwellen
		\item Guss- und Schmiedeteile im sicherheitsrelevanten Bereich
	\end{itemize}
\item Fertigungsverfahren
	\begin{itemize}
		\item Gie{\ss}en
		\item Schmieden
		\item Pulverbeschichten
		\item Thermisches Spritzen
	\end{itemize}
\end{itemize}
}

\frame{\frametitle{Erstmusterpr\"ufung}
\framesubtitle{}
\begin{itemize}
\item Am ersten unter Serienbedingungen hergestellten Teil
\item Nachweis der Erf\"ullung der (Qualit\"ats-)Anforderungen
\item Erstmusterpr\"ufung durchzuf\"uhren bei:
\begin{itemize}
	\item Erstproduktionen
	\item Produkt\"anderungen
	\item Produktionsverlagerung
	\item \"Anderung von Produktionsverfahren
	\item \"Anderung der Produktions- oder Prozessabl\"aufe
	\item Aussetzen der Produktion mehr als 12 Monate
	\item Neuen Lieferanten
\end{itemize}
\item Vorab durchzuf\"uhren: 
\item Ergebnisse:
\begin{itemize}
	\item Freigabe f\"ur Serienfertigung
	\item Freigabe f\"ur Serienfertigung mit Auflagen
	\item Gesperrt f\"ur Serienfertigung
\end{itemize}
\end{itemize}
}

\frame{\frametitle{Typpr\"ufungen}
\framesubtitle{}
\begin{itemize}
\item Umfang der Typpr\"ufungen in Normen, Spezifikationen oder beh\"ordlich geregelt
\item Nachweis der Konformit\"at mit o.g. Anforderungen
\item Durchf\"uhrung vor Erstbemesterung bzw. Serienfertigung
\item Pr\"ufplan i. d. R. abzustimmen
\item Typnachweis bzw. Typpr\"ufbericht, evtl. mit Bewertung durch Sachverst\"andige
\end{itemize}
}

\frame{\frametitle{Pr\"ufpunkte}
\framesubtitle{}
\begin{itemize}
\item A-Punkt:
\begin{itemize}
	\item Abstimmungspflichtiger Pr\"ufpunkt
	\item Schriftliche Meldung an zust\"andigen Pr\"ufingenieur
	\item Anwesenheit Pr\"ufingenieur verpflichtend
\end{itemize}
\item F-Punkt:
\begin{itemize}
		\item Meldepunkt
		\item Schriftliche Meldung an zust\"andigen Pr\"ufingenieur
		\item Anwesenheit Pr\"ufingenieur optional (Entscheidung Pr\"ufingenieur)
		\item Pr\"ufung am n\"achstm\"oglichen Pr\"ufpunkt in Anwesenheit des Pr\"ufingeniurs nachholen
\end{itemize}
\item S-Punkt:
\begin{itemize}
		\item Stichprobenpr\"ufung
		\item Schriftliche Meldung an zust\"andigen Pr\"ufingenieur
		\item Anwesenheit Pr\"ufingenieur optional (Entscheidung Pr\"ufingenieur)
		\item Pr\"ufung muss nicht vom Pr\"ufingenieur \"uberwacht werden
\end{itemize}
\end{itemize}
}

\frame{\frametitle{}
\framesubtitle{}
         	\begin{center}
            		\includegraphics[width=0.95\textwidth]{LgPS19}\source{Quelle: \cite[S. 19]{dblgp}}
        		\end{center}
 }
 
 \frame{\frametitle{DB Richtlinie 951}
\framesubtitle{DB Richtlinie 951 klassifiziert Schwei{\ss}n\"ahte stenger als EN 15085.}
\begin{itemize}
\item DB-Gruppe 1:
	\begin{itemize}
		\item z.B. Drehgestellrahmen, Untergestell, Zug- und Sto{\ss}einrichtung, Schwingungs- und Sto{\ss}d\"ampfer
		\item Schwei{\ss}technische Bauweisepr\"ufung Teil 1 und 2 erforderlich, CL 1 nach EN 15085
	\end{itemize}
\item DB-Gruppe 2:
	\begin{itemize}
		\item z.B. Einstiegst\"uren, Drehgestellanbauten, Kabelkupplungen an automatischen Kupplungen
		\item Schwei{\ss}technische Bauweisenpr\"ufung Teil 1 erforderlich, CL 1 nach EN 15085
	\end{itemize}
\item DB-Gruppe 3:
	\begin{itemize}
		\item z.B. Innenausbau, Tragrahmen innen, WC-Bauteile und Wasserbeh\"alter
		\item Schwei{\ss}technische Bauweisenpr\"ufung nicht erforderlich, CL 2 nach EN 15085
	\end{itemize}
\item DB-Gruppe 4:
	\begin{itemize}
		\item z.B. Halter f\"ur Schilder, Tritte, Griffe, Gel\"ander innen 
		\item Schwei{\ss}technische Bauweisepr\"ufung nicht erforderlich, CL 3 nach EN 15085
	\end{itemize}
\end{itemize}
}


\frame{\frametitle{Zeugnistypen gem\"a{\ss} EN 10204}
\framesubtitle{Die EN 10204 wird h\"aufig zur Spezifikation der Zeugnistypen (``Pr\"ufbescheinigung'') herangezogen.}
\small
         	\begin{center}
            		\begin{tabular}{|l|p{2cm}|p{5.5cm}|p{2.5cm}|} 
			\hline 
			Art & Bezeichnung & Inhalt & Erstellt durch\\ \hline
			2.1 & Werksbe-scheinigung & Bestätigung der Übereinstimmung mit der Bestellung& Hersteller \\ \hline
			2.2 & Werkszeugnis & Bestätigung der Übereinstimmung mit der Bestellung unter Angabe von Ergebnissen nichtspezifischer Prüfung& Hersteller \\ \hline
			3.1 & Abnahme-pr\"ufzeugnis 3.1 & Bestätigung der Übereinstimmung mit der Bestellung unter Angabe von Ergebnissen spezifischer Prüfung& Unabh\"angige Stelle des Herstellers \\ \hline
			3.2 & Abnahme-pr\"ufzeugnis 3.2 & Bestätigung der Übereinstimmung mit der Bestellung unter Angabe von Ergebnissen spezifischer Prüfung& Unabh\"angige Stelle des Herstellers und Abnehmer des Kunden o.\"a.\\ \hline
			\end{tabular}
        		\end{center}
    }
