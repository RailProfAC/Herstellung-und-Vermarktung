% !TEX root = SFV-15014_HuV.tex
\section{Schwei{\ss}en nach EN 15085}
\sectionpage

\frame{\frametitle{Schwei{\ss}en an Schienenfahrzeugen - Allgemeines}
\framesubtitle{}
\begin{definition}[Schwei{\ss}en]
Schwei{\ss}en bezeichnet das unl\"osbare Verbinden von Bauteilen unter Anwendung von W\"arme oder Druck, mit oder ohne Schwei{\ss}zusatzwerkstoffe. 
\end{definition}
\textbf{Anwendung in der Schienenfahrzeugtechnik}
\begin{itemize}
	\item Verbindungen: z.B. Wagenkasten und Drehgestellfertigung
	\item Reparaturschwei{\ss}ungen im Stahlgussprozess
	\item Auftragsschwei{\ss}ungen, z.B. verschlei{\ss}mindernde Schichten
\end{itemize}
\textbf{Herausforderungen in der Schienenfahrzeugtechnik}
\begin{itemize}
	\item Lange Produktlebensdauer, hohe Schwingspielzahlen
	\item Hohes Sicherheitsbed\"urfnis
	\item Zertifizierungs und Pr\"ufaufwand
\end{itemize}
}

\frame{\frametitle{Prozessschritte Herstellung Schwei{\ss}verbindungen}
\framesubtitle{}
\begin{enumerate}
\item Wareneingangspr\"ufung (mind. Zeugnisse)
\item Schwei{\ss}nahtvorbereitung
\begin{itemize}
	\item i.d.R. maschinelle Bearbeitung
	\item verh\"altnism\"a{\ss}ig eng toleriert
	\item je nach Nahtform Badsicherung erforderlich
\end{itemize}
\item Aufnahme in Vorrichtung
\item Heften
\item Entnahme aus Vorrichtung
\item Schwei{\ss}en
\item Pr\"ufung duch Schwei{\ss}er/Bediener
\item Ggf. Nachbearbeitung der Schwei{\ss}naht 
\item Pr\"ufung durch Pr\"ufpersonal bzw. Werkerselbstpr\"ufung
\item Ggf. Abnahme duch Kunden
\item Bearbeitung zur Erreichung von Schnittstellenma{\ss}en
\end{enumerate}
}

\frame{\frametitle{Prozessschritte Konstruktion Schwei{\ss}verbindungen}
\framesubtitle{}
\begin{enumerate}
\item Technische Vertragspr\"ufung
\item Konstruktionsentwurf
\begin{itemize}
	\item Ermittlung Beanspruchungszustand
	\item Ermittlung Sicherheitsbed\"urfnis
\end{itemize}
\item Freigabe durch verantwortliche Schwei{\ss}aufsichtsperson (vSAP)
\begin{itemize}
	\item F\"ur DB: Ggf. Schwei{\ss}technische Bauweisenpr\"ufung Teil 1 (STBP 1) 
\end{itemize}
\item Festlegung ben\"otigter Verfahrens- und Arbeitsproben
\item Durchf\"uhrung und Analyse Verfahrens- und Arbeitsproben (evtl. iterativ)
\item Fertigung der Bauteile
\begin{itemize}
	\item F\"ur DB: Ggf. Schwei{\ss}technische Bauweisenpr\"ufung Teil 2 (STBP 2) 
\end{itemize}
\end{enumerate}
}



\frame{\frametitle{Europ\"aische Normierung nach EN 15085}
\framesubtitle{Bahnanwendungen - Schwei{\ss}en von Schienenfahrzeugen und -fahrzeugteilen}
\textbf{Teile der EN 15085}
\begin{enumerate}
	\item Allgemeines, Begriffe
	\item Qualit\"atsanforderungen und Zertifizierung von Schwei{\ss}betrieben
	\item Konstruktionsvorgaben
	\item Fertigungsanforderungen
	\item Pr\"ufung und Dokumentation
\end{enumerate}
\textbf{Anwendungsbereich}
\begin{itemize}
	\item Schwei{\ss}en metallischer Werkstoffe 
	\begin{itemize}
		\item Pflicht f\"ur Stahl und Aluminium, auch Gusslegierungen)
		\item Fakultativ f\"ur andere Werkstoffe
		\end{itemize}
	\item Herstellung und Instandsetzung
	\item Ausnahme: spezielle Regelwerke, z.B. Druckbeh\"alter
\end{itemize}
}

\frame[allowframebreaks]{\frametitle{Begriffe}
\framesubtitle{In Anlehnung an \cite{en150851}}
\begin{definition}[Zertifizierungsstufe (CL)]
Stufe zur Klassifizierung der geschwei{\ss}ten Schienenfahrzeuge und geschwei{\ss}ter Komponenten on Abh\"angigkeit von der Schwei{\ss}nahtg\"uteklasse.
\end{definition}
\begin{definition}[Schwei{\ss}nahtg\"uteklasse (CP)]
G\"uteanforderungen an die Schwei{\ss}verbindung in Abh\"angigkeit von Beanspruchungszustand und von Sicherheitsbed\"urfnis der einzelnen Schwei{\ss}naht.
\end{definition}
\begin{definition}[Schwei{\ss}nahtpr\"ufklasse (CT)]
Durchzuf\"uhrende Pr\"ufung f\"ur die Schwei{\ss}verbindung in Abh\"angigkeit von der Schwei{\ss}nahtg\"uteklasse.
\end{definition}
\newpage
\begin{definition}[Hersteller]
Organisation, die
\begin{itemize}
		\item eine schwei{\ss}technische Fertigung zur Herstellung und Instandsetzung betreibt oder
		\item geschwei{\ss}te Komponenten konstruiert, einkauft oder vertreibt.
		\end{itemize}
\end{definition}
\begin{definition}[Statische Auslegung]
Dimensionierung von Schwei{\ss}verbindungen, bei der die Kennwerte der statischen Festigkeit eingehalten werden.
\end{definition}
\begin{definition}[Dauerfestigkeitsauslegung]
Dimensionierung von Schwei{\ss}verbindungen, bei der die Kennwerte der Erm\"udungsfestigkeit eingehalten werden.
\end{definition}
\begin{definition}[Ausnutzung der Beanspruchbarkeit]
Verh\"altnis zwischen berechneter Erm\"udungsfestigkeit und der durch den entsprechenden Sicherheitsfaktor abgeglichenen zul\"assigen Erm\"udungsfestigkeit.
\end{definition}
\begin{definition}[Zul\"assige Erm\"udungsfestigkeit]
Maximale Spannung, die unter Ber\"ucksichtigung eines speziellen Faktors f\"ur die Schwei{\ss}verbindung vom eingesetzten Werkstoff aufnehmbar ist.
\end{definition}
\begin{definition}[Sicherheitsbed\"urfnis]
Definiert die Auswirkungen eines Versagens einer einzelnen Schwei{\ss}naht im Hinblick auf die Folgen f\"ur Personen, Einrichtungen und die Umwelt.
\end{definition}
\newpage
\begin{definition}[Arbeitsprobe]
Musterschwei{\ss}verbindungen zum Nachweis der Handfertigkeit des Schwei{\ss}ers oder der bedingungsgem\"a{\ss}en Ausf\"uhrung von Schwei{\ss}verbindungen.
\end{definition}
}



\subsection{Schwei{\ss}nahtklassifizierung nach EN 15085}
\subsectionpage
\frame{\frametitle{Bestimmung der Ausnutzung der Beanspruchbarkeit}
\framesubtitle{}
\begin{itemize}
\item Verh\"altnis $S$ berechneter zu zul\"assiger Spannung der Verbindung
\begin{itemize}
	\item Bezogen auf Dauerfestigkeit
	\item Festigkeitsanforderungen gem\"a{\ss} EN 12663, EN 13749 oder nationalen Normen
	\item Bewertung der Festigkeit nach nationalen Regelwerken, z.B. DVS 1612
	\begin{itemize}
		\item Abh\"angig von Nahtform und Grundwerkstoff
		\item Betrachtung h\"oherfester Werkstoffe konservativ
		\end{itemize}	
	\item Alternativ Dauerversuch m\"oglich
\end{itemize}
\item Bei Berechnung nach Norm:
\begin{itemize}
		\item Beanspruchungszustand Hoch: $S \geq 0{,}9$
		\item Beanspruchungszustand Mittel: $0{,}75 \leq S < 0{,}9$
		\item Beanspruchungszustand Niedrig: $S < 0{,}75$ 
	\end{itemize}
\end{itemize}
}

\frame{\frametitle{Sicherheitsbed\"urfnis der Schwei{\ss}naht}
\framesubtitle{}
\begin{itemize}
\item Versagen einer einzelnen Schwei{\ss}naht f\"uhrt:
\begin{itemize}
	\item zwangsl\"aufig zu Ereignissen mit Personensch\"aden und Versagen der Gesamtfunktion $\rightarrow$ Hoch
	\item m\"oglicherweise zu Ereignissen mit Personensch\"aden und Beeintr\"achtigung der Gesamtfunktion $\rightarrow$ Mittel
	\item zu unwahrscheinlichen Ereignissen mit Personensch\"aden und keiner direkten Beeintr\"achtigung der Gesamtfunktion $\rightarrow$ Niedrig
\end{itemize}
\item Nur Betrachtung Einfachfehler
\end{itemize}
}

\frame{\frametitle{Schwei{\ss}nahtg\"uteklasse}
\framesubtitle{}
\begin{center}
\begin{tabular}{|c|c |c |c |}
\hline
\multirow{2}{*}{Beanspruchungszustand} & \multicolumn{3}{|c|}{Sicherheitsbed\"urfnis} \\ \cline{2-4}
 & Hoch & Mittel & Niedrig \\ \hline
Hoch & CP A & CP B & CP C2 \\ \hline
Mittel & CP B & CP C2 & CP C3 \\ \hline
Niedrig & CP C1 & CP C3 & CP D \\ \hline 
\end{tabular}
\end{center}
\begin{itemize}
	\item CP A: Nur f\"ur voll durchgeschwei{\ss}te und f\"ur \"Uberpr\"ufung w\"ahrend Fertigung und Instandhaltung zug\"angliche Schwei{\ss}n\"ahte
	\item CP B, Sicherheitsbed\"urfnis hoch: CP A: Nur f\"ur voll durchgeschwei{\ss}te und f\"ur \"Uberpr\"ufung w\"ahrend Fertigung und Instandhaltung zug\"angliche Schwei{\ss}n\"ahte
	\item Weitere Erg\"anzungen zu eingeschr\"ankt volumetrisch pr\"ufbaren Schwei{\ss}n\"ahten siehe \cite[Tabelle 2]{en150853}.
\end{itemize}
}

\frame{\frametitle{Schwei{\ss}nahtpr\"ufklassen}
\framesubtitle{}
\begin{center}
\begin{tabular}{|c|c|}
\hline
\multirow{2}{*}{Schwei{\ss}nahtg\"uteklasse} & Schwei{\ss}nahtpr\"ufklasse \\
& (Mindestanforderung) \\ \hline 
 CP A & CT 1 \\ \hline
 CP B & CT 2 \\ \hline
 CP C1 & CT 2 \\ \hline
 CP C2 & CT 3 \\ \hline
 CP C3 & CT 4 \\ \hline
 CP D & CT 4 \\ \hline
\end{tabular}
\end{center}
}

\frame{\frametitle{Schwei{\ss}nahtpr\"ufklassen}
\framesubtitle{}
\begin{center}
\begin{tabular}{|c|c|c|c|}
\hline
\multirow{2}{*}{Schwei{\ss}nahtpr\"ufklasse} & Volumetrisch & Oberfl\"ache & Sichtpr\"ufung \\
& RT oder UT & MT oder PT &  VT\\ \hline 
 CT 1 & 100 \% & 100 \% & 100 \%\\ \hline
 CT 2 & 10 \% & 10 \% & 100 \%\\ \hline
 CT 3 & n/a & n/a & 100 \%\\ \hline
 CT 4 & n/a & n/a & 100 \%\\ \hline
\end{tabular}
\end{center}
\begin{itemize}
	\item Jeweils Mindestanforderungen
	\item CT 3: VT durch Pr\"ufpersonal (damit CP 2 CT 3 \"ahnlich SGK 2.3 nach DIN 6700)
	\item CT 4: VT als Werkerselbstpr\"ufung, Dokumentation nicht erforderlich
	\item Falls volumetrische Pr\"ufung nicht m\"oglich: ersatzweise 100\% Oberfl\"achenpr\"ufung und Arbeitsprobe
\end{itemize}
}

\frame{\frametitle{Zertifizierung der Schwei{\ss}betriebe}
\framesubtitle{}
\begin{itemize}
\item[CL 1:] Schwei{\ss}en der G\"uteklassen CP A bis CP D, Handel und Konstruktion
\item[CL 2:] Schwei{\ss}en der G\"uteklassen CP C2 bis CP D, Handel un Konstruktion nach CL 2 und CL 3 ist eingeschlossen
\item[CL 3:] Schwei{\ss}en der G\"uteklasse CP D
\item[CL 4:] Schwei{\ss}konstruktion von Teilen der CL 1 bis CL 3 sowie Handel mit solchen Teilen
\end{itemize}
}


\subsection{Schwei{\ss}en an Schienenfahrzeugen - Aluminium und Stahl}
\subsectionpage

\frame[allowframebreaks]{\frametitle{Konstruktive Grundlagen}
\framesubtitle{}
\begin{itemize}
\item Vermeiden:
\begin{itemize}
	\item Scharfe Ecken
	\item Querschnitts\"anderungen
	\item Gemischte Verbindungsarten (z.B. Schwei{\ss}en und Schrauben)
	\item Anh\"aufungen von Schwei{\ss}n\"ahten
	\item Quern\"ahte zur Befestigung untergeordneter Teile bei Zugbeanspruchung
\end{itemize}
\item Zug\"anglichkeit zum Schwei{\ss}en und Pr\"ufen gew\"ahrleisten
\item Kaltverformte Bereiche (einschlie{\ss}lich Umgebung $5 t$):
\begin{itemize}
	\item Schwei{\ss}en nur an Teilen der CL 3 zul\"assig
	\item F\"ur CL 1 und CL 2: 
	\begin{itemize}
		\item Normalgl\"uhen 
		\item Eingeschr\"ankte Radien \cite[Tabelle 9]{en150853}
		\end{itemize}
\end{itemize}
\newpage
\item Stumpfn\"ahte an Bauteilen unterschiedlicher Dicke
\begin{itemize}
		\item \"Ubergang mit Neigung:
		\begin{itemize}
		\item CP C3 und CP D: Neigung 1:1
		\item CP A, CP B, CP C1 und CP C2: Neigung 1:4
		\end{itemize}
		\end{itemize}
\item Abstand zwischen Schwei{\ss}n\"ahten: i.d.R. 50 mm
\item Freischnitte sollen Umschwei{\ss}barkeit gew\"ahrleisten
\item Randabstand Kehlnaht $\geq 1{,}5 a + t$
\item Korrosionsschutz.
\begin{itemize}
		\item Umschwei{\ss}en
		\item R\"uckseite abdichten: Dichtschwei{\ss}en, Acryl, ...
		\end{itemize}
\end{itemize}
}

\frame{\frametitle{Angaben auf Schwei{\ss}zeichnungen}
\framesubtitle{}
\begin{itemize}
\item Schwei{\ss}nahtg\"uteklasse
\begin{itemize}
	\item falls einheitlich: in Zeichung
	\item falls unterschiedlich: nahe bei der Schwei{\ss}naht
\end{itemize}
\item Zertifizierungsstufe
\begin{itemize}
	\item je Bauteil
\end{itemize}
\item Schwei{\ss}nahtform
\item Schwei{\ss}nahtdicke
\item Schwei{\ss}nahtl\"ange
\item Schwei{\ss}zus\"atze (in Zeichnungen, St\"ucklisten oder anderen Dokumenten)
\end{itemize}
}



%\offslide{Schwei{\ss}nahtformen 1}
%
%\offslide{Schwei{\ss}nahtformen 2}
%
%\offslide{DVS 1612}

%\subsection{Schwei{\ss}en an Schienenfahrzeugen - Auslastungs- und Festigkeitsberechnungen}
%\subsectionpage
