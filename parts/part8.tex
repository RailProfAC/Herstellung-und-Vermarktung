% !TEX root = SFV-15014_HuV.tex
\section{Einf\"uhrung Lean Manufacturing}
\sectionpage

\frame{\frametitle{Schlanke Fertigung bei der Sendung mit der Maus}
\framesubtitle{}
\begin{center}
\url{http://www.ardmediathek.de/tv/Die-Sendung-mit-der-Maus/Die-Sendung-mit-der-Maus-25-11-2012-Fl/Das-Erste/Video-Podcast?documentId=12567908&bcastId=1458}
\end{center}
}

\frame{\frametitle{Lean Production bei Porsche}
\framesubtitle{}
\begin{center}
On July 27, 1994, something remarkable happened in the assembly hall of the Porsche company in Stuttgart, Germany. A Porsche Carrera rolled off the line with nothing wrong with it. The army of blue-coated craftsmen waiting in the vast rectification area could pause for a moment because, for the first time in forty-four years, they had nothing to do. This was the first defect-free car ever to roll off a Porsche assembly line or to emerge from the earlier system of bench assembly. This first perfect Porsche—and there have been many more since—was a small but highly visible milestone in the efforts of Chairman Wendelin Wiedeking and his associates to introduce lean thinking into a veritable industrial institution—indeed, into one of the great symbols of the German industrial tradition. [...] What’s more, there’s already evidence that when lean concepts are married to the strengths of the German tradition, embodied in the concept of superior technology, or technik, a remarkably competitive hybrid form can emerge. \citep{womack2010}
\end{center}
}


\frame{\frametitle{Lean als Produktionssystem}
\framesubtitle{}
     	\begin{itemize}
		\item Eingef\"uhrt von japanischen Automobilunternehmen
		\item Starke Fokussierung auf Kundennutzen
     		\item Im Gegensatz zu gepufferter Produktion
		\item Ziele
		\begin{itemize}
		\item Kompetenz und Verantwortung zusammenf\"uhren
		\item in Netzwerken arbeiten
		\item Verschwendung und Fehler vermeiden
		\item Abl\"aufe synchronisieren
		\item Kontinuierlich im Kleinen besser werden
		\item bei Bedarf im Gro{\ss}en \"andern
		\end{itemize}
     	\end{itemize}
   }
   
   \frame{\frametitle{Elemente des Lean Manufacturing}
\framesubtitle{}
\begin{columns}[t] 
     \begin{column}[T]{6cm} 
     	\begin{itemize}
     		\item Angemessene technische Ausstattung
		\item Wenig hierarchische Arbeitsorganisation
		\item Konsequentes Qualit\"atsmanagement
		\item Kontinierliche Verbesserung
		\item Qualifikation und Motivation
		\item Just-In-Time/Sequence, Pull
		\item Wertsch\"opfungsorientierung
     	\end{itemize}
     \end{column}
     	\begin{column}[T]{6cm} 
         	\begin{center}
            		\includegraphics[width=0.8\textwidth]{LeanHouse}\rotatebox{90}{\tiny \color{gray}Quelle: Laurens van Lieshout}
        		\end{center}
     \end{column}
 \end{columns}
}

\frame{\frametitle{Vermeidung von Verschwendung}
\framesubtitle{Sieben Arten der Verschwendung \textit{muda} werden genannt}
\begin{enumerate}
\item Transport: Kein Kundennutzen durch Wege
\item Best\"ande: Binden Kapital, Fl\"ache, erzeugen Handhabungsaufwand
\item Bewegung: Mehr Bewegung als der Prozess ben\"otigt
\item Warten: Wartezeiten erzeugen keinen Kundennutzen
\item \"Uberproduktion: Kein Kunde, kein Nutzen
\item Aufw\"andige Prozesse: Fehleranf\"allig, unflexible Prozesse
\item Fehler: Kein Kundennutzen durch Fehlersuche und -behebung \color{red!80!black}{$6 \sigma$}
\end{enumerate}
}

\frame{\frametitle{Tools auf dem Weg zum Lean Manufacturing}
\framesubtitle{}
\begin{itemize}
\item 5 S: Sortiere aus! Stelle ordentlich hin! S\"aubere! Sauberkeit bewahren! Selbstdisziplin \"uben!
\item One-Piece-Flow: Losgr\"o{\ss}enreduzierung. Ben\"otigt kurze R\"ustzeiten.
\item Visual Management: Zustand des Prozesses, Verbesserungen etc. visualieren.
\item Jidoka: Fehler an der Quelle finden.
\item Poka Yoke: Vermeiden ``ungl\"ucklicher'' Fehler, z.B. durch Kodierung.
\item Heijunka: Nivellierung des Produktionslevels.
\item Kanban: Bedarf steuert Produktion.
\item Andon: Zustand des Prozesses in Echtzeit abbilden.
\item Kaizen: Stetige Verbesserung.
\item Genba: Ort der Produktion.
\item Obeya: ``Gro{\ss}er Raum''.
\item \color{red!80!black}{\textbf{Genchi Genbutsu: Go and see for yourself!}}
\end{itemize}
}

\frame{\frametitle{Lean Development}
\framesubtitle{}
\begin{itemize}
\item Wert: Spezifiziere den Wert deines Produktes
\item Wertstrom: Erkenne den Wertstrom 
\item Flow: Erzeuge einen Wertstromfluss ohne Unterbrechungen
\item Pull: Lasse den Kunden den Takt der Bearbeutung bestimmen
\item Perfektion: Verbessere die Dinge kontinuierlich
\end{itemize}
}

\frame{\frametitle{Agile Development}
\framesubtitle{}
\begin{itemize}
\item Werte (Agile Manifesto):
\begin{itemize}
	\item Menschen und Interaktionen mehr als Prozesse und Werkzeuge
	\item Funktionierende Software [Produkte] mehr als umfassende Dokumentation
	\item Zusammenarbeit mit dem Kunden mehr als Vertragsverhandlung
	\item Reagieren auf Ver\"anderung mehr als Befolgen eines Plans
\end{itemize}
\item Prinzipien:
\begin{itemize}
	\item Kurze Iterationen
	\item Einfachheit
	\item Selbstorganisation
	\item Pers\"onliche Kommunikation
	\item Teamarbeit
\end{itemize}
\end{itemize}
}


