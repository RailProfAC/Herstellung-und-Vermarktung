% !TEX root = SFV-15014_HuV.tex
\section{Inbetriebsetzung}
%\frame{\sectionpage}

\subsection{Root-Cause-Analysis}
%\frame{\subsectionpage}

\frame{\frametitle{Aufgaben der Root-Cause-Analysis}
\framesubtitle{}
\begin{itemize}
\item Im Verlauf der IBS h\"aufig:
\begin{itemize}
		\item Systematische Abweichungen
		\item Ausf\"alle
		\item Kundenbeschwerden
		\end{itemize}
\item Typisch: nur Symptome werden geschildert
	\begin{itemize}
		\item Auftreten sporadisch
		\item Zugang zum Fahrzeug eingeschr\"ankt
		\item Nicht reproduzierbar
		\item Keine Referenzsystem vorhanden
	\end{itemize}
\item Problem: vorgefertigte Meinungen zur Ursache
\item M\"oglichkeiten:
	\begin{itemize}
		\item Ishikawa-Diagramm
		\item 5-Why
	\end{itemize}
\end{itemize}
}


\frame{\frametitle{Ishikawa-Diagramm}
\framesubtitle{Hauptkategorien im Ishikawa-Diagramm variieren nach Anwendungskontext}
	\begin{center}
            		\includegraphics[width=0.95\textwidth]{Ishikawa}
        	\end{center}
}

\frame{\frametitle{5-Why}
\framesubtitle{}
\begin{itemize}
\item Einfaches, direktes Fragen nach der Ursache f\"uhrt h\"aufig nicht zur Root-Cause
\item Wiederholtes Fragen kommt ``tiefer''
\item Generell akzeptiert sind 5 Why
\end{itemize}
}


