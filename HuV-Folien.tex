\documentclass[slidestop,compress,mathserif, aspectratio = 1610]{beamer}
\usecolortheme{seagull} % Beamer color theme
\useinnertheme{rectangles}
\usepackage[round]{natbib}
\usepackage{bibentry}
\usepackage{graphicx}
\usepackage{epigraph}
\usepackage{makeidx}
\usepackage[]{amsmath}
\usepackage[]{amssymb}
\usepackage{color}
\usepackage{pict2e}
\usepackage{algorithm2e}
\usepackage[ngerman]{babel}
\usepackage{tikz}
\usetikzlibrary{mindmap}
\usepackage{multirow}
\usepackage[utf8]{inputenc}
\usepackage{multimedia}
\usepackage{xcolor}
\usepackage{keyval}
\usepgfmodule{shapes}
\usepackage{pgfplots}
\usepackage{sansmath}
\usepackage{pgfgantt}
\usetikzlibrary{arrows}
\usepackage{colortbl}
\usepackage{array}


\definecolor{mint}{rgb}{0,177,172}
\graphicspath{{../../figures/}}

\newcommand{\source}[1]{\rotatebox{90}{\tiny \color{gray} #1}}


\mode<handout>{
\usepackage{pgfpages}
\pgfpagesuselayout{2 on 1}[a4paper,border shrink=5mm]
}

%\AtBeginSection{\frame{\sectionpage}}
\newtranslation[to=ngerman]{Section}{Teil}
\newtranslation[to=ngerman]{Subsection}{Abschnitt}

\pgfplotsset{width=6cm, 
	tick label style = {font=\sansmath\sffamily},
  every axis label = {font=\sansmath\sffamily},
  legend style = {font=\sansmath\sffamily},
  label style = {font=\sansmath\sffamily}}
\definecolor{RYB1}{RGB}{240,249,232}
\definecolor{RYB2}{RGB}{186,228,188}
\definecolor{RYB3}{RGB}{123,204,196}
\definecolor{RYB4}{RGB}{67,162,202}
\definecolor{RYB5}{RGB}{18,104,172}

\definecolor{RYB11}{RGB}{141,211,199}
\definecolor{RYB12}{RGB}{255,255,179}
\definecolor{RYB13}{RGB}{190,186,218}
\definecolor{RYB14}{RGB}{251,128,114}
\definecolor{RYB15}{RGB}{128,177,211}
\definecolor{RYB16}{RGB}{253,180,98}
\definecolor{RYB17}{RGB}{179,222,105}
\definecolor{RYB18}{RGB}{252,205,229}
\definecolor{RYB19}{RGB}{217,217,217}
\definecolor{RYB20}{RGB}{188,128,189}
\definecolor{RYB21}{RGB}{204,235,197}
\definecolor{RYB22}{RGB}{255,237,111}



%%\pgfplotsset{
%    select row/.style={
%        x filter/.code={\ifnum\coordindex=#1\else\def\pgfmathresult{}\fi}
%    }
%}

\begin{document}

\tikzstyle{element} = [draw,rectangle, align = center, text width = 2.5 cm, node distance = 1.25 cm]


%% Formalia
\newcommand{\revision}{Rev. B}
\newcommand{\docnum}{SFV-15014}
\newcommand{\offslide}[2]{\frame{\frametitle{\includegraphics[scale=0.01]{Off} \hspace{.1cm} #1}
%\newcommand{\tikzmark}[1]{\tikz[overlay,remember picture] \node (#1) {};}
\framesubtitle{#2}
\begin{center} \end{center}}}

\title[86512 HuV]{86512 Herstellung und Vermarktung von Schienenfahrzeugen}
\subtitle{}
\author[R. Pfaff]{Prof. Dr. Raphael Pfaff}
\institute[Dokument \docnum, \revision]{Fachhochschule Aachen}
\logo{\put(-18, 200){\includegraphics[width=1.4cm]{logoR}}}

% Pr\"aliminarien
% !TEX root = ../HuV-Folien.tex
\begin{frame} % Cover slide
\titlepage
\end{frame}
\section{Einf\"uhrung in die Veranstaltung}

%\frame{\frametitle{Ihre Dozenten}
%\framesubtitle{}
%\begin{columns}[t] 
%     \begin{column}[T]{5cm} 
%     	\begin{center}
%         \includegraphics[width=0.8\textwidth]{Axel} 
% 	\end{center}
%	Prof. Dr. Axel Thomas\\
%	Gesch\"aftsf\"uhrer\\
%	WFG Aachen\\
%	A.Thomas@wfg-aachen.de
%     \end{column}
%     	\begin{column}[T]{5cm} 
%         	\begin{center}
%          \includegraphics[width=0.8\textwidth]{Raphael}
% 	\end{center}
%	Prof. Dr. Raphael Pfaff \\
%	Schienenfahrzeugtechnik\\
%	pfaff@fh-aachen.de\\
%	@RailProfAC
%     \end{column}
% \end{columns}
%}
%
%
%\frame{\frametitle{Vorstellungsrunde}
%\framesubtitle{Beispielfragen - stellen Sie sich vor!}
%\begin{itemize}
%\item (Schul-)Bildung?
%\item (Berufs-)Erfahrung?
%\item Warum Schienenfahrzeugtechnik?
%\item Was erwarte ich vom BEng Schienenfahrzeugtechnik?
%\item Was erwarte ich von Betriebswirtschaft und Technik der Eisenbahnen?
%\item Sprachkenntnisse (insbesondere Englisch)?
%\item Bus/Bahn/Auto/Fahrrad/...?
%\item Hobbies?
%\item ...
%\end{itemize}
%\note{\"Ubung Name mit/ohne BEng}
%}
%
%\frame{\frametitle{Anforderungen ``First Cycle'' - Bachelor}
%\framesubtitle{Anforderungen gem\"a{\ss} Dublin Descriptors \citep{joint2004shared}}
%\begin{columns}[t] 
%     \begin{column}[T]{7cm} 
%     	\begin{itemize}
%     		\item Knowledge and understanding in a field of study
%		\begin{itemize}
%		\item Typically supported by textbooks
%		\item Some aspects informed by knowledge on the forefront of the field of study
%		\end{itemize}
%		\item Apply knowledge and understanding indicating a professional approach
%		\item Gather and interpret data to inform judgement
%		\item Communicate information, ideas, problems and solutions
%		\item Learning skills to undertake further study with high degree of autonomy
%     	\end{itemize}
%     \end{column}
%     	\begin{column}[T]{5cm} 
%         	\begin{center}
%	\vspace{1cm}
%            		\includegraphics[width=0.8\textwidth]{GraduationHat}
%        		\end{center}
%     \end{column}
% \end{columns}
%}
%
%
%\frame{\frametitle{Anforderungen ``Niveau 6'' - Bachelor}
%\framesubtitle{Anforderungen gem\"a{\ss} Deutschem Qualifizierungsrahmen}
%\begin{columns}[t] 
%     \begin{column}[T]{7cm} 
%     	\begin{itemize}
%     		\item Breites und integriertes Wissen
%		\begin{itemize}
%		\item Wissenschaftliche Grundlagen
%		\item Praktische Anwendungen 
%		\end{itemize}
%		\item Breites Spektrum an Methoden
%		\begin{itemize}
%		\item Neue L\"osungen erarbeiten und bewerten
%		\end{itemize}
%		\item Verantwortlich in Expertenteams arbeiten oder leiten
%		\item Ziele f\"ur Lern- und Arbeitsprozesse definieren, reflektieren und bewerten
%		\item Lern- und Arbeitsprozesse eigenst\"andig und nachhaltig gestalten
%     	\end{itemize}
%     \end{column}
%     	\begin{column}[T]{5cm} 
%         	\begin{center}
%	\vspace{1cm}
%            		\includegraphics[width=0.8\textwidth]{GraduationHat}
%        		\end{center}
%     \end{column}
% \end{columns}
%}
%
%
%\frame{\frametitle{Anforderungen BEng Schienenfahrzeugtechnik}
%\framesubtitle{Was zeichnet einen Bachelor der Schienenfahrzeugtechnik aus?}
%\begin{columns}[t] 
%     \begin{column}[T]{6cm} 
%     	\begin{itemize}
%		\item Wissenschaftliches Arbeiten
%		\begin{itemize}
%		\item Nutzung Prim\"arliteratur und Normen
%		\item Erstellung Seminararbeiten
%		\end{itemize}
%     		\item Selbstlernkompetenz
%		\begin{itemize}
%		\item Beispiel: Nutzung Lehrbuch statt Skript
%		\end{itemize}
%		\item Verfassung wissenschaftlicher und technischer Texte
%		\item Fachvortrag zu Seminararbeit
%		\item Bericht zum Praktikum
%     	\end{itemize}
%     \end{column}
%     	\begin{column}[T]{6cm} 
%         	\begin{center}
%	\vspace{1cm}
%            		\includegraphics[width=0.8\textwidth]{GraduationHat}
%        		\end{center}
%     \end{column}
% \end{columns}
%}

\frame{\frametitle{Themenplan}
\framesubtitle{Plan - Verschiebungen sind m\"oglich!}
\begin{center}
\small
\vspace{-.5cm}
\begin{tabular}{|c|l|l|}
\hline
Datum & Thema & Dozent\\ \hline
 & Verkaufsgespräche richtig führen & Thomas\\ \hline
 & Verkaufsgespräche richtig führen & Thomas\\ \hline
 & Vertragsinhalte & Thomas\\ \hline
 & Vertragsinhalte & Thomas\\ \hline
 & Verkaufsgespr\"ache/Vertragsinhalte & Thomas\\ \hline
 & Kostenmanagement & Thomas\\ \hline
 & Finanzierungsaspekte zur Vertriebsunterstützung & Thomas\\ \hline
29.5. & Einf\"uhrung, Marktsegmente, Marktschranken & Pfaff\\ \hline
 & Lebenszyklusmodelle, Projektmanagement & Pfaff\\ \hline
19.6. & Requirements Engineering, Aufwandssch\"atzung & Pfaff\\ \hline
26.6. & Projektplanung & Pfaff\\ \hline
3.7. & Schwei{\ss}en  & Pfaff\\ \hline
 & Schrauben & Pfaff\\ \hline
 & Korrosionsschutz, DB G\"utepr\"ufung, IBS & Pfaff\\ \hline
 & Lessons learned Railway Challenge & Pfaff \\ \hline
\end{tabular}
\end{center}
}



% Marketing
\frame{\subsectionpage}% !TEX root = SFV-15014_HuV.tex
\section{Marketing}
%\sectionpage

\subsection{Marktsegmentierung}
\frame{\subsectionpage}

\frame{\frametitle{Warum Marktsegmentierung?}
\framesubtitle{Any customer can have a car painted any colour that he wants so long as it is black. - Henry Ford}
\begin{itemize}
\item Marktidentifizierung
\begin{itemize}
		\item Abgrenzung des relevanten Gesamtmarktes
		\item Bestimmung der relevanten Teilmärkte
		\item Auffinden vernachlässigter Teilmärkte (Marktlücken, Marktnischen)
\end{itemize}
\item Rechtzeitige Beurteilung von Neueinführungen der Konkurrenz und rechtzeitiges Ergreifen von Gegenmaßnahmen
\item Beurteilung der eigenen Markenpositionierung im Vergleich zur Positionierung der Konkurrenzprodukte
\item Richtige Positionierung von Neuprodukten
\item Fundierte Prognose der (segmentspezifischen) Marktentwicklung
\item Optimale Allokation des Budgets auf einzelne Segmente
\item Erhöhung der Zielerreichungsgrade 
\item Preisfindung
\end{itemize}
}

\offslide{Segmentierungskriterien f\"ur Schienenfahrzeuge und ihre Komponenten}

\subsection{Marktgrundlagen}
\frame{\subsectionpage}

\frame{\frametitle{Gr\"o{\ss}e und Entwicklung des Marktes}
\framesubtitle{}
\begin{columns}[t] 
     \begin{column}[T]{4cm} 
     	\begin{itemize}
     		\item Gesamtmarkt Schienenfahrzeuge weltweit: 47 Mrd EUR (Stand 2012)
		\item Dominierende Teilm\"arkte:
		\begin{itemize}
		\item Asien
		\item Europa
		\end{itemize}
		\item Relevante Fahrzeugsegmente variieren lokal stark
     	\end{itemize}
     \end{column}
     	\begin{column}[T]{8cm} 
         	\begin{center}
		\begin{tikzpicture}%[scale = 0.9]
		\begin{axis}[ybar stacked,
		ylabel={bn EUR/a},
		legend style={at={(0.5,-0.4)},
		anchor=north, legend columns=3},
		symbolic x coords={EMEA, ASEAN, CIS, E. Europe, NAFTA, RoA, W. Europe},
		xtick=data,
		x tick label style={rotate=45,anchor=east},
		title={Conventional Rail Market Size}]]
		\addplot[fill = RYB1, draw = RYB1!50!black] coordinates
			{(EMEA,0.220) (ASEAN,1.367) (CIS, 1.219) (E. Europe, .248) (NAFTA, .192) (RoA, .037) (W. Europe, .421)};
		\addplot[fill = RYB2, draw = RYB2!50!black] coordinates
		{(EMEA,.294) (ASEAN,.823) (CIS, .196) (E. Europe, .103) (NAFTA, .992) (RoA, .218) (W. Europe, .244)};
		\addplot[fill = RYB3, draw = RYB3!50!black] coordinates
		{(EMEA,.012) (ASEAN, .067) (CIS, .212) (E. Europe, .026) (NAFTA, .027) (RoA, .005) (W. Europe, .128)};
		\addplot[fill = RYB4, draw = RYB4!50!black] coordinates
		{(EMEA,.172) (ASEAN,.330) (CIS, .686) (E. Europe, .196) (NAFTA, .383) (RoA, .209) (W. Europe, .501)};
		\addplot[fill = RYB5, draw = RYB5!50!black] coordinates
		{(EMEA,.129) (ASEAN, .932) (CIS, 2.671) (E. Europe, .281) (NAFTA, 1.337) (RoA, .125) (W. Europe, .582)};
		\legend{E-Locos, D-Locos, Shunters, Coaches, Wagons};
		\end{axis}
		\end{tikzpicture}
        		\end{center}
     \end{column}
 \end{columns}
}

\frame{\frametitle{Gr\"o{\ss}e und Entwicklung des Marktes}
\framesubtitle{}
\begin{columns}[t] 
     \begin{column}[T]{6cm} 
     	\begin{center}
		\begin{tikzpicture}[scale = 0.9]
		\begin{axis}[ybar stacked, 
		colormap/bluered,
		ylabel={bn EUR/a},
		legend style={at={(0.5,-0.4)},
		anchor=north, legend columns=2},
		symbolic x coords={EMEA, ASEAN, CIS, E. Europe, NAFTA, RoA, W. Europe},
		xtick=data,
		x tick label style={rotate=45,anchor=east},
		title={Multiple Unit Market Size}]]
		\addplot[fill = RYB2, draw = RYB2!50!black] coordinates
		{(EMEA,0) (ASEAN,0.049) (CIS, 0) (E. Europe, 0) (NAFTA, 0) (RoA, 0) (W. Europe, 1.006)};
		\addplot[fill = RYB3, draw = RYB3!50!black] coordinates
		{(EMEA,0) (ASEAN, 1.638) (CIS, .893) (E. Europe, .668) (NAFTA, .283) (RoA, .463) (W. Europe, 3.342)};
		\addplot[fill = RYB5, draw = RYB5!50!black] coordinates
		{(EMEA,0) (ASEAN,0) (CIS, 0) (E. Europe, 0) (NAFTA, 0) (RoA, 0) (W. Europe, 0)};
		\legend{IC EMU, Regional EMU, IC DMU, Regional DMU};
		\addplot[fill = RYB4, draw = RYB4!50!black] coordinates
		{(EMEA,.033) (ASEAN,.512) (CIS, .024) (E. Europe, .145) (NAFTA, .054) (RoA, .004) (W. Europe, .556)};
		\end{axis}
		\end{tikzpicture}
        		\end{center}
     \end{column}
     	\begin{column}[T]{6cm} 
         	\begin{center}
		\begin{tikzpicture}[scale = 0.9]
		\begin{axis}[ybar stacked, 
		colormap/bluered,
		ylabel={m EUR/a},
		legend style={at={(0.5,-0.40)},
		anchor=north, legend columns=3},
		symbolic x coords={EMEA, ASEAN, CIS, E. Europe, NAFTA, RoA, W. Europe},
		xtick=data,
		x tick label style={rotate=45,anchor=east},
		title={Urban Vehicle Market Size}]]
		\addplot[fill = RYB1, draw = RYB1!50!black] coordinates
			{(EMEA,.054) (ASEAN,.152) (CIS, .082) (E. Europe, .392) (NAFTA, .563) (RoA, .043) (W. Europe, 1.309)};
		\addplot[fill = RYB2, draw = RYB2!50!black] coordinates
		{(EMEA,.257) (ASEAN,1.638) (CIS, .052) (E. Europe, .215) (NAFTA, 1.131) (RoA, .607) (W. Europe, .661)};
		\addplot[fill = RYB3, draw = RYB3!50!black] coordinates
		{(EMEA,.142) (ASEAN, .094) (CIS, 0) (E. Europe, 0) (NAFTA, .085) (RoA, .151) (W. Europe, .104)};
		\legend{LRV, Metro, APM};
		\end{axis}
		\end{tikzpicture}
        		\end{center}
     \end{column}
 \end{columns}
}

\frame{\frametitle{Gesamtmarkt Schienenfahrzeuge}
\framesubtitle{}
\begin{center}
\pgfplotsset{width=8.6cm}
		\begin{tikzpicture}[scale = 0.95]
		\begin{axis}[ybar stacked,
		ylabel={bn EUR/a},
		legend style={at={(1.5,1)}},
		%anchor=north, legend columns=3},
		symbolic x coords={EMEA, ASEAN, CIS, E. Europe, NAFTA, RoA, W. Europe},
		xtick=data,
		x tick label style={rotate=45,anchor=east},
		title={Rail Vehicle Market Size}]]
		\addplot[fill = RYB11, draw = RYB11!50!black] coordinates
			{(EMEA,0.220) (ASEAN,1.367) (CIS, 1.219) (E. Europe, .248) (NAFTA, .192) (RoA, .037) (W. Europe, .421)};
		\addplot[fill = RYB12, draw = RYB12!50!black] coordinates
		{(EMEA,.294) (ASEAN,.823) (CIS, .196) (E. Europe, .103) (NAFTA, .992) (RoA, .218) (W. Europe, .244)};
		\addplot[fill = RYB13, draw = RYB13!50!black] coordinates
		{(EMEA,.012) (ASEAN, .067) (CIS, .212) (E. Europe, .026) (NAFTA, .027) (RoA, .005) (W. Europe, .128)};
		\addplot[fill = RYB14, draw = RYB14!50!black] coordinates
		{(EMEA,.172) (ASEAN,.330) (CIS, .686) (E. Europe, .196) (NAFTA, .383) (RoA, .209) (W. Europe, .501)};
		\addplot[fill = RYB15, draw = RYB15!50!black] coordinates
		{(EMEA,.129) (ASEAN, .932) (CIS, 2.671) (E. Europe, .281) (NAFTA, 1.337) (RoA, .125) (W. Europe, .582)};
		\addplot[fill = RYB16, draw = RYB16!50!black] coordinates
		{(EMEA,0) (ASEAN,0.049) (CIS, 0) (E. Europe, 0) (NAFTA, 0) (RoA, 0) (W. Europe, 1.006)};
		\addplot[fill = RYB17, draw = RYB17!50!black] coordinates
		{(EMEA,0) (ASEAN, 1.638) (CIS, .893) (E. Europe, .668) (NAFTA, .283) (RoA, .463) (W. Europe, 3.342)};
		\addplot[fill = RYB18, draw = RYB18!50!black] coordinates
		{(EMEA,0) (ASEAN,0) (CIS, 0) (E. Europe, 0) (NAFTA, 0) (RoA, 0) (W. Europe, 0)};
		\addplot[fill = RYB19, draw = RYB19!50!black] coordinates
		{(EMEA,.033) (ASEAN,.512) (CIS, .024) (E. Europe, .145) (NAFTA, .054) (RoA, .004) (W. Europe, .556)};
		\addplot[fill = RYB20, draw = RYB20!50!black] coordinates
			{(EMEA,.054) (ASEAN,.152) (CIS, .082) (E. Europe, .392) (NAFTA, .563) (RoA, .043) (W. Europe, 1.309)};
		\addplot[fill = RYB21, draw = RYB21!50!black] coordinates
		{(EMEA,.257) (ASEAN,1.638) (CIS, .052) (E. Europe, .215) (NAFTA, 1.131) (RoA, .607) (W. Europe, .661)};
		\addplot[fill = RYB22, draw = RYB22!50!black] coordinates
		{(EMEA,.142) (ASEAN, .094) (CIS, 0) (E. Europe, 0) (NAFTA, .085) (RoA, .151) (W. Europe, .104)};
		\legend{E-Locos, D-Locos, Shunters, Coaches, Wagons, IC EMU, Regional EMU, IC DMU, Regional DMU, LRV, Metro, APM};
		\end{axis}
		\end{tikzpicture}
        		\end{center}

}



\frame{\frametitle{Vertrieb und Beschaffung}
\framesubtitle{}
\begin{columns}[t] 
     \begin{column}[T]{5cm} 
     	\begin{itemize}
     		\item Typisch: B-to-B-Markt
		\item Eigenschaften:
		\begin{itemize}
		\item Investition statt Konsum
		\item Abgeleitete Nachfrage
		\item Multipersonalit\"at
		\item Formalisierte Beschaffung
		\item Individualisierung
		\item Internationalit\"at
		\end{itemize}
     	\end{itemize}
     \end{column}
     	\begin{column}[T]{7cm} 
         	\begin{center}
            		\includegraphics[width=\textwidth]{Oligopol}
        		\end{center}
     \end{column}
 \end{columns}
}

\frame{\frametitle{Bechaffungsmanagement}
\framesubtitle{}
%\begin{columns}[t] 
    % \begin{column}[T]{8cm} 
     	\begin{itemize}
     		\item Ziele:
		\begin{itemize}
		\item Kosten 
		\item Qualit\"at
		\item Risiko
		\item Flexibilit\"at 		
		\end{itemize}
		\item Strategien:
		\begin{itemize}
		\item Mutiple Sourcing
		\begin{itemize}
		\item[+] Wettbewerb, Risiken minimieren
		\item[-] Aufwand z.B. bei Qualit\"atsunterschienden 
		\end{itemize}
		\item Single Sourcing
		\begin{itemize}
		\item[+] enge Zusammenarbeit, Entwicklung
		\item[-] Wettbewerb eingeschr\"ankt 
		\end{itemize}
		\item Dual Sourcing
		\begin{itemize}
		\item[+/-] Vereint Vor- und Nachteile
		\end{itemize}
		\end{itemize}
     	\end{itemize}
     }
     
\frame{\frametitle{Konzepte der Beschaffung}
\framesubtitle{}
\begin{itemize}
\item Komplexit\"at und Umfang
	\begin{itemize}
	\item System / Module Sourcing
	\item Component Sourcing
	\item Parts Sourcing
	\end{itemize}
\item Ort der Beschaffung
	\begin{itemize}
	\item Lokal oder global
	\item Intern oder extern (Make or Buy)
	\end{itemize}
\item Bereitstellung
	\begin{itemize}
	\item Stock Sourcing
	\item Demand Tailored Sourcing
	\item Just-In-Time-Sourcing
	\end{itemize}
\end{itemize}
}

\frame{\frametitle{Marktschranken}
\framesubtitle{In der Bahnindustrie sind im Vergleich zu anderen Branchen die Marktschranken hoch.}
     	\begin{itemize}
     		\item Markteintrittsschranke: erschwert den Eintritt neuer Marktteilnehmer
		\item In Bahnm\"arkten h\"aufig:
		\begin{itemize}
		\item Regulatorische Schranken
		\item Normative Schranken
		\item K\"auferpr\"aferenzen
		\end{itemize}
		\item Marktaustrittsschranke: erschwert den Austritt der Marktteilnehmer
		\item Typisch f\"ur M\"arkte mit hohen Schranken: Hohe Margen, Oligopole
     	\end{itemize}
    }
    
\offslide{Sammlung Marktschranken}

\frame{\frametitle{Marktrisiken}
\framesubtitle{}
\begin{columns}[t] 
     \begin{column}[T]{6cm} 
     	\begin{itemize}
     		\item Marktspezifische Risiken:
		\begin{itemize}
		\item Implizite Anforderungen
		\item Marktverdr\"angung durch Wettbewerber
		\end{itemize} 
     	\end{itemize}
     \end{column}
     	\begin{column}[T]{6cm} 
         	\begin{center}
            		%\includegraphics[width=0.8\textwidth]{}
        		\end{center}
     \end{column}
 \end{columns}
}

%\offslide{Sammlung von Marktrisiken}
%
%\frame{\frametitle{Diversifikation: Analyse mittels Ansoff-Matrix}
%\framesubtitle{}
%\begin{columns}[t] 
%     \begin{column}[T]{6cm} 
%     	\begin{itemize}
%     		\item Strategische Marktziele
%		\begin{itemize}
%		\item z.B. Markteintritt in Australien
%		\end{itemize}
%		\item ben\"otigen Produkte
%		\begin{itemize}
%		\item z.B. Kupplung AAR Type H
%		\end{itemize}
%		\item Sowohl der Markteintritt als auch das neue Produkt bergen Risiken!
%     	\end{itemize}
%     \end{column}
%     	\begin{column}[T]{6cm} 
%         	\begin{center}
%            		\includegraphics[width=\textwidth]{Ansoff}
%        		\end{center}
%     \end{column}
% \end{columns}
%}


%Projektmanagement
% !TEX root = ../HuV-Folien.tex
\section{Projektablauf, -kriterien und organisation}
%\sectionpage

\subsection{Projektablauf}
\frame{\subsectionpage}

\offslide{Tafelbild Projektablauf, Ausblick V-Modell, Phasen, Meilensteine}

\subsection{Projektkriterien}
\frame{\subsectionpage}

\frame{\frametitle{Fragen an die Projektorganisation}
\framesubtitle{Eine gute Vorbereitung der Organisation und der Infrastruktur macht erfolgreiche Projektarbeit m\"oglich.}
\begin{itemize}
\item Was verstehen wir unter einem Projekt?
\item Wie binden wir Projekte in unser Unternehmen ein?
\item Welche standardisierten Vorgehensmodelle wenden wir an?
\item Wie stellen wir sicher, dass alle Informationen zur richtigen Zeit verf\"ugbar sind?
\item Welche Dokumente/Dokumentenarten werden eingesetzt? Wie werden sie verwaltet?
\item Gibt es Verhaltensregeln f\"ur das Projektteam?
\item Wie sichern wir die Qualit\"at der Projektbearbeitung?
\end{itemize}
}

\frame{\frametitle{Projektmerkmale}
\framesubtitle{Notwendige Projektmerkmale nach \citep{felkai}}
\begin{itemize}
\item Zeitliche Befristung
\item Eindeutige Zielsetzung
\item Eindeutige Zuordnung der Verantwortungsbereiche
\item Einmaliger (azyklischer) Ablauf/Einmaligkeit
\item Vorgegebener finanzieller Rahmen und begrenzte Ressourcen
\item Komplexit\"at
\item Interdisziplin\"arer Charakter der Aufgabenstellung
\item Relative Neuartigkeit
\item Projektspezifische Organisation
\item Arbeitsteilung
\item Unsicherheit und Risiko
\end{itemize}
}

\subsection{Projektorganisation}
\frame{\subsectionpage}

%\offslide{Formen der Aufbauorganisation}

\offslide{Formen der Projektorganisation}





%Projektdokumentation, Vertragspr\"ufung
% !TEX root = SFV-15014_HuV.tex
\section{Projektdokumentation}
%\sectionpage

\frame{\frametitle{Dokumentationssystem}
\framesubtitle{Im Bereich Bahn bestehen zum Teil lange Aufbewahrungspflichten und noch l\"angere Aufbewahrungsinteressen, z.B. f\"ur Ersatzteile, Prozesse, etc.}
\begin{itemize}
\item Identifizieren erwarteter Dokumente
	\begin{itemize}
		\item z.B. Vertragsdokumente, Kalkulationen, Berichte, Entwicklungs- und Testdokumentation
	\end{itemize}
\item Kennzeichnungssystem
	\begin{itemize}
		\item Eindeutigkeit, Aktualit\"at, Relevanz des Dokuments
	\end{itemize}
\item Anforderungen an Dokumente
	\begin{itemize}
		\item Formale Anforderungen: Name und Status des Dokuments, Projekt, Ersteller, Pr\"ufer, Freigeber, Verteiler, Integrit\"at (z.B. durch Seitenzahlen)
		\item Inhaltliche Anforderungen
	\end{itemize}
\item Verantwortlichkeiten
	\begin{itemize}
		\item z.B. anhand einer Dokumentenmatrix
	\end{itemize}
\item Ablagestruktur
	\begin{itemize}
		\item z.B. gemeinsamer Netzwerkordner
	\end{itemize}
\item Datensicherung
\end{itemize}
}


\frame{\frametitle{Warum Projektdokumentation?}
\framesubtitle{}
\begin{itemize}
\item Multipersonalit\"at
\item Langer Projektlebenszyklus
\item Rechtliche Auswirkungen
	\begin{itemize}
		\item Strafrechtlich (Sorgfaltspflicht!)
		\item Zivilrechtlich 
		\begin{itemize}
		\item Nachforderungen
		\item Nichterf\"ullung von Anforderungen
		\end{itemize}
	\end{itemize}
\item Dokumentationspflichten
	\begin{itemize}
		\item Kundenforderung
		\item Normative Anforderungen (z.B. ISO 9001, IRIS)
	\end{itemize}
\item Wiederverwendbarkeit der Entwicklung
\item Nachvollziehbarkeit von Entscheidungen, Kalkulationen, etc.
\item Zulassung	
\end{itemize}
}

\frame{\frametitle{Welche Arten von Dokumenten?}
\begin{itemize}
 \item<1-> Projektmanagement-Dokumente
\begin{itemize}
	\item<2-> Projektauftrag, Aufgabenlisten, -zuordnungen, Terminpl\"ane, Dokumentenmatrix, Statusberichte, Budget-Reporting, Lieferstaffeln, Gespr\"achsprotokolle, Re...
	%\item Steuerung des Projekts
\end{itemize}
\item<1-> Technische Dokumente
\begin{itemize}
	\item<2-> Anforderungs-Dokumente, St\"ucklisten, Zeichnungen, Nachweise (z.B. Berechnungen), Berichte, Abweichungs- und \"Anderungsmitteilungen, ...
\end{itemize}
\item<1-> Betriebswirtschaftliche Dokumente
\begin{itemize}
	\item<2-> Kalkulationen, Preiseskalation, Angebote von Zulieferern und Dienstleistern, Verhandlungsprotokolle, 
\end{itemize}
\item<1-> Dokumente des Qualit\"atsmanagements
\begin{itemize}
	\item<2-> Pr\"ufanweisungen, Ergebnisse, Zeugnisse, Lieferantenaudits
\end{itemize}
\item<1-> Rechtliche Dokumente
\begin{itemize}
	\item<2-> Vertrag, Rahmenvertrag, 
\end{itemize}
\end{itemize}
}

\frame{\frametitle{Dokumentenlenkung}
\framesubtitle{Lenkung von Informationen gem\"a{\ss} ISO 9001}
\begin{itemize}
\item Erstellung und Aktualisierung
\begin{itemize}
	\item Angemessene Kennzeichnung und Beschreibung
	\begin{itemize}
		\item z.B. Titel, Datum, Autor, Referenznummer
	\end{itemize}
	\item angemessenes Format und Medium
	\begin{itemize}
		\item z.B. Sprache, Softwareversion, Grafiken
	\end{itemize}
	\item Angemessene \"Uberpr\"ufung und Genehmigung im Hinblick auf Eignung und Angemessenheit
\end{itemize}
\item Lenkung der Informationen
\begin{itemize}
	\item Informationen sind verf\"ugbar und geeignet
	\item Informationen sind angemessen gesch\"utzt
\end{itemize}
\item Besondere Aufgaben der Dokumentenlenkung
\begin{itemize}
	\item Verteilung, Zugriff, Auffindung und Verwendung
	\item Ablage/Speicherung und Erhaltung (einschlie{\ss}lich Lesbarkeit)
	\item \"Uberwachung von \"Anderungen (z.B. Versionskontrolle)
	\item Aufbewahrung und Verf\"ugung \"uber den weiteren Verbleib
\end{itemize}
\end{itemize}
}


%\offslide{Informationsgehalt wichtiger Dokumente}{Generell, Aktionsliste, Besprechungsprotokoll, Dokumentenmatrix}
%
%\offslide{Erstellung Projektauftrag}

\section{Vertragspr\"ufung}
%\sectionpage

%\offslide{Warum Vertragspr\"ufung?}{Teilweise normativ vorgeschrieben, z.B. IRIS, EN 15085,...}

\frame{\frametitle{Aufgaben der Vertragspr\"ufung}
\framesubtitle{}
\begin{itemize}
\item Angebotsphase
\begin{itemize}
	\item Pr\"ufung auf Vollst\"andigkeit
	\item Pr\"ufung auf Risiken
\end{itemize}
\item Vertragsabschlussphase
\begin{itemize}
	\item Pr\"ufung auf Vollst\"andigkeit
	\item Pr\"ufung auf Unstimmigkeit
	\item Pr\"ufung auf Widerspr\"uchlichkeit
\end{itemize}
\item Abwicklungsphase
\begin{itemize}
	\item Verfolgen von \"Anderungen
	\item Verfolgen von Abweichungen
\end{itemize}
\end{itemize}
}

\frame{\frametitle{Vorgehen in der Angebotsphase}
\framesubtitle{}
\begin{itemize}
\item Rechte und Pflichten der Vertragsparteien
\begin{itemize}
	\item Dokumentenhierarchie
	\item Liefer- und Leistungsumfang
\end{itemize}
\item Mitwirkungspflichten
\begin{itemize}
	\item Auftraggeber
	\item Auftragnehmer
\end{itemize}
\item Analyse der Regelungen u.a. zu
\begin{itemize}
	\item Vertragsstrafen (z.B. Gewichtsp\"onale, Lieferverzug,...)
	\item Abnahmen
	\item \"Anderungen
	\item Verz\"ogerungen
\end{itemize}
\item Beurteilen besonderer vertraglicher Risiken
\end{itemize}
\begin{enumerate}
\item Lesen der Dokumente
\item Herausforderungen erkennen
\item Ma{\ss}nahmen erarbeiten und umsetzen
\end{enumerate}
}

\frame[allowframebreaks]{\frametitle{Wichtige Aspekte bei der Vertragspr\"ufung}
\framesubtitle{}
\begin{itemize}
\item Anwendbares Recht, Gerichtsstand
\item Regelung von Folgesch\"aden
\item Verzeichnis der Vertragsdokumente (inkl. Ausgabestand)
\item Liefer- und Leistungsumfang
\item Preisstellung (DDP Oslo vs. EXW), Preiseskalation
\item Umgang mit Abweichungen, technischem Fortschritt
\item Technische Termine
\item Optionen
\item Teillieferungen
\item Versp\"atung bei Lieferung, Dokumentation, IBS und P\"onalen
\item Nichteinhalten der vertraglichen Leistungswerte (Qualit\"at, RAMS, LCC,...)
\item Force-Majeur-Klausel
\item Produktionsstandorte
\item Logistik, Verpackung und Konservierung
\item Pr\"ufungen und Tests
\item Schulungen (Kunde und Betreiber)
\item Zertifikate
\item Gew\"ahrleistung
\end{itemize}
}


%Requirements Engineering
% !TEX root = SFV-15014_HuV.tex
\section{Requirements Engineering}
%\sectionpage

\frame{\frametitle{Warum Requirements Engineering (RE)?}
\framesubtitle{Requirements Engineering befasst sich mit dem systematischen Erfassen, Umsetzen und Pr\"ufen von Anforderungen im Entwicklungsprozess.}
\begin{itemize}
\item Qualit\"at: Qualit\"at ist das Ma{\ss} der Erf\"ullung der Anforderungen an ein Produkt.
\item Kosten- und Termintreue
\item Einbindung der Stakeholder (Anspruchsteller)
\item Systematisierung der Beschaffung und des Engineerings
 
\end{itemize}
}

\frame{\frametitle{Key-Aspects of Requirements Engineering}
\framesubtitle{}
\begin{itemize}
\item Stakeholder Involvement
\item Technical Reviews
\item Traceability
\end{itemize}
}

%\offslide{Generisches Phasenmodell}{Modell einer beliebigen Phase eines Entwicklungsprozesses}

\frame{\frametitle{Generisches Phasenmodell}
\framesubtitle{Modell einer beliebigen Phase eines Entwicklungsprozesses}
\begin{columns}[t] 
     \begin{column}[T]{5cm} 
     \textbf{F\"ur jede Phase festzulegen:}
     	\begin{itemize}
     		\item Purpose
		\item Inputs
		\item Entry Criteria
		\item Roles
		\item Verification steps
		\item Outputs
		\item Exit criteria
		\item Resources
		\item Management review activities
     	\end{itemize}
     \end{column}
     	\begin{column}[T]{7cm} 
         	\begin{center}
            		\includegraphics[width=1.0\textwidth]{Phase.png}
        		\end{center}
     \end{column}
 \end{columns}
}

\offslide{V-Modell f\"ur Requirements Engineering}

\frame{\frametitle{Requirements Analysis}
\framesubtitle{Ermitteln der System Level Requirements}
\begin{columns}[t] 
     \begin{column}[T]{6cm} 
     \textbf{Leitfragen:}
     	\begin{itemize}
		\item What are the stakeholders?
     		\item What is the system to do?
		\item How well it is to do it?
		\item Under what conditions?
     	\end{itemize}
	\textbf{Typischer Meilenstein: Initial Design Review (IDR)}
     \end{column}
     	\begin{column}[T]{8cm} 
         	\begin{center}
            		\includegraphics[width=0.95\textwidth]{Phase}
        		\end{center}
     \end{column}
 \end{columns}
}

%\offslide{Erg\"anzen des generischen Phasenmodells}{}

\frame{\frametitle{System Specification}
\framesubtitle{Top Level Design: Architektur, L\"osungen, Subsysteme}
\begin{columns}[t] 
     \begin{column}[T]{6cm} 
     \textbf{Leitfragen:}
     	\begin{itemize}
		\item Is the required system feasible?
     		\item What are system and subsystem borders?
		\item What are associated costs/lead times/risks?
		\item How can the risk be reduced?
		\item Which system integration steps are necessary?
     	\end{itemize}
	\textbf{Typischer Meilenstein: Preliminary Design Review (PDR)}
     \end{column}
     	\begin{column}[T]{8cm} 
         	\begin{center}
            		\includegraphics[width=0.95\textwidth]{Phase}
        		\end{center}
     \end{column}
 \end{columns}
}

%\offslide{Erg\"anzen des generischen Phasenmodells}%{Durchf\"uhrung in der \"Ubung}

\frame{\frametitle{Subsystem Design}
\framesubtitle{Lower Level Design: Architektur, L\"osungen, Module}
\begin{columns}[t] 
     \begin{column}[T]{6cm} 
     \textbf{Leitfragen:}
     	\begin{itemize}
		\item What are the subsystem requirements?
		\item Make or Buy?
		\item Which deliverables (e.g. documentation) are requested?
		\item What is the suitable subsystem structure?
     	\end{itemize}
	\textbf{Typischer Meilenstein: Critical Design Review (CDR)}
     \end{column}
     	\begin{column}[T]{8cm} 
         	\begin{center}
            		\includegraphics[width=0.95\textwidth]{Phase}
        		\end{center}
     \end{column}
 \end{columns}
}

%\offslide{Erg\"anzen des generischen Phasenmodells}%{Durchf\"uhrung in der \"Ubung}

\frame{\frametitle{Module Design}
\framesubtitle{``Build to Specifications'': Zeichnungen, Schemata,...}
\begin{columns}[t] 
     \begin{column}[T]{6cm} 
     \textbf{Leitfragen:}
     	\begin{itemize}
		\item How can the module be realised efficiently?
		\item What are critical characteristics of the module and its parts?
		\item Can service proven modules be used or adapted?
     	\end{itemize}
     \end{column}
     	\begin{column}[T]{8cm} 
         	\begin{center}
            		\includegraphics[width=0.95\textwidth]{Phase}
        		\end{center}
     \end{column}
 \end{columns}
}

%\offslide{Erg\"anzen des generischen Phasenmodells}

%ISO15288


%Aufwandssch\"atzung
% !TEX root = SFV-15014_HuV.tex
\section{Kosten- und Aufwandssch\"atzung}
%%\sectionpage

\frame{\frametitle{Warum Kosten- und Aufwandssch\"atzung?}
\framesubtitle{Kosten- und Aufwandssch\"atzung ist die Grundlage f\"ur erfolgreiche Projektbearbeitung.}
\begin{itemize}
\item Aufwandssch\"atzung (Gr\"o{\ss}e: Zeit)
\begin{itemize}
	\item Identifikation von Arbeitspaketen
	\item Input f\"ur Kostensch\"atzung
	\item Ressourcenplanung und -allokation
	\item Terminplanung (auch projekt\"ubergreifend)
\end{itemize}
\item Kostensch\"atzung (Gr\"o{\ss}e: Geld)
\begin{itemize}
	\item Bestimmung von:
	\begin{itemize}
		\item Einmalkosten \textit{non recurring cost (NRC)}
		\item St\"uckkosten \textit{recurring cost (RC)}
	\end{itemize}
	\item Identifikation von Investitionen
\end{itemize}
\item Entscheidungshilfe im Entwicklungsprozess
\item Bestimmung des Angebotspreises
\end{itemize}
}

\frame{\frametitle{Herausforderungen Aufwands- und Kostensch\"atzung}
\framesubtitle{}
\begin{itemize}
\item Informationen:
\begin{itemize}
	\item unvollst\"andig
	\item unsicher
	\item fehlerbehaftet 
	\item Daher: Sch\"atzung, d.h. wahrscheinlichste Vorhersage \"uber den wahren Aufwand %{\color{red!80!black} Was wird gesch\"atzt?}
\end{itemize}
\item Projektdefinition:
\begin{itemize}
	\item Anforderungen nicht final (``to be defined during design stage'')
	\item \"Anderungen m\"oglich
\end{itemize}
\item Projektablauf:
\begin{itemize}
	\item Beginn durch Angebotsrunden verz\"ogert
	\item Projektverlauf durch externe Einfl\"usse (teil-)gesteuert 
\end{itemize}
\item Projektressourcen:
\begin{itemize}
	\item Durch andere Projekte Ressourcen blockiert oder eingeschr\"ankt nutzbar 
\end{itemize}
\end{itemize}
}

\subsection{Aufwandssch\"atzung}
\frame{\subsectionpage}

\frame{\frametitle{Ans\"atze zur Aufwandssch\"atzung}
\framesubtitle{}
\begin{itemize}
\item Expertensch\"atzverfahren, z.B.:
\begin{itemize}
	\item Projektstrukturplan-basiert \textit{(WBS-based)}
	\item Gruppensch\"atzung
\end{itemize}
\item Formale Sch\"atzverfahren, z.B.:
\begin{itemize}
	\item Analogie-basiert (z.B. Bremszange wie ..., jedoch mit ...)
	\item Parametrische Modelle (z.B. E-Kupplung Verkabelung: 100 h)
	\item Gr\"o{\ss}enbasiert: (z.B. Anpasskonstruktion: 500 h)
\end{itemize}
\item Kombinierte Sch\"atzverfahren, z.B.:
\begin{itemize}
	\item Zerlegung mit WBS, parametrische Sch\"atzung der Pakete
\end{itemize}
\item Auswahl des Verfahrens:
\begin{itemize}
	\item Abh\"angig von der Organisation
	\item Formale Verfahren weniger ``lernf\"ahig''
	\item Expertensch\"atzverfahren anf\"allig f\"ur ``wishful thinking''
\end{itemize}
\item Psychologische Herausforderungen \textit{(Cognitive biases)}:
\begin{itemize}
	\item \textit{Planning fallacy, cognitive dissonance, anchoring, confirmation bias, wishful thinking}
\end{itemize}
\end{itemize}
}


\frame{\frametitle{Projektstrukturplan}
\framesubtitle{Work Breakdown Structure (WBS) f\"ur die Ermittlung von Arbeitspaketen}
\begin{columns}[t] 
     \begin{column}[T]{6cm} 
     	\begin{itemize}
		\item Dekomposition eines Projekts
		\begin{itemize}
		\item Hierarchisch
		\item Inkrementell
		\end{itemize}
		\item Baumstruktur
		\item Gliederung gem\"a{\ss} DIN 69900
		\begin{itemize}
		\item Funktionsorientiert
		\item \textbf{Objektorientiert}
		\item Zeitorientiert
		\end{itemize}
		\item Starke Abh\"angigkeit von Deliverables
		\item Erstellung \"ublicherweise Top-Down
		\item Nutzen: Vollst\"andige \"Ubersicht
		\item Hilfreich: ``Tickler list'' 
     	\end{itemize}
     \end{column}
     	\begin{column}[T]{6cm} 
         	\begin{center}
            		\includegraphics[width=1\textwidth]{WBS}
        		\end{center}
     \end{column}
 \end{columns}
}

%\offslide{WBS f\"ur Beispielprojekt}

\frame{\frametitle{Aufbereitung WBS f\"ur Projektplanung}
\framesubtitle{Die Identifikation der Arbeitspakete allein l\"asst keine Planung des Projekts zu.}
\begin{itemize}
\item Sch\"atzung des Aufwands
\begin{itemize}
	\item Besprechung: m\"oglicher Bias
	\item Alternative: Planning Poker
\end{itemize}
\item Abh\"angigkeit (Reihenfolge) der Projektbearbeitung
\item Externe Inputs oder Vorbedingungen f\"ur Arbeitspakete
\item Vertraglich zugesicherte Termine
\item Zuordnung zu:
\begin{itemize}
	\item Ressourcen
	\item Phasen
\end{itemize}
\item Zieldefinition (Definition of Done)
\end{itemize}
}

\subsection{Kostensch\"atzung}
\frame{\subsectionpage}

\frame{\frametitle{Kostensch\"atzung}
\framesubtitle{}
\begin{columns}[t] 
     \begin{column}[T]{6cm} 
     \textbf{NRC estimation}
     	\begin{itemize}
     		\item Basierend auf Aufwandssch\"atzung
		\item Erg\"anzend:
		\begin{itemize}
		\item Kundenbetreuung
		\item Reisekosten
		\item Externe Dienstleistungen (z.B. Tests, Abnahmen, ...)
		\item Prototypen, Muster
		\item Investitionen
		\end{itemize}
		\item Zu beachten:
		\begin{itemize}
		\item Stundens\"atze
		\item Kostenentwicklung
		\end{itemize}
		\item N\"utzlich: Checkliste
     	\end{itemize}
     \end{column}
     	\begin{column}[T]{6cm} 
	\textbf{RC estimation} (\cite{niazi05, pahlbeitz})
         \begin{itemize}
         	\item Intuitive Verfahren:
		\begin{itemize}
		\item Basierend auf Expertensch\"atzung
		\item Unterst\"utzt durch Regeln
		\end{itemize} 
         	\item Analogiebasierte Sch\"atzung
		\begin{itemize}
		\item \"Ahnlichkeit
		\item Komplexit\"at 
		\end{itemize} 
     		\item Parametrische Sch\"atzung:
		\begin{itemize}
		\item z.B. Gewicht, Material oder kombiniert %, z.B. \tiny
%		\begin{equation*} 
%		C = FC+\left(C_{co} N_{co} +  \frac{C_{rm}TF}{1-SC} \right)W
%		\end{equation*}
		\end{itemize}
		\item Analytische Verfahren, z.B.
		\begin{itemize}
		\item Bearbeitungssimulation
		\item Feature based cost estimation
		\end{itemize}
     	\end{itemize}
     \end{column}
 \end{columns}
}

%\frame{\frametitle{Kostenrechnung}
%\framesubtitle{Grundbegriffe}
%\begin{itemize}
%\item Fixe Kosten:
%\begin{itemize}
%	\item Produktionsunabh\"angig anfallende Kosten 
%\end{itemize}
%\item Variable Kosten:
%\begin{itemize}
%	\item Produktionsabh\"angig anfallende Kosten 
%\end{itemize}
%\item Gemeinkosten:
%\begin{itemize}
%	\item Kosten, die keinem Kostentr\"ager zugeordnet werden k\"onnen 
%\end{itemize}
%\item Einzelkosten:
%\begin{itemize}
%	\item Kosten, die direkt einem Kostentr\"ager zugeordnet werden k\"onnen 
%\end{itemize}
%\item Deckungsbeitrag:
%\begin{itemize}
%	\item DB I: Umsatz - variable Kosten
%	\item DB II DB I - fixe Kosten 
%\end{itemize}
%\end{itemize}
%}




%Projektplanung
% !TEX root = SFV-15014_HuV.tex
\section{Projektplanung}
%\sectionpage

\frame{\frametitle{Was bedeutet Projektplanung?}
\framesubtitle{\textit{[...] I have always found that plans are useless but planning is indispensable.}  \\ \hspace{9cm} {Dwight D. Eisenhower}}
\begin{columns}[t] 
     \begin{column}[T]{6cm} 
     	\begin{itemize}
     		\item Organisation verschiedener Projektaspekte:
		\begin{itemize}
		\item Projektumfang
		\item Arbeitspakete
		\item Projektrisiken
		\item Finanz- und Kostenplanung
		\item Einsatzmittelplanung
		\item Materialplanung
		\item Berichten des Fortschritts
		\item Verfolgen von Abweichungen
		\end{itemize}
		\item Begleitung der Projektdurchf\"uhrung
		\begin{itemize}
		\item Ggf. Plananpassung, Krisenmanagement
		\end{itemize}
     	\end{itemize}
     \end{column}
     	\begin{column}[T]{6cm} 
         	\begin{center}
            		\includegraphics[width=0.9\textwidth]{NasaPP}\\
		\tiny {\color{gray} NASA Project Planning Process}
        		\end{center}
     \end{column}
 \end{columns}
}

\frame{\frametitle{Projekt: Mars Voyage}
\framesubtitle{Zur Veranschaulichung der Projektplanungsmethodiken dient das recht engagierte Ziel, innerhalb von 10 Monaten Astronauten zum Mars zu senden.}
\begin{center}
\begin{tabular}{|c|c|c|c|}
\hline
Step & Description & Estimated Effort & Predecessor\\ \hline
0 & Kick Off & Milestone & -\\ \hline
A & Build spaceship & 2 & 0\\ \hline
B & Equip spaceship & 1 & A\\ \hline
C & Test spaceship & 2 & B\\ \hline
D & Train astronauts & 4 & 0\\ \hline
E & Flight to Mars & 5 & C, D\\ \hline
1 & Mars Landing & Milestone & E\\ \hline
\end{tabular}
\end{center}
}

%\frame{\frametitle{M\"ogliche Sequenzen von Arbeitspaketen}
%\framesubtitle{}
%\begin{itemize}
%\item Abh\"angig
%\item Unabh\"angig
%\item Ineinandergreifend
%\end{itemize}
%}


\frame{\frametitle{Kritischer Pfad}
\framesubtitle{Der kritische Pfad zeigt unter parallelen Aktivit\"aten die (derzeit) terminbestimmende Kombination an.}
     	\begin{itemize}
     		\item Voraussetzungen:
		\begin{itemize}
		\item Liste der Arbeitspakete (z.B. aus WBS)
		\item Dauer der Aufgaben
		\item Abh\"angigkeiten
		\item Zwischen-/Endpunkte, z.B. Meilensteine, Deliverables
		\end{itemize}
     	\end{itemize}
             	\begin{center}
		\begin{tikzpicture}[->,>=stealth',shorten >=1pt,auto,node distance=2cm,
  thick,main node/.style={circle,fill=blue!20,draw,font=\sffamily\Large\bfseries}]

  \node[main node, fill = gray] (0) {0};
  \node[main node] (A) [below right of=0] {A:2};
  \node[main node] (B) [right of=A] {B:1};
  \node[main node] (C) [right of=B] {C:2};
  \node[main node] (D) [above of=B] {D:4};
  \node[main node] (E) [above right of=C] {E:5};
  \node[main node, fill = gray] (1) [right of=E] {1};


  \path[every node/.style={font=\sffamily\small}]
    	(0) edge[draw = red!80!black] node {} (A)
	(A) edge[draw = red!80!black] node {} (B)
	(B) edge[draw = red!80!black] node {} (C)
	(0) edge node {} (D)
	(C) edge[draw = red!80!black] node {} (E)
	(D) edge node {} (E)
	(E) edge[draw = red!80!black] node {} (1);
	%(0) edge [bend left] node[above]{10} (1);
\end{tikzpicture}
	        		\end{center}
  }

\frame{\frametitle{Gantt-Diagramme}
\framesubtitle{}
\begin{columns}[t] 
     \begin{column}[T]{6cm} 
     	\begin{itemize}
     		\item Tabellenform:
		\begin{itemize}
		\item Erste Zeile: Zeitachse
		\item Erste Spalte: Aktivit\"aten
		\item Aktivit\"aten als Balken
		\end{itemize}
		\item Weitere Elemente:
		\begin{itemize}
		\item Gruppen
		\item Meilensteine
		\item Abh\"angigkeiten
		\end{itemize}
		\item Un\"ubersichtlich f\"ur gro{\ss}e Projekte
		\item Toolunterst\"utzung:
		\begin{itemize}
		\item MS Project
		\item Project Libre
		\end{itemize}
     	\end{itemize}
     \end{column}
     	\begin{column}[T]{6cm} 
         	\begin{center}
            		\begin{ganttchart}[vgrid, hgrid, 
		%today = 4, 
		progress=today,
		progress label text=\relax,
		today=3,
		y unit chart = 5 mm,
		x unit = 4 mm,
		bar/.append style={fill=blue!80!black},
		%chart element start border=right
		]{1}{12}
			%\gantttitle{Mars voyage}{12} \\
			\gantttitlelist{1,...,12}{1}\\
			\ganttgroup{}{2}{11} \\
			\ganttmilestone{0}{1}\\
			\ganttbar{A}{2}{3} \\
			\ganttbar{B}{4}{4} \\
			\ganttbar{C}{5}{6} \\
			\ganttbar[progress = 90]{D}{2}{5} \\
			\ganttbar{E}{7}{11} \\
			\ganttmilestone{1}{11}
			\ganttlink{elem1}{elem2}
			\ganttlink{elem1}{elem5}
			\ganttlink{elem2}{elem3}
			\ganttlink{elem3}{elem4}
			\ganttlink{elem4}{elem6}
			\ganttlink{elem4}{elem6}
			\ganttlink{elem5}{elem6}
			\end{ganttchart}

        		\end{center}
     \end{column}
 \end{columns}
}

\frame{\frametitle{Projektplanung im Gantt-Diagram}
\framesubtitle{}
         	\begin{center}
            		\begin{ganttchart}[vgrid, hgrid, 
		%today = 4, 
		progress=today,
		progress label text=\relax,
		today=3,
		y unit chart = 7 mm,
		bar/.append style={fill=blue!80!black},
		%chart element start border=right
		]{1}{12}
			%\gantttitle{Mars voyage}{12} \\
			\gantttitlelist{1,...,12}{1}\\
			\ganttgroup{Get there}{2}{11} \\
			\ganttmilestone{0: Kick Off}{1}\\
			\ganttbar{A: Build spaceship}{2}{3} \\
			\ganttbar{B: Equip spaceship}{4}{4} \\
			\ganttbar{C: Test spaceship}{5}{6} \\
			\ganttbar[progress = 90]{D: Train astronauts}{2}{5} \\
			\ganttbar{E: Flight to Mars}{7}{11} \\
			\ganttmilestone{1: Mars landing}{11}
			\ganttlink{elem1}{elem2}
			\ganttlink{elem1}{elem5}
			\ganttlink{elem2}{elem3}
			\ganttlink{elem3}{elem4}
			\ganttlink{elem4}{elem6}
			\ganttlink{elem4}{elem6}
			\ganttlink{elem5}{elem6}
			\end{ganttchart}
        		\end{center}
}

%\frame{\frametitle{Netzplanmethode}
%\framesubtitle{}
%\begin{columns}[t] 
%     \begin{column}[T]{6cm} 
%     	\begin{itemize}
%     		\item Graph, Elemente:
%		\begin{itemize}
%		\item Knoten
%		\item Kanten
%		\end{itemize}
%		\item Darstellungen u.a.:
%		\begin{itemize}
%		\item Vorgangsknoten-Netzplan
%		\item Vorgangspfeil-Netzplan
%		\end{itemize}
%     	\end{itemize}
%     \end{column}
%     	\begin{column}[T]{6cm} 
%         	\begin{itemize}
%		\item Daten eines Vorgangs:
%		\begin{itemize}
%		\item Aus Vorw\"artsplanung:
%		\begin{itemize}
%		\item Fr\"uhester Anfangs-, Endzeitpunkt
%		\end{itemize}
%		\item Aus R\"uckw\"artsplanung:
%		\begin{itemize}
%		\item Sp\"atester Anfangs-, Endzeitpunkt
%		\end{itemize}
%		\item Dauer
%		\item Pufferzeiten
%		\end{itemize}
%		\end{itemize}
%     \end{column}
% \end{columns}
%\begin{center}
%            		\includegraphics[width=0.7\textwidth]{Netzplan}
%			\rotatebox{90}{\tiny \color{gray} Quelle: Wikipedia/Skaler}
%        		\end{center}
%}
%
%\frame{\frametitle{Design Structure Matrix}
%\framesubtitle{}
%\begin{columns}[t] 
%     \begin{column}[T]{6cm} 
%     	\begin{itemize}
%     		\item Modellierung des Informationsflusses
%		\item Erleichtert:
%		\begin{itemize}
%		\item Finden von (sequentiellen) Aufgaben-Clustern
%		\item Entdecken von Interationen
%		\end{itemize}
%		\item Eintr\"age:
%		\begin{itemize}
%		\item Hauptdiagonale: Dauer
%		\item Unterhalb: Sequentielle Arbeitspakete
%		\item Oberhalb: Iterationen
%		\end{itemize}
%		\item Cluster:
%		\begin{itemize}
%		\item Bilden eine Submatrix
%		\item K\"onnen zusammengefasst werden
%		\end{itemize}
%     	\end{itemize}
%     \end{column}
%     	\begin{column}[T]{6cm} 
%         	\begin{center}
%            		\only<1-2>{\begin{tabular}{c|c|c|c|c|c|c|c|}
%                            & 0 & A & B & C & D & E & 1\\ \hline
%                            0 & 0 &  &  &  &  &  & \\ \hline
%                            A &  X &\only<2>{\cellcolor{red!25} \color{red!80!black}} 2 & \only<2>{\cellcolor{red!25}} & \only<2>{\cellcolor{red!25}} &  &  & \\ \hline
%                            B &  & \only<2>{\cellcolor{red!25} \color{red!80!black}}X & \only<2>{\cellcolor{red!25}\color{red!80!black}}1 & \only<2>{\cellcolor{red!25}} &  &  & \\ \hline
%                            C &  &  \only<2>{\cellcolor{red!25}} & \only<2>{ \cellcolor{red!25}\color{red!80!black}}X & \only<2>{\cellcolor{red!25}\color{red!80!black}}2 &  &  & \\ \hline
%                            D & X &  &  &  &\only<2>{\cellcolor{blue!25} \color{blue!80!black}} 4 &  & \\ \hline
%                            E &  &  &  & X & X & 5 & \\ \hline
%                            1 &  &  &  &  & X & X & 0\\ 
%                            \end{tabular}}
%                            \only<3>{\begin{tabular}{c|c|c|c|c|c|}
%                            & 0 & ABC & D & E & 1\\ \hline
%                            0 & 0 &  &  &  & \\ \hline
%                            ABC &  X &\cellcolor{red!25} \color{red!80!black} 5 &  &  & \\ \hline
%                            D & X  &  &\cellcolor{blue!25} \color{blue!80!black} 4 &  & \\ \hline
%                            E &  & X & X & 5 & \\ \hline
%                            1 &  &  & X & X & 0\\ 
%                            \end{tabular}
%                            }
%        		\end{center}
%     \end{column}
% \end{columns}
%}

% Grundlagen Lean Manufacturing
%% !TEX root = SFV-15014_HuV.tex
\section{Einf\"uhrung Lean Manufacturing}
%\sectionpage

\frame{\frametitle{Schlanke Fertigung bei der Sendung mit der Maus}
\framesubtitle{}
\begin{center}
\url{http://www.ardmediathek.de/tv/Die-Sendung-mit-der-Maus/Die-Sendung-mit-der-Maus-25-11-2012-Fl/Das-Erste/Video-Podcast?documentId=12567908&bcastId=1458}
\end{center}
}

\frame{\frametitle{Lean Production bei Porsche}
\framesubtitle{}
\begin{center}
On July 27, 1994, something remarkable happened in the assembly hall of the Porsche company in Stuttgart, Germany. A Porsche Carrera rolled off the line with nothing wrong with it. The army of blue-coated craftsmen waiting in the vast rectification area could pause for a moment because, for the first time in forty-four years, they had nothing to do. This was the first defect-free car ever to roll off a Porsche assembly line or to emerge from the earlier system of bench assembly. This first perfect Porsche—and there have been many more since—was a small but highly visible milestone in the efforts of Chairman Wendelin Wiedeking and his associates to introduce lean thinking into a veritable industrial institution—indeed, into one of the great symbols of the German industrial tradition. [...] What’s more, there’s already evidence that when lean concepts are married to the strengths of the German tradition, embodied in the concept of superior technology, or technik, a remarkably competitive hybrid form can emerge. \citep{womack2010}
\end{center}
}


\frame{\frametitle{Lean als Produktionssystem}
\framesubtitle{}
     	\begin{itemize}
		\item Eingef\"uhrt von japanischen Automobilunternehmen
		\item Starke Fokussierung auf Kundennutzen
     		\item Im Gegensatz zu gepufferter Produktion
		\item Ziele
		\begin{itemize}
		\item Kompetenz und Verantwortung zusammenf\"uhren
		\item in Netzwerken arbeiten
		\item Verschwendung und Fehler vermeiden
		\item Abl\"aufe synchronisieren
		\item Kontinuierlich im Kleinen besser werden
		\item bei Bedarf im Gro{\ss}en \"andern
		\end{itemize}
     	\end{itemize}
   }
   
   \frame{\frametitle{Elemente des Lean Manufacturing}
\framesubtitle{}
\begin{columns}[t] 
     \begin{column}[T]{6cm} 
     	\begin{itemize}
     		\item Angemessene technische Ausstattung
		\item Wenig hierarchische Arbeitsorganisation
		\item Konsequentes Qualit\"atsmanagement
		\item Kontinierliche Verbesserung
		\item Qualifikation und Motivation
		\item Just-In-Time/Sequence, Pull
		\item Wertsch\"opfungsorientierung
     	\end{itemize}
     \end{column}
     	\begin{column}[T]{8cm} 
         	\begin{center}
            		\includegraphics[width=0.8\textwidth]{LeanHouse}\rotatebox{90}{\tiny \color{gray}Quelle: Laurens van Lieshout}
        		\end{center}
     \end{column}
 \end{columns}
}

\frame{\frametitle{Vermeidung von Verschwendung}
\framesubtitle{Sieben Arten der Verschwendung \textit{muda} werden genannt}
\begin{enumerate}
\item Transport: Kein Kundennutzen durch Wege
\item Best\"ande: Binden Kapital, Fl\"ache, erzeugen Handhabungsaufwand
\item Bewegung: Mehr Bewegung als der Prozess ben\"otigt
\item Warten: Wartezeiten erzeugen keinen Kundennutzen
\item \"Uberproduktion: Kein Kunde, kein Nutzen
\item Aufw\"andige Prozesse: Fehleranf\"allig, unflexible Prozesse
\item Fehler: Kein Kundennutzen durch Fehlersuche und -behebung \color{red!80!black}{$6 \sigma$}
\end{enumerate}
}

\frame{\frametitle{Tools auf dem Weg zum Lean Manufacturing}
\framesubtitle{}
\begin{itemize}
\item 5 S: Sortiere aus! Stelle ordentlich hin! S\"aubere! Sauberkeit bewahren! Selbstdisziplin \"uben!
\item One-Piece-Flow: Losgr\"o{\ss}enreduzierung. Ben\"otigt kurze R\"ustzeiten.
\item Visual Management: Zustand des Prozesses, Verbesserungen etc. visualieren.
\item Jidoka: Fehler an der Quelle finden.
\item Poka Yoke: Vermeiden ``ungl\"ucklicher'' Fehler, z.B. durch Kodierung.
\item Heijunka: Nivellierung des Produktionslevels.
\item Kanban: Bedarf steuert Produktion.
\item Andon: Zustand des Prozesses in Echtzeit abbilden.
\item Kaizen: Stetige Verbesserung.
\item Genba: Ort der Produktion.
\item Obeya: ``Gro{\ss}er Raum''.
\item \emph{Genchi Genbutsu: Go and see for yourself!}
\end{itemize}
}

\frame{\frametitle{Lean Development}
\framesubtitle{}
\begin{itemize}
\item Wert: Spezifiziere den Wert deines Produktes
\item Wertstrom: Erkenne den Wertstrom 
\item Flow: Erzeuge einen Wertstromfluss ohne Unterbrechungen
\item Pull: Lasse den Kunden den Takt der Bearbeutung bestimmen
\item Perfektion: Verbessere die Dinge kontinuierlich
\end{itemize}
}

\frame{\frametitle{Agile Development}
\framesubtitle{}
\begin{itemize}
\item Werte (Agile Manifesto):
\begin{itemize}
	\item Menschen und Interaktionen mehr als Prozesse und Werkzeuge
	\item Funktionierende Software [Produkte] mehr als umfassende Dokumentation
	\item Zusammenarbeit mit dem Kunden mehr als Vertragsverhandlung
	\item Reagieren auf Ver\"anderung mehr als Befolgen eines Plans
\end{itemize}
\item Prinzipien:
\begin{itemize}
	\item Kurze Iterationen
	\item Einfachheit
	\item Selbstorganisation
	\item Pers\"onliche Kommunikation
	\item Teamarbeit
\end{itemize}
\end{itemize}
}




%Rohbau
%% !TEX root = SFV-15014_HuV.tex
\section{Wagenkastenrohbau}
\sectionpage

\frame{\frametitle{Anforderungen an den Wagenkasten \textit{car body}}
\framesubtitle{}
\begin{itemize}
\item Festigkeit (EN 12663):
\begin{itemize}
	\item Zug-/Druckkr\"afte im Zugverband
	\item Crash-Szenarien (EN 15227)
	\item Drucks\"o{\ss}e, Druckdichtigkeit
	\item Durchbiegung unter Beladung
	\item Schwingungen
\end{itemize}
\item Kunden-/ betriebliche Anforderungen
\begin{itemize}
	\item Lebensdauer
	\item Reparaturfreundlichkeit, Ersatzteilverf\"ugbarkeit
	\item Geringe Masse
	\item Design
	\item Entsorgung/Recycling
\end{itemize}
\item Normative/gesetzliche Anforderungen
\begin{itemize}
	\item Brandschutz (DIN 5510, EN 45545, ...)
	\item Material (EG 1907/2006 REACH)
	\item Crash und Festigkeit s.o.
\end{itemize}
\item Systemimmanente Anforderungen (Schwingungen, elastische Verformung,...)
\end{itemize}
}


\frame{\frametitle{Konstruktionsprinzipien}
\framesubtitle{}
\begin{columns}[t] 
\begin{column}[T]{.5cm}
\end{column} 
     \begin{column}[T]{5.5cm} 
     \textbf{Differenzialbauweise}
     	\begin{itemize}
     		\item Fertigung aus Halbzeugen:
		\begin{itemize}
		\item Einzelteile einfach geformt
		\item Formgebung durch F\"ugen und Umformen
		\end{itemize}
     	\end{itemize}
	\textbf{Integralbauweise}
     	\begin{itemize}
     		\item Fertigung aus komplex geformten Elementen:
		\begin{itemize}
		\item z.B. Strangpressprofile
		\item Formgebung durch F\"ugen und Zerspanen
		\end{itemize}
     	\end{itemize}
	\textbf{Tragfunktion}
     	\begin{itemize}
     		\item Tragendes Untergestell
		\item Selbsttragender Wagenkasten
	\end{itemize}
     \end{column}
     	\begin{column}[T]{6cm} 
         	\begin{center}
            		\includegraphics[width=0.8\textwidth]{Integral}\rotatebox{90}{\tiny \color{gray} Quelle: Siemens Pressebild}\\
		Wagenkasten in Integralbauweise
        		\end{center}
     \end{column}
 \end{columns}
}

\frame{\frametitle{Leichtbau der Wagenk\"asten}
\framesubtitle{}
\begin{itemize}
\item Alle Elemente an Aufnahme der Beanspruchungen beteiligen
\item Gut (leicht) ertragbar:
\begin{itemize}
	\item Zug- und Druckkr\"afte
\end{itemize}
\item Mit zus\"atzlichem Material ertragbar:
\begin{itemize}
	\item Torsions- und Biegemomente
\end{itemize}
\item H\"oherfeste Materialien werden z\"ogerlich angenommen
\begin{itemize}
	\item Bedenken bei Wartbarkeit und Lebensdauer
\end{itemize}
\end{itemize}
\begin{center}
\includegraphics[width=0.8\textwidth]{Class66}
\end{center}
}

\frame{\frametitle{Werkstoffe f\"ur Wagenk\"asten}
\framesubtitle{}
\begin{itemize}
\item Stahl:
\begin{itemize}
	\item Klassisch eingesetzt: Baust\"ahle S235, S355
	\item Ebenfalls anzutreffen: Edelst\"ahle, z.B. X5CrNi18-10
	\item Gut zu f\"ugen und umzuformen
	\item Dauerfestigkeit und elastisch/plastisches Verhalten gutm\"utig
\end{itemize}
\item Aluminium:
\begin{itemize}
	\item Geringere Dichte, geringerer E-Modul
	\item Dauerfestigkeitsgrenze wenig ausgepr\"agt
	\item Schwei{\ss}n\"ahte wenig erm\"udungsfest
	\item F\"ugeverfahren erfordern getrennte Behandlung von Stahl
\end{itemize}
\item Kunststoffe:
\begin{itemize}
	\item In der Regel faserverst\"arkt (GFK, CFK)	
	\item Erm\"oglichen Integralbauweise und Funktionsintegration
	\item Auch als Sandwichmaterialien
\end{itemize}
\item Waben und Schaummaterialien:
\begin{itemize}
	\item Eingesetzt im Deformationsbereich
\end{itemize}
\end{itemize}
}

\frame{\frametitle{Hauptbaugruppen des Rohbaus}
\framesubtitle{}
\begin{columns}[t] 
     \begin{column}[T]{6cm} 
     	\begin{itemize}
     		\item Untergestell
		\begin{itemize}
		\item (Mittel/Aussen-) Langtr\"ager
		\item Quertr\"ager
		\end{itemize}
		\item Seitenw\"ande
		\begin{itemize}
		\item Druckwechselbelastung
		\end{itemize}
		\item Dach
		\begin{itemize}
		\item Wasserablauf
		\end{itemize}
		\item Endw\"ande
		\begin{itemize}
		\item Schnittstellen, Crash
		\end{itemize}
		\item Kopfmodule
		\begin{itemize}
		\item Vorfertigung, Schnittstellen, Crash
		\end{itemize}
     	\end{itemize}
     \end{column}
     	\begin{column}[T]{6cm} 
         	\begin{center}
            		\includegraphics[width=0.8\textwidth]{FrontNose}
		\rotatebox{90}{\tiny \color{gray}{Quelle: Voith Pressebild}}
        		\end{center}
     \end{column}
 \end{columns}
}

\frame{\frametitle{Prozess Wagenkastenfertigung}
\framesubtitle{Bei allen Schritten zu beachten: Teils extremer Verzug durch W\"armeeinleitung}
\begin{columns}[t] 
     \begin{column}[T]{6cm} 
     	\begin{enumerate}
     		\item Einzelteilfertigung
		\begin{itemize}
		\item Schneiden, Schwei{\ss}nahtvorbereitung, Kanten, etc.
		\end{itemize}
		\item Baugruppenfertigung
		\begin{itemize}
		\item Schwei{\ss}en, evtl. Bearbeitung
		\item Hand- oder Roboterschwei{\ss}en je nach Naht
		\item Vermessung
		\end{itemize}
		\item Wagenkastenaufbau
		\begin{itemize}
		\item Vorsprengung bei statischer Durchbiegung
		\item Dichtigkeitspr\"ufung
		\end{itemize}
		\item Richten
		\item Sandstrahlen
     	\end{enumerate}
     \end{column}
     	\begin{column}[T]{6cm} 
         	\begin{center}
            		\includegraphics[width=0.8\textwidth]{Rohbau}\rotatebox{90}{\tiny \color{gray} Quelle: Siemens Pressebild}\\  
		Pr\"ufen der Aussenkontur     		
		\end{center}
     \end{column}
 \end{columns}
}


%Schweissen nach EN 15085
% !TEX root = SFV-15014_HuV.tex
\section{Schwei{\ss}en nach EN 15085}
%\sectionpage

\frame{\frametitle{Schwei{\ss}en an Schienenfahrzeugen - Allgemeines}
\framesubtitle{}
\begin{definition}[Schwei{\ss}en]
Schwei{\ss}en bezeichnet das unl\"osbare Verbinden von Bauteilen unter Anwendung von W\"arme oder Druck, mit oder ohne Schwei{\ss}zusatzwerkstoffe. 
\end{definition}
\textbf{Anwendung in der Schienenfahrzeugtechnik}
\begin{itemize}
	\item Verbindungen: z.B. Wagenkasten und Drehgestellfertigung
	\item Reparaturschwei{\ss}ungen im Stahlgussprozess
	\item Auftragsschwei{\ss}ungen, z.B. verschlei{\ss}mindernde Schichten
\end{itemize}
\textbf{Herausforderungen in der Schienenfahrzeugtechnik}
\begin{itemize}
	\item Lange Produktlebensdauer, hohe Schwingspielzahlen
	\item Hohes Sicherheitsbed\"urfnis
	\item Zertifizierungs und Pr\"ufaufwand
\end{itemize}
}

\frame{\frametitle{Prozessschritte Herstellung Schwei{\ss}verbindungen}
\framesubtitle{}
\begin{enumerate}
\item Wareneingangspr\"ufung (mind. Zeugnisse)
\item Schwei{\ss}nahtvorbereitung
\begin{itemize}
	\item i.d.R. maschinelle Bearbeitung
	\item verh\"altnism\"a{\ss}ig eng toleriert
	\item je nach Nahtform Badsicherung erforderlich
\end{itemize}
\item Aufnahme in Vorrichtung
\item Heften
\item Entnahme aus Vorrichtung
\item Schwei{\ss}en
\item Pr\"ufung duch Schwei{\ss}er/Bediener
\item Ggf. Nachbearbeitung der Schwei{\ss}naht 
\item Pr\"ufung durch Pr\"ufpersonal bzw. Werkerselbstpr\"ufung
\item Ggf. Abnahme duch Kunden
\item Bearbeitung zur Erreichung von Schnittstellenma{\ss}en
\end{enumerate}
}

\frame{\frametitle{Prozessschritte Konstruktion Schwei{\ss}verbindungen}
\framesubtitle{}
\begin{enumerate}
\item Technische Vertragspr\"ufung
\item Konstruktionsentwurf
\begin{itemize}
	\item Ermittlung Beanspruchungszustand
	\item Ermittlung Sicherheitsbed\"urfnis
\end{itemize}
\item Freigabe durch verantwortliche Schwei{\ss}aufsichtsperson (vSAP)
\begin{itemize}
	\item F\"ur DB: Ggf. Schwei{\ss}technische Bauweisenpr\"ufung Teil 1 (STBP 1) 
\end{itemize}
\item Festlegung ben\"otigter Verfahrens- und Arbeitsproben
\item Durchf\"uhrung und Analyse Verfahrens- und Arbeitsproben (evtl. iterativ)
\item Fertigung der Bauteile
\begin{itemize}
	\item F\"ur DB: Ggf. Schwei{\ss}technische Bauweisenpr\"ufung Teil 2 (STBP 2) 
\end{itemize}
\end{enumerate}
}



\frame{\frametitle{Europ\"aische Normierung nach EN 15085}
\framesubtitle{Bahnanwendungen - Schwei{\ss}en von Schienenfahrzeugen und -fahrzeugteilen}
\textbf{Teile der EN 15085}
\begin{enumerate}
	\item Allgemeines, Begriffe
	\item Qualit\"atsanforderungen und Zertifizierung von Schwei{\ss}betrieben
	\item Konstruktionsvorgaben
	\item Fertigungsanforderungen
	\item Pr\"ufung und Dokumentation
\end{enumerate}
\textbf{Anwendungsbereich}
\begin{itemize}
	\item Schwei{\ss}en metallischer Werkstoffe 
	\begin{itemize}
		\item Pflicht f\"ur Stahl und Aluminium, auch Gusslegierungen)
		\item Fakultativ f\"ur andere Werkstoffe
		\end{itemize}
	\item Herstellung und Instandsetzung
	\item Ausnahme: spezielle Regelwerke, z.B. Druckbeh\"alter
\end{itemize}
}

\frame[allowframebreaks]{\frametitle{Begriffe}
\framesubtitle{In Anlehnung an \cite{en150851}}
\begin{definition}[Zertifizierungsstufe (CL)]
Stufe zur Klassifizierung der geschwei{\ss}ten Schienenfahrzeuge und geschwei{\ss}ter Komponenten on Abh\"angigkeit von der Schwei{\ss}nahtg\"uteklasse.
\end{definition}
\begin{definition}[Schwei{\ss}nahtg\"uteklasse (CP)]
G\"uteanforderungen an die Schwei{\ss}verbindung in Abh\"angigkeit von Beanspruchungszustand und von Sicherheitsbed\"urfnis der einzelnen Schwei{\ss}naht.
\end{definition}
\begin{definition}[Schwei{\ss}nahtpr\"ufklasse (CT)]
Durchzuf\"uhrende Pr\"ufung f\"ur die Schwei{\ss}verbindung in Abh\"angigkeit von der Schwei{\ss}nahtg\"uteklasse.
\end{definition}
\newpage
\begin{definition}[Hersteller]
Organisation, die
\begin{itemize}
		\item eine schwei{\ss}technische Fertigung zur Herstellung und Instandsetzung betreibt oder
		\item geschwei{\ss}te Komponenten konstruiert, einkauft oder vertreibt.
		\end{itemize}
\end{definition}
\begin{definition}[Statische Auslegung]
Dimensionierung von Schwei{\ss}verbindungen, bei der die Kennwerte der statischen Festigkeit eingehalten werden.
\end{definition}
\begin{definition}[Dauerfestigkeitsauslegung]
Dimensionierung von Schwei{\ss}verbindungen, bei der die Kennwerte der Erm\"udungsfestigkeit eingehalten werden.
\end{definition}
\begin{definition}[Ausnutzung der Beanspruchbarkeit]
Verh\"altnis zwischen berechneter Erm\"udungsfestigkeit und der durch den entsprechenden Sicherheitsfaktor abgeglichenen zul\"assigen Erm\"udungsfestigkeit.
\end{definition}
\begin{definition}[Zul\"assige Erm\"udungsfestigkeit]
Maximale Spannung, die unter Ber\"ucksichtigung eines speziellen Faktors f\"ur die Schwei{\ss}verbindung vom eingesetzten Werkstoff aufnehmbar ist.
\end{definition}
\begin{definition}[Sicherheitsbed\"urfnis]
Definiert die Auswirkungen eines Versagens einer einzelnen Schwei{\ss}naht im Hinblick auf die Folgen f\"ur Personen, Einrichtungen und die Umwelt.
\end{definition}
\newpage
\begin{definition}[Arbeitsprobe]
Musterschwei{\ss}verbindungen zum Nachweis der Handfertigkeit des Schwei{\ss}ers oder der bedingungsgem\"a{\ss}en Ausf\"uhrung von Schwei{\ss}verbindungen.
\end{definition}
}



\subsection{Schwei{\ss}nahtklassifizierung nach EN 15085}
\subsectionpage
\frame{\frametitle{Bestimmung der Ausnutzung der Beanspruchbarkeit}
\framesubtitle{}
\begin{itemize}
\item Verh\"altnis $S$ berechneter zu zul\"assiger Spannung der Verbindung
\begin{itemize}
	\item Bezogen auf Dauerfestigkeit
	\item Festigkeitsanforderungen gem\"a{\ss} EN 12663, EN 13749 oder nationalen Normen
	\item Bewertung der Festigkeit nach nationalen Regelwerken, z.B. DVS 1612
	\begin{itemize}
		\item Abh\"angig von Nahtform und Grundwerkstoff
		\item Betrachtung h\"oherfester Werkstoffe konservativ
		\end{itemize}	
	\item Alternativ Dauerversuch m\"oglich
\end{itemize}
\item Bei Berechnung nach Norm:
\begin{itemize}
		\item Beanspruchungszustand Hoch: $S \geq 0{,}9$
		\item Beanspruchungszustand Mittel: $0{,}75 \leq S < 0{,}9$
		\item Beanspruchungszustand Niedrig: $S < 0{,}75$ 
	\end{itemize}
\end{itemize}
}

\frame{\frametitle{Sicherheitsbed\"urfnis der Schwei{\ss}naht}
\framesubtitle{}
\begin{itemize}
\item Versagen einer einzelnen Schwei{\ss}naht f\"uhrt:
\begin{itemize}
	\item zwangsl\"aufig zu Ereignissen mit Personensch\"aden und Versagen der Gesamtfunktion $\rightarrow$ Hoch
	\item m\"oglicherweise zu Ereignissen mit Personensch\"aden und Beeintr\"achtigung der Gesamtfunktion $\rightarrow$ Mittel
	\item zu unwahrscheinlichen Ereignissen mit Personensch\"aden und keiner direkten Beeintr\"achtigung der Gesamtfunktion $\rightarrow$ Niedrig
\end{itemize}
\item Nur Betrachtung Einfachfehler
\end{itemize}
}

\frame{\frametitle{Schwei{\ss}nahtg\"uteklasse}
\framesubtitle{}
\begin{center}
\begin{tabular}{|c|c |c |c |}
\hline
\multirow{2}{*}{Beanspruchungszustand} & \multicolumn{3}{|c|}{Sicherheitsbed\"urfnis} \\ \cline{2-4}
 & Hoch & Mittel & Niedrig \\ \hline
Hoch & CP A & CP B & CP C2 \\ \hline
Mittel & CP B & CP C2 & CP C3 \\ \hline
Niedrig & CP C1 & CP C3 & CP D \\ \hline 
\end{tabular}
\end{center}
\begin{itemize}
	\item CP A: Nur f\"ur voll durchgeschwei{\ss}te und f\"ur \"Uberpr\"ufung w\"ahrend Fertigung und Instandhaltung zug\"angliche Schwei{\ss}n\"ahte
	\item CP B, Sicherheitsbed\"urfnis hoch: CP A: Nur f\"ur voll durchgeschwei{\ss}te und f\"ur \"Uberpr\"ufung w\"ahrend Fertigung und Instandhaltung zug\"angliche Schwei{\ss}n\"ahte
	\item Weitere Erg\"anzungen zu eingeschr\"ankt volumetrisch pr\"ufbaren Schwei{\ss}n\"ahten siehe \cite[Tabelle 2]{en150853}.
\end{itemize}
}

\frame{\frametitle{Schwei{\ss}nahtpr\"ufklassen}
\framesubtitle{}
\begin{center}
\begin{tabular}{|c|c|}
\hline
\multirow{2}{*}{Schwei{\ss}nahtg\"uteklasse} & Schwei{\ss}nahtpr\"ufklasse \\
& (Mindestanforderung) \\ \hline 
 CP A & CT 1 \\ \hline
 CP B & CT 2 \\ \hline
 CP C1 & CT 2 \\ \hline
 CP C2 & CT 3 \\ \hline
 CP C3 & CT 4 \\ \hline
 CP D & CT 4 \\ \hline
\end{tabular}
\end{center}
}

\frame{\frametitle{Schwei{\ss}nahtpr\"ufklassen}
\framesubtitle{}
\begin{center}
\begin{tabular}{|c|c|c|c|}
\hline
\multirow{2}{*}{Schwei{\ss}nahtpr\"ufklasse} & Volumetrisch & Oberfl\"ache & Sichtpr\"ufung \\
& RT oder UT & MT oder PT &  VT\\ \hline 
 CT 1 & 100 \% & 100 \% & 100 \%\\ \hline
 CT 2 & 10 \% & 10 \% & 100 \%\\ \hline
 CT 3 & n/a & n/a & 100 \%\\ \hline
 CT 4 & n/a & n/a & 100 \%\\ \hline
\end{tabular}
\end{center}
\begin{itemize}
	\item Jeweils Mindestanforderungen
	\item CT 3: VT durch Pr\"ufpersonal (damit CP 2 CT 3 \"ahnlich SGK 2.3 nach DIN 6700)
	\item CT 4: VT als Werkerselbstpr\"ufung, Dokumentation nicht erforderlich
	\item Falls volumetrische Pr\"ufung nicht m\"oglich: ersatzweise 100\% Oberfl\"achenpr\"ufung und Arbeitsprobe
\end{itemize}
}

\frame{\frametitle{Zertifizierung der Schwei{\ss}betriebe}
\framesubtitle{}
\begin{itemize}
\item[CL 1:] Schwei{\ss}en der G\"uteklassen CP A bis CP D, Handel und Konstruktion
\item[CL 2:] Schwei{\ss}en der G\"uteklassen CP C2 bis CP D, Handel un Konstruktion nach CL 2 und CL 3 ist eingeschlossen
\item[CL 3:] Schwei{\ss}en der G\"uteklasse CP D
\item[CL 4:] Schwei{\ss}konstruktion von Teilen der CL 1 bis CL 3 sowie Handel mit solchen Teilen
\end{itemize}
}


\subsection{Schwei{\ss}en an Schienenfahrzeugen - Aluminium und Stahl}
\subsectionpage

\frame[allowframebreaks]{\frametitle{Konstruktive Grundlagen}
\framesubtitle{}
\begin{itemize}
\item Vermeiden:
\begin{itemize}
	\item Scharfe Ecken
	\item Querschnitts\"anderungen
	\item Gemischte Verbindungsarten (z.B. Schwei{\ss}en und Schrauben)
	\item Anh\"aufungen von Schwei{\ss}n\"ahten
	\item Quern\"ahte zur Befestigung untergeordneter Teile bei Zugbeanspruchung
\end{itemize}
\item Zug\"anglichkeit zum Schwei{\ss}en und Pr\"ufen gew\"ahrleisten
\item Kaltverformte Bereiche (einschlie{\ss}lich Umgebung $5 t$):
\begin{itemize}
	\item Schwei{\ss}en nur an Teilen der CL 3 zul\"assig
	\item F\"ur CL 1 und CL 2: 
	\begin{itemize}
		\item Normalgl\"uhen 
		\item Eingeschr\"ankte Radien \cite[Tabelle 9]{en150853}
		\end{itemize}
\end{itemize}
\newpage
\item Stumpfn\"ahte an Bauteilen unterschiedlicher Dicke
\begin{itemize}
		\item \"Ubergang mit Neigung:
		\begin{itemize}
		\item CP C3 und CP D: Neigung 1:1
		\item CP A, CP B, CP C1 und CP C2: Neigung 1:4
		\end{itemize}
		\end{itemize}
\item Abstand zwischen Schwei{\ss}n\"ahten: i.d.R. 50 mm
\item Freischnitte sollen Umschwei{\ss}barkeit gew\"ahrleisten
\item Randabstand Kehlnaht $\geq 1{,}5 a + t$
\item Korrosionsschutz.
\begin{itemize}
		\item Umschwei{\ss}en
		\item R\"uckseite abdichten: Dichtschwei{\ss}en, Acryl, ...
		\end{itemize}
\end{itemize}
}

\frame{\frametitle{Angaben auf Schwei{\ss}zeichnungen}
\framesubtitle{}
\begin{itemize}
\item Schwei{\ss}nahtg\"uteklasse
\begin{itemize}
	\item falls einheitlich: in Zeichung
	\item falls unterschiedlich: nahe bei der Schwei{\ss}naht
\end{itemize}
\item Zertifizierungsstufe
\begin{itemize}
	\item je Bauteil
\end{itemize}
\item Schwei{\ss}nahtform
\item Schwei{\ss}nahtdicke
\item Schwei{\ss}nahtl\"ange
\item Schwei{\ss}zus\"atze (in Zeichnungen, St\"ucklisten oder anderen Dokumenten)
\end{itemize}
}



%\offslide{Schwei{\ss}nahtformen 1}
%
%\offslide{Schwei{\ss}nahtformen 2}
%
%\offslide{DVS 1612}

%\subsection{Schwei{\ss}en an Schienenfahrzeugen - Auslastungs- und Festigkeitsberechnungen}
%\subsectionpage


%Schrauben
% !TEX root = SFV-15014_HuV.tex
\section{Schraubenverbindungen}
%\sectionpage

\frame{\frametitle{Schraubenverbindungen an Schienenfahrzeugen - Allgemeines}
\framesubtitle{}
\begin{definition}[Schraubenverbindungen]
Schraubverbindungen erm\"oglichen das l\"osbare Verbinden von Werkst\"ucken mittels Verbindungselementen. Eine Schraubenverbindung umfasst sowohl die Verbindungelemente als auch die zu verbindenden Teile.  
\end{definition}
\textbf{Anwendung in der Schienenfahrzeugtechnik}
\begin{itemize}
	\item Bremszange an Drehgestellrahmen
	\item Radbremsscheibe an Rad
	\item Elektrische Kontakte (z.B. Erdung, Stromversorgung)
\end{itemize}
\textbf{Herausforderungen in der Schienenfahrzeugtechnik}
\begin{itemize}
	\item Lange Produktlebensdauer, hohe Schwingspielzahlen
	\item Hohes Sicherheitsbed\"urfnis
	\item Zertifizierungs- und Pr\"ufaufwand
\end{itemize}
}

\frame{\frametitle{Grundlage: DIN 25201}
\framesubtitle{}
\begin{itemize}
\item Sieben Teile:
\begin{enumerate}
\item Einteilung, Kategorien der Schraubverbindungen
\item Konstruktion - maschinenbauliche Anwendungen
\item Konstruktion - elektrische Anwendungen
\item Sichern von Schraubenverbindungen
\item Korrosionsschutz
\item Anschlussma{\ss}e
\item Montage
\end{enumerate}
\end{itemize}
}

\frame{\frametitle{Risikoklassen der Schraubenverbindungen}
\framesubtitle{}
\begin{itemize}
\item Risikoklasse H (hoch)
	\begin{itemize}
		\item Das Versagen der Schraubenverbindung stellt eine direkte oder indirekte Gefahr f\"ur Leib und Leben dar.
	\end{itemize}
\item Risikoklasse M (mittel)
	\begin{itemize}
		\item Das Versagen der Schraubenverbindung f\"uhrt zu einer Funktionsst\"orung des Fahrzeugs.
	\end{itemize}
\item Risikoklasse G (gering)
	\begin{itemize}
		\item Das Versagen der Schraubenverbindung f\"uhrt maximal zu Komforteinbu{\ss}en f\"ur die Fahrg\"aste oder das Bedienpersonal.
	\end{itemize}
\end{itemize}
}

\frame{\frametitle{Anforderungen DIN 25201}
\framesubtitle{}
         	\begin{center}
            		\includegraphics[width=0.8\textwidth]{25201T1}\source{\cite[Tab. 1]{din252011}}
        		\end{center}
}

%\frame{\frametitle{Beispiele f\"ur Risikoklassen: Hoch}
%\framesubtitle{}
%         	\begin{center}
%            		\includegraphics[width=0.8\textwidth]{25201TA1}\source{\cite[Tab. A.1]{din252011}}
%        		\end{center}
%}
%
%\frame{\frametitle{Beispiele f\"ur Risikoklassen: Mittel}
%\framesubtitle{}
%         	\begin{center}
%            		\includegraphics[width=0.8\textwidth]{25201TA2}\source{\cite[Tab. A.2]{din252011}}
%        		\end{center}
%}
%
%\frame{\frametitle{Beispiele f\"ur Risikoklassen: Gering}
%\framesubtitle{}
%         	\begin{center}
%            		\includegraphics[width=0.8\textwidth]{25201TA3}\source{\cite[Tab. A.3]{din252011}}
%        		\end{center}
%}

\frame{\frametitle{Schraubf\"alle}
\framesubtitle{Verhindern von Klaffen oder Schlupf}
         	\begin{center}
            		\includegraphics[width=0.8\textwidth]{DIN25201A11}\source{\cite[Abb. 1]{din252012}}
        		\end{center}
}

\frame{\frametitle{Schraubf\"alle}
\framesubtitle{Verhindern von Klaffen oder Schlupf}
         	\begin{center}
            		\includegraphics[width=0.8\textwidth]{DIN25201A12}\source{\cite[Abb. 1]{din252012}}
        		\end{center}
}

\frame{\frametitle{Belastung und Versagen der Schrauben}
\framesubtitle{Grundlage: Schraube als schw\"achstes Element}
         	\begin{center}
            		\includegraphics[width=0.8\textwidth]{DIN25201A2}\source{\cite[Abb. 2]{din252012}}
        		\end{center}
}

\frame{\frametitle{Klemml\"ange $l_{K}$}
\framesubtitle{Grundlage: $l_{K} > 3{,}5 d$, damit elastische Verspannung erhalten bleibt}
         	\begin{center}
            		\includegraphics[width=0.8\textwidth]{DIN25201A3}\source{\cite[Abb. 3]{din252012}}
        		\end{center}
}

\frame{\frametitle{Konstruktive Grundlagen der Verschraubung}
\framesubtitle{}
         	\begin{center}
            		\includegraphics[width=0.8\textwidth]{DIN252012T1}\source{\cite[Tab. 1]{din252012}}
        		\end{center}
}

\frame{\frametitle{Haftreibungszahlen in der Trennfuge}
\framesubtitle{}
         	\begin{center}
            		\includegraphics[width=0.8\textwidth]{DIN252012T2}\source{\cite[Tab. 2]{din252012}}
        		\end{center}
}

\frame{\frametitle{Anforderungen an die Verbindung}
\framesubtitle{}
\begin{itemize}
\item Elastische Nachgiebigkeit: Hoch bei Schraube, gering bei verspannte Bauteilen
\item Schrauben und Muttern gleicher Festigkeitsklassen
\item Anzahl Unterlegeteile minimiert
\item Montagewerkzeuge: Innen- bzw Au{\ss}ensechskant oder -sechsrund
\item Metrisches ISO-Regelgewinde
\item Schraubenwerkstoff:
\begin{itemize}
	\item Bevorzugte Festigkeitsklassen: 8.8, A2-70 und A4-80
	\item Festigkeitsklasse 12.9 wird nicht betrachtet (vgl. DB G\"utep\"ufung)
\end{itemize}
\item Oberfl\"achenbeschichtung
\begin{itemize}
	\item Korrosionbest\"andigkeit
	\item Bei hohen Festigkeiten: Waserstoffverspr\"odung vermeiden
	\item Definiertes und enges Reibungszahlfenster
\end{itemize}
\end{itemize}
}

\subsection{Schraubensicherung}
\frame{\frametitle{Schraubensicherung - L\"osen der Schraubenverbindung}
\framesubtitle{}
         	\begin{center}
            		\includegraphics[width=0.7\textwidth]{DIN252014A1}\source{\cite[Abb. 1]{din252014}}
        		\end{center}
}


\frame{\frametitle{Schraubensicherung - Methoden}
\framesubtitle{}
         	\begin{center}
            		\includegraphics[width=0.8\textwidth]{DIN252014T1}\source{\cite[Tab. 1]{din252014}}
        		\end{center}
}
\subsection{Schraubensicherung}
\frame{\frametitle{Schraubensicherung - Sicherungsmittel}
\framesubtitle{}
         	\begin{center}
            		\includegraphics[width=0.9\textwidth]{DIN252014TA1}\source{\cite[Tab. A.1]{din252014}}\\
%        		\end{center}
%}
%\subsection{Schraubensicherung}
%\frame{\frametitle{Schraubensicherung - L\"osen der Schraubenverbindung}
%\framesubtitle{}
%         	\begin{center}
            		\includegraphics[width=0.9\textwidth]{DIN252014TA11}%\source{\cite[Tab. A.1]{din252014}}
        		\end{center}
}




%Materialien und Korrosionsschutz, DB G\"utepr\"ufung
% !TEX root = SFV-15014_HuV.tex
\section{Materialien, Korrosionsschutz, DB G\"utepr\"ufung}
\frame{\sectionpage}

\subsection{Materialien}
\frame{\subsectionpage}

\frame{\frametitle{Anforderungen an Materialien f\"ur Schienenfahrzeuge}
\framesubtitle{}
\begin{itemize}
\item Statische Festigkeit
\item Dauerfestigkeit
\item Gut zu f\"ugen
\item In Abmessungen und Mengen verf\"ugbar
\item Geringes Gewicht (gemessen an der Festigkeit)
\item Best\"andig gegen Umwelteinfl\"usse
\item Recyclingf\"ahig
\item Keine Freisetzung gef\"ahrlicher Substanzen
\item Angemessene Kosten
\item Reparierbarkeit
\item Betriebserfahrung
\item G\"unstiges Brandverhalten
\end{itemize}
}

\frame[allowframebreaks]{\frametitle{Materialien f\"ur Schienenfahrzeuge}
\framesubtitle{}
\begin{itemize}
\item Baustahl (S235, S355):
	\begin{itemize}
		\item Blechst\"arke bis 12 mm
		\item Zus\"atzlich: Tieftemperatureignung
	\end{itemize}
\item Feinkornbaustahl (S500, S690):
	\begin{itemize}
		\item Gewichtsersparnis bei hochbelasteten Teilen
	\end{itemize}
\item Edelst\"ahle
	\begin{itemize}
		\item Korrosionsbest\"andige St\"ahle, z.B. X5CrNi18-10: Rohrleitungen
		\item Verschleisbest\"andige St\"ahle, z.B. X120Mn12: Gleitelemente
	\end{itemize}	
\item Aluminiumwerkstoffe
	\begin{itemize}
		\item Strangpressprofile: AlMgSi
		\item Bleche: AlMg
	\end{itemize}\newpage
\item Gusswerkstoffe
	\begin{itemize}
		\item Grauguss:
		\begin{itemize}
		\item Gusseisen mit Lamellargraphit, z.B. EN-GJL-300: Geh\"ause, Bremsenteile
		\item Gusseisen mit Kugelgraphit, z.B. EN-GJS-500: Zugstangen, Bremszangen, Bremsscheiben
		\item Bainitisches Gusseisen mit Kugelgraphit, z.B. EN-GJ-800/1000: Hochbelastete Bauteile, Bremshebel
		\end{itemize}
		\item Gussstahl, z.B. G18NiMoCr3-6: hochbelastete Teile, Bremsscheiben
		\item Aluminiumguss
	\end{itemize}
	\item Kunststoffe, z.B. PA, PE
	\item Elastomere, z.B. Silikon, Fluorelastomere (Viton)
\end{itemize}
}


\subsection{Korrosionsschutz}
\frame{\subsectionpage}

\frame[allowframebreaks]{\frametitle{Korrosionsschutz}
\framesubtitle{}
\begin{itemize}
\item Aufgaben:
	\begin{itemize}
		\item Lebensdauer und Atmosph\"are belasten Schienenfahrzeugkomponenten extrem
		\item (Extrem-)Beispiel: Kanaltunnelz\"uge, Metro Uijeongbu
	\end{itemize}
\item Anforderungen:
\begin{itemize}
		\item Hohe Feuchtigkeitseintr\"age
		\item Temperaturschwankungen
		\item Metalleintr\"age in Umgebung
		\item Korrosive Substanzen
		\item Schotterflug
		\item Wartbarkeit (Vandalismus)
		\item Reparierbarkeit
		\item \"Asthethik
		\end{itemize} \newpage
	\item Lack:
\begin{itemize}
		\item Prozess:
	\begin{itemize}
		\item Rohbau bzw. Komponentenfertigung
		\item Strahlen und Reinigen
		\item Abkleben (f\"ur Fl\"achen ohne Grundierung)
		\item Grundierung (f\"ur DB gem. TL 918300: $(30\ldots 80)\, \mu \mathrm{m}$ 2K-EP)
		\item Ggf. Zwischenschicht 
		\item Ggf. F\"uller/Spachtel
		\item Decklack (f\"ur DB gem. TL 918300: $(200\ldots 300)\, \mu \mathrm{m}$ 2K-EP)
		\end{itemize}
		\item \"Ublich: Lacksystem der Betreiber, z.B.
		\begin{itemize}
		\item DB: 2-Komponenten EP-Lack
		\item SNCF: PU-Lack
		\end{itemize}
		\end{itemize}
	\item Verzinken
	\item Chromatieren
	\item GEOMET
	\item Pulverbeschichten
	\item Fett
\end{itemize}
}


\subsection{DB G\"utepr\"ufung}
\frame{\subsectionpage}

\frame{\frametitle{Einleitung}
\framesubtitle{DB G\"utepr\"ufung als Beispiel f\"ur bahnspezifische Qualit\"atsforderungen.}
\begin{itemize}
\item Einkaufsvolumen DB AG 2014: 23{,}2 Mrd. EUR
	\begin{itemize}
		\item Industrielle Produkte: 4{,}3 Mrd. EUR
	\end{itemize}
\item Langj\"ahrige Kenntnis qualit\"ats- und sicherheitsrelevanter Aspekte der Produkte
\item \"Ahnlich bei vielen ehemals staatlichen Bahnen
\end{itemize}
\begin{definition}[Qualit\"at]
Qualit\"at ist der Grad, zu welchem Anforderungen an Produkte, Systeme und Dienstleistungen von diesen erf\"ullt werden.
\end{definition}
}

\frame{\frametitle{Zweck der G\"utepr\"ufung}
\framesubtitle{}
\begin{columns}[t] 
     \begin{column}[T]{6cm} 
     	\begin{itemize}
     		\item Zweck:
		\begin{itemize}
		\item Regelung des Umfangs der QS-Ma{\ss}nahmen
		\item Beschaffung f\"ur DB AG und verbundene Unternehmen
		\item Gilt auch f\"ur Unterlieferanten
		\end{itemize}
		\item Fokus auf Sicherheit (und Verf\"ugbarkeit)
		\item Achtung: nur kostenneutral, wenn Bestellung durch die DB vorliegt
     	\end{itemize}
     \end{column}
     	\begin{column}[T]{6cm} 
         	\begin{center}
            		\includegraphics[width=0.8\textwidth]{LgPS1}\source{}
        		\end{center}
     \end{column}
 \end{columns}
}

\frame{\frametitle{Pr\"ufstufen - Lieferanteneinstufung}
\framesubtitle{}
\begin{columns}[t] 
     \begin{column}[T]{6cm} 
     	\begin{itemize}
     		\item Pr\"ufstufe 1:
		\begin{itemize}
		\item Hochsicherheitsrelevante Teile, z.B.
		\begin{itemize}
		\item Fahrzeuge
		\item Bremsscheiben, Bremszylinder
		\end{itemize}
		\end{itemize}
		\item Pr\"ufstufe 2:
		\begin{itemize}
		\item Sicherheitsrelevante Teile, z.B.
		\begin{itemize}
		\item Herzst\"uck Kupplung
		\item Notausstiege
		\end{itemize}
		\end{itemize}
     	\end{itemize}
     \end{column}
     	\begin{column}[T]{6cm} 
         	\begin{itemize}
		\item Ermittelt im Rahmen der Herstellerbezogenen Produktqualifikation ``HPQ'':
		\begin{itemize}
		\item Q1: Stichprobenpr\"ufung f\"ur P1, Herstellerabnahme f\"ur P2
		\item Q2: 100\%-Pr\"ufung f\"ur P1, Stichprobenpr\"ufung f\"ur P2
		\item Q3: 100\%-Pr\"ufung f\"ur alle Lieferungen, Sperrung m\"oglich
		\end{itemize}
		\item F\"ur bestimmte Produkte, darunter Guss- und Schmiedeteile im sicherheitsrelevanten Bereich
		\end{itemize}     
		\end{column}
 \end{columns}
}

\frame{\frametitle{Qualifikationspflichtige Produkte und Fertigungsverfahren}
\framesubtitle{}
\begin{itemize}
\item Produkte
	\begin{itemize}
		\item Rads\"atze und Radsatzteile
		\item Gesenkschmiedeteile aus dem Bereich Zug- und Sto{\ss}einrichtung
		\item Zughaken, Schraubenkupplung
		\item Puffer
		\item Bremsklotzsohlen gegossen
		\item Bremsscheiben
		\item Radsatzlager
		\item Kunststoffk\"afige f\"ur Rollenlager
		\item Sicherheitsglas f\"ur Schienenfahrzeuge
		\item Molybd\"anbeschichtete Radsatzwellen
		\item Guss- und Schmiedeteile im sicherheitsrelevanten Bereich
	\end{itemize}
\item Fertigungsverfahren
	\begin{itemize}
		\item Gie{\ss}en
		\item Schmieden
		\item Pulverbeschichten
		\item Thermisches Spritzen
	\end{itemize}
\end{itemize}
}

\frame{\frametitle{Erstmusterpr\"ufung}
\framesubtitle{}
\begin{itemize}
\item Am ersten unter Serienbedingungen hergestellten Teil
\item Nachweis der Erf\"ullung der (Qualit\"ats-)Anforderungen
\item Erstmusterpr\"ufung durchzuf\"uhren bei:
\begin{itemize}
	\item Erstproduktionen
	\item Produkt\"anderungen
	\item Produktionsverlagerung
	\item \"Anderung von Produktionsverfahren
	\item \"Anderung der Produktions- oder Prozessabl\"aufe
	\item Aussetzen der Produktion mehr als 12 Monate
	\item Neuen Lieferanten
\end{itemize}
\item Vorab durchzuf\"uhren: 
\item Ergebnisse:
\begin{itemize}
	\item Freigabe f\"ur Serienfertigung
	\item Freigabe f\"ur Serienfertigung mit Auflagen
	\item Gesperrt f\"ur Serienfertigung
\end{itemize}
\end{itemize}
}

\frame{\frametitle{Typpr\"ufungen}
\framesubtitle{}
\begin{itemize}
\item Umfang der Typpr\"ufungen in Normen, Spezifikationen oder beh\"ordlich geregelt
\item Nachweis der Konformit\"at mit o.g. Anforderungen
\item Durchf\"uhrung vor Erstbemesterung bzw. Serienfertigung
\item Pr\"ufplan i. d. R. abzustimmen
\item Typnachweis bzw. Typpr\"ufbericht, evtl. mit Bewertung durch Sachverst\"andige
\end{itemize}
}

\frame{\frametitle{Pr\"ufpunkte}
\framesubtitle{}
\begin{itemize}
\item A-Punkt:
\begin{itemize}
	\item Abstimmungspflichtiger Pr\"ufpunkt
	\item Schriftliche Meldung an zust\"andigen Pr\"ufingenieur
	\item Anwesenheit Pr\"ufingenieur verpflichtend
\end{itemize}
\item F-Punkt:
\begin{itemize}
		\item Meldepunkt
		\item Schriftliche Meldung an zust\"andigen Pr\"ufingenieur
		\item Anwesenheit Pr\"ufingenieur optional (Entscheidung Pr\"ufingenieur)
		\item Pr\"ufung am n\"achstm\"oglichen Pr\"ufpunkt in Anwesenheit des Pr\"ufingeniurs nachholen
\end{itemize}
\item S-Punkt:
\begin{itemize}
		\item Stichprobenpr\"ufung
		\item Schriftliche Meldung an zust\"andigen Pr\"ufingenieur
		\item Anwesenheit Pr\"ufingenieur optional (Entscheidung Pr\"ufingenieur)
		\item Pr\"ufung muss nicht vom Pr\"ufingenieur \"uberwacht werden
\end{itemize}
\end{itemize}
}

\frame{\frametitle{}
\framesubtitle{}
         	\begin{center}
            		\includegraphics[width=0.95\textwidth]{LgPS19}\source{Quelle: \cite[S. 19]{dblgp}}
        		\end{center}
 }
 
 \frame{\frametitle{DB Richtlinie 951}
\framesubtitle{DB Richtlinie 951 klassifiziert Schwei{\ss}n\"ahte stenger als EN 15085.}
\begin{itemize}
\item DB-Gruppe 1:
	\begin{itemize}
		\item z.B. Drehgestellrahmen, Untergestell, Zug- und Sto{\ss}einrichtung, Schwingungs- und Sto{\ss}d\"ampfer
		\item Schwei{\ss}technische Bauweisepr\"ufung Teil 1 und 2 erforderlich, CL 1 nach EN 15085
	\end{itemize}
\item DB-Gruppe 2:
	\begin{itemize}
		\item z.B. Einstiegst\"uren, Drehgestellanbauten, Kabelkupplungen an automatischen Kupplungen
		\item Schwei{\ss}technische Bauweisenpr\"ufung Teil 1 erforderlich, CL 1 nach EN 15085
	\end{itemize}
\item DB-Gruppe 3:
	\begin{itemize}
		\item z.B. Innenausbau, Tragrahmen innen, WC-Bauteile und Wasserbeh\"alter
		\item Schwei{\ss}technische Bauweisenpr\"ufung nicht erforderlich, CL 2 nach EN 15085
	\end{itemize}
\item DB-Gruppe 4:
	\begin{itemize}
		\item z.B. Halter f\"ur Schilder, Tritte, Griffe, Gel\"ander innen 
		\item Schwei{\ss}technische Bauweisepr\"ufung nicht erforderlich, CL 3 nach EN 15085
	\end{itemize}
\end{itemize}
}


\frame{\frametitle{Zeugnistypen gem\"a{\ss} EN 10204}
\framesubtitle{Die EN 10204 wird h\"aufig zur Spezifikation der Zeugnistypen (``Pr\"ufbescheinigung'') herangezogen.}
\small
         	\begin{center}
            		\begin{tabular}{|l|p{2cm}|p{5.5cm}|p{2.5cm}|} 
			\hline 
			Art & Bezeichnung & Inhalt & Erstellt durch\\ \hline
			2.1 & Werksbe-scheinigung & Bestätigung der Übereinstimmung mit der Bestellung& Hersteller \\ \hline
			2.2 & Werkszeugnis & Bestätigung der Übereinstimmung mit der Bestellung unter Angabe von Ergebnissen nichtspezifischer Prüfung& Hersteller \\ \hline
			3.1 & Abnahme-pr\"ufzeugnis 3.1 & Bestätigung der Übereinstimmung mit der Bestellung unter Angabe von Ergebnissen spezifischer Prüfung& Unabh\"angige Stelle des Herstellers \\ \hline
			3.2 & Abnahme-pr\"ufzeugnis 3.2 & Bestätigung der Übereinstimmung mit der Bestellung unter Angabe von Ergebnissen spezifischer Prüfung& Unabh\"angige Stelle des Herstellers und Abnehmer des Kunden o.\"a.\\ \hline
			\end{tabular}
        		\end{center}
    }



%% IBS und Abhname:
% !TEX root = SFV-15014_HuV.tex
\section{Inbetriebsetzung}
\frame{\sectionpage}

\subsection{Root-Cause-Analysis}
\frame{\subsectionpage}

\frame{\frametitle{Aufgaben der Root-Cause-Analysis}
\framesubtitle{}
\begin{itemize}
\item Im Verlauf der IBS h\"aufig:
\begin{itemize}
		\item Systematische Abweichungen
		\item Ausf\"alle
		\item Kundenbeschwerden
		\end{itemize}
\item Typisch: nur Symptome werden geschildert
	\begin{itemize}
		\item Auftreten sporadisch
		\item Zugang zum Fahrzeug eingeschr\"ankt
		\item Nicht reproduzierbar
		\item Keine Referenzsystem vorhanden
	\end{itemize}
\item Problem: vorgefertigte Meinungen zur Ursache
\item M\"oglichkeiten:
	\begin{itemize}
		\item Ishikawa-Diagramm
		\item 5-Why
	\end{itemize}
\end{itemize}
}


\frame{\frametitle{Ishikawa-Diagramm}
\framesubtitle{Hauptkategorien im Ishikawa-Diagramm variieren nach Anwendungskontext}
	\begin{center}
            		\includegraphics[width=0.95\textwidth]{Ishikawa}
        	\end{center}
}

\frame{\frametitle{5-Why}
\framesubtitle{}
\begin{itemize}
\item Einfaches, direktes Fragen nach der Ursache f\"uhrt h\"aufig nicht zur Root-Cause
\item Wiederholtes Fragen kommt ``tiefer''
\item Generell akzeptiert sind 5 Why
\end{itemize}
}





\frame[allowframebreaks]{\frametitle{Literatur}
\framesubtitle{}
\nocite{kleinaltenkamp}
\nocite{kanitzky}
\nocite{felkai}
\nocite{en150853, en150854, en150855, din252011, din252012, din252014, dvs1612}
\bibliographystyle{plainnat}
\bibliography{../../../bib}
}


\end{document}
