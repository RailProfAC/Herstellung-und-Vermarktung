\documentclass[10pt,a4paper,headsepline,smallheadings]{scrartcl}
\usepackage[utf8]{inputenc}
\usepackage[T1]{fontenc}
\usepackage[ngerman]{babel}
\usepackage{amsmath}
\usepackage{amsthm}
\usepackage{amssymb}
\usepackage{amsfonts}
\usepackage[scaled]{helvet}
\usepackage{amssymb}
\usepackage{multirow}
\usepackage{textcomp}
\usepackage{graphicx}
\usepackage{paralist}
\usepackage{textcomp}
\usepackage{pdflscape} 
\usepackage{marvosym}
\usepackage{float}
\usepackage{siunitx}
\usepackage[siunitx,european,cuteinductors,smartlabels]{circuitikz}
\usepackage{fancyhdr}
\usepackage{pgfplots}
\usepackage{sansmath}
\usepackage{lscape}
\usepackage{multicol}

\usetikzlibrary{calc}


\theoremstyle{definition}
\newtheorem{aufgabe}{Aufgabe}


\renewcommand*\familydefault{\sfdefault}
\renewcommand{\arraystretch}{1.1}


\KOMAoptions{parskip=half,DIV=15,fontsize=11pt}
\unitlength1cm
\titlehead{

\begin{center}\begin{tabular}{p{10cm}p{6.8cm}}
\textbf{FH Aachen} & \textbf{FB Maschinenbau und Mechatronik}\\[0.5cm]
\textbf{Modul 86512} &  Prof. Dr. Raphael Pfaff\\
Herstellung und Vermarktung& Sommersemester 2015\\
\end{tabular}

\end{center}
\begin{picture}(0,0)(0,0)\put(17,-23){\includegraphics[height=5cm]{fh_logo}}\end{picture}
}
\newif\ifuelsg %als slides
%\uelsgtrue
\uelsgfalse
\newif\ifnotuelsg
\ifuelsg\notuelsgfalse\else\notuelsgtrue\fi
\graphicspath{
{../Bilder/Uebungen/}
{../Bilder/Wirkungsplan/}
}

%\titlehead{83105 \hfill Mess-Steuerungs- und Regelungstechnik \hfill Prof. Manfred Enning}
\title{Herstellung und Vermarktung von Schienenfahrzeugen  -- \"Ubung 2}
\date{}
\makeatletter
\let\Title\@title
\let\Author\@author
\makeatother
\pagestyle{fancy}
\fancyhead[LE, LO]{Prof. Dr. Raphael Pfaff}
\fancyhead[RE, RO]{\Title}

\begin{document}
\thispagestyle{empty}
\maketitle
\vspace{-2cm}
% \hyphenation{Abwei-chungen}

\section*{Schwei{\ss}nahtg\"uteklassen}

\begin{aufgabe}[Sicherheitsbed\"urfnis. Gruppenarbeit 2-3] 
Bewerten Sie die Sicherheitsbed\"urfnisse der Schwei{\ss}n\"ahte der unten in Seitenansicht und Draufsicht gezeigten Kupplung.

\begin{enumerate}[a)]
\item F\"ugen\"ahte Elektrokupplung (Pos. 006) (Kehlnaht a3)
\item Heftn\"ahte Kegel an Kuppelkopf (Pos. 001) (4 x Kehlnaht a4)
\item Verbindung Kuppelkopf-F\"uhrung Elektrokupplung (Pos. 001, 025) (Kehlnaht a4)
\item Verbindung Karabiner-Anschraubplatte (an Pos. 001) (Kehlnaht a3)
\item Fertigungsschwei{\ss}en an Schalenmuffe (Pos. 004)
\item Verbindung Verformungsrohr zu Flansch (Pos. 003) (Steilflankennaht 23mm)
\item F\"ugen\"ahte und Anschraubteile f\"ur HLL-Rohr (an Pos. 001) (Kehlnaht a3)
\item Anlenkteller an Zugstangengeh\"ause (Pos. 002) (V-Naht 15 mm)
\end{enumerate}

\end{aufgabe}
\vspace{.5cm}

\begin{aufgabe}[Beanspruchungszustand. Gruppenarbeit 2-3] 
Bewerten Sie die Beanspruchungszust\"ande der Schwei{\ss}n\"ahte der unten in Seitenansicht und Draufsicht gezeigten Kupplung. Bei welchen Schwei{\ss}n\"ahten w\"urden Sie auf eine Berechnung verzichten?

\begin{enumerate}[a)]
\item F\"ugen\"ahte Elektrokupplung (Pos. 006) (Kehlnaht a3)
\item Heftn\"ahte Kegel an Kuppelkopf (Pos. 001) (4 x Kehlnaht a4)
\item Verbindung Kuppelkopf-F\"uhrung Elektrokupplung (Pos. 001, 025) (Kehlnaht a4)
\item Verbindung Karabiner-Anschraubplatte (an Pos. 001) (Kehlnaht a3)
\item Fertigungsschwei{\ss}en an Schalenmuffe (Pos. 004)
\item Verbindung Verformungsrohr zu Flansch (Pos. 003) (Steilflankennaht 23mm)
\item F\"ugen\"ahte und Anschraubteile f\"ur HLL-Rohr (an Pos. 001) (Kehlnaht a3)
\item Anlenkteller an Zugstangengeh\"ause (Pos. 002) (V-Naht 15 mm)
\end{enumerate}

\end{aufgabe}

\begin{aufgabe}[Beanspruchungszustand. Gruppenarbeit 2-3] 
Bewerten Sie die Beanspruchungszust\"ande der Schwei{\ss}n\"ahte der unten in Seitenansicht und Draufsicht gezeigten Kupplung. Bei welchen Schwei{\ss}n\"ahten w\"urden Sie auf eine Berechnung verzichten? Sch\"atzen Sie die wirkenden Kr\"afte ab.

\begin{enumerate}[a)]
\item F\"ugen\"ahte Elektrokupplung (Pos. 006) (Kehlnaht a3)
\item Heftn\"ahte Kegel an Kuppelkopf (Pos. 001) (4 x Kehlnaht a4)
\item Verbindung Kuppelkopf-F\"uhrung Elektrokupplung (Pos. 001, 025) (Kehlnaht a4)
\item Verbindung Karabiner-Anschraubplatte (an Pos. 001) (Kehlnaht a3)
\item Fertigungsschwei{\ss}en an Schalenmuffe (Pos. 004)
\item Verbindung Verformungsrohr zu Flansch (Pos. 003) (Steilflankennaht 23mm)
\item F\"ugen\"ahte und Anschraubteile f\"ur HLL-Rohr (an Pos. 001) (Kehlnaht a3)
\item Anlenkteller an Zugstangengeh\"ause (Pos. 002) (V-Naht 15 mm)
\end{enumerate}

\end{aufgabe}
\vspace{0.5cm}
\begin{aufgabe}[Schwei{\ss}nahtg\"uteklassen und -pr\"ufklassen. Gruppenarbeit 2-3] 
Nehmen Sie f\"ur alle Schwei{\ss}n\"ahte eine geringe Ausnutzung der Festigkeit an. Legen Sie die Schwei{\ss}nahtg\"uteklassen und -pr\"ufklassen fest. \"Uber welche Zertifizierung muss ein Zulieferer verf\"ugen, wenn er diese Teile fertigt?

\begin{enumerate}[a)]
\item F\"ugen\"ahte Elektrokupplung (Pos. 006) (Kehlnaht a3)
\item Heftn\"ahte Kegel an Kuppelkopf (Pos. 001) (4 x Kehlnaht a4)
\item Verbindung Kuppelkopf-F\"uhrung Elektrokupplung (Pos. 001, 025) (Kehlnaht a4)
\item Verbindung Karabiner-Anschraubplatte (an Pos. 001) (Kehlnaht a3)
\item Fertigungsschwei{\ss}en an Schalenmuffe (Pos. 004)
\item Verbindung Verformungsrohr zu Flansch (Pos. 003) (Steilflankennaht 23mm)
\item F\"ugen\"ahte und Anschraubteile f\"ur HLL-Rohr (an Pos. 001) (Kehlnaht a3)
\item Anlenkteller an Zugstangengeh\"ause (Pos. 002) (V-Naht 15 mm)
\end{enumerate}

\end{aufgabe}
\newpage
\section*{Festigkeitsberechnung}
\begin{aufgabe}[Festigkeitsberechnung] 
Die dynamischen Druck- und Zuglasten auf die Kupplung betragen 200 kN. Bestimmen Sie die Auslastung gem\"a{\ss} DVS 1612 f\"ur folgende Schwei{\ss}n\"ahte:
 \begin{enumerate}[a)]
\item Verbindung Verformungsrohr zu Flansch (Pos. 003)
\begin{itemize}
		\item Steilflankennaht 23mm
		\item 350 mm lang
		\item Verbleibende Badsicherung
		\item \"Ubertr\"agt nur Drucklasten
\end{itemize}
\item Anlenkteller an Zugstangengeh\"ause (Pos. 002)
\begin{itemize}
		\item V-Naht 15mm
		\item durchgeschwei{\ss}t mit Gegenlage
		\item Nahtwinkel 30$^\circ$
		\item 400 mm lang
		\item Verbleibende Badsicherung
		\item \"Ubertr\"agt Zug- und Drucklasten
\end{itemize}

\end{enumerate}
\end{aufgabe}


\begin{landscape}
\begin{center}
\includegraphics[width = 1.3\textwidth]{DML1}
\newpage
\includegraphics[width = 1.3\textwidth]{DML2}
\end{center}
\end{landscape}
\end{document}